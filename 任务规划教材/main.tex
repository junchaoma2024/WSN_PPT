% !TEX program = xelatex
% 任务规划教材 - 主文件
% 编译命令: xelatex main.tex && xelatex main.tex (需要编译两次以生成正确的目录)

\documentclass[12pt,a4paper,openany]{ctexbook}

% ==================== 宏包引入 ====================
\usepackage{geometry}
\usepackage{graphicx}
\usepackage{amsmath,amssymb,amsthm}
\usepackage{algorithm2e}
\usepackage{listings}
\usepackage[dvipsnames,svgnames,x11names]{xcolor}
\usepackage{hyperref}
\usepackage{booktabs}
\usepackage{longtable}
\usepackage{enumitem}
\usepackage{fancyhdr}
\usepackage{titlesec}
\usepackage{tocloft}
\usepackage{float}
\usepackage{subcaption}
\usepackage{tikz}
\usetikzlibrary{shapes,arrows,positioning,calc}

% ==================== 页面设置 ====================
\geometry{
    left=2.5cm,
    right=2.5cm,
    top=3cm,
    bottom=3cm,
    headheight=15pt
}

% ==================== 中文设置 ====================
\ctexset{
    part/format = \Huge\bfseries\centering,
    chapter/format = \LARGE\bfseries,
    chapter/beforeskip = 20pt,
    chapter/afterskip = 20pt,
    section/format = \Large\bfseries,
    subsection/format = \large\bfseries,
    subsubsection/format = \normalsize\bfseries
}

% ==================== 页眉页脚 ====================
\pagestyle{fancy}
\fancyhf{}
\fancyhead[LE,RO]{\thepage}
\fancyhead[LO]{\leftmark}
\fancyhead[RE]{任务规划}
\renewcommand{\headrulewidth}{0.4pt}

% ==================== 定理环境 ====================
\theoremstyle{definition}
\newtheorem{definition}{定义}[chapter]
\newtheorem{example}{示例}[chapter]
\newtheorem{theorem}{定理}[chapter]
\newtheorem{lemma}{引理}[chapter]
\newtheorem{corollary}{推论}[chapter]
\newtheorem{proposition}{命题}[chapter]
\newtheorem{remark}{注记}[chapter]
\newtheorem{algorithm_box}{算法}[chapter]

% ==================== 代码样式 ====================
\lstset{
    basicstyle=\small\ttfamily,
    keywordstyle=\color{blue}\bfseries,
    commentstyle=\color{green!60!black},
    stringstyle=\color{orange},
    numbers=left,
    numberstyle=\tiny\color{gray},
    numbersep=5pt,
    breaklines=true,
    frame=single,
    backgroundcolor=\color{gray!10},
    tabsize=4,
    showspaces=false,
    showstringspaces=false
}

% ==================== PDDL语言定义 ====================
\lstdefinelanguage{PDDL}{
    keywords={define, domain, problem, :requirements, :types, :constants, :predicates, :action, :parameters, :precondition, :effect, :objects, :init, :goal, and, or, not, when, forall, exists, either},
    sensitive=false,
    morecomment=[l]{;},
    morestring=[b]",
}

% ==================== 超链接设置 ====================
\hypersetup{
    colorlinks=true,
    linkcolor=blue!70!black,
    citecolor=green!60!black,
    urlcolor=blue!80!black,
    bookmarksnumbered=true,
    pdfstartview=FitH
}

% ==================== 算法设置 ====================
\SetAlgorithmName{算法}{算法}{算法列表}
\SetKwInput{KwIn}{输入}
\SetKwInput{KwOut}{输出}
\SetKwInput{KwData}{数据}
\SetKwInput{KwResult}{结果}
\SetKw{KwRet}{返回}
\SetKw{KwTo}{到}
\SetKwBlock{Begin}{开始}{结束}
\SetKwFor{For}{for}{do}{end for}
\SetKwFor{ForEach}{foreach}{do}{end foreach}
\SetKwIF{If}{ElseIf}{Else}{if}{then}{else if}{else}{end if}
\SetKwSwitch{Switch}{Case}{Other}{switch}{do}{case}{otherwise}{end case}{end switch}
\SetKwRepeat{Repeat}{repeat}{until}
\SetKwFor{While}{while}{do}{end while}

% ==================== 自定义命令 ====================
\newcommand{\keyterm}[1]{\textbf{#1}}
\newcommand{\englishterm}[1]{\textit{#1}}
\newcommand{\grad}{\texorpdfstring{\textcolor{blue!70!black}{$\bigstar$}}{★}} % 研究生拓展标记

% ==================== 文档信息 ====================
\title{
    \vspace{2cm}
    {\Huge\bfseries 任务规划} \\[0.5cm]
    {\Large Task Planning} \\[2cm]
    {\large 理论、方法与应用}
}
\author{
    \Large 编著者 \\[1cm]
    \large (待填写)
}
\date{\today}

% ==================== 正文开始 ====================
\begin{document}

% 封面
\maketitle

% 版权页
\thispagestyle{empty}
\vspace*{\fill}
\begin{center}
    \textbf{内部资料,仅供教学使用}\\[1cm]
    版权所有 \copyright\ \the\year \\
    未经许可,不得翻印
\end{center}
\vspace*{\fill}
\clearpage

% 前言
\frontmatter
\chapter*{前言}
\addcontentsline{toc}{chapter}{前言}

任务规划(Task Planning)是人工智能领域的核心研究方向之一,它研究如何让智能系统自主地制定行动序列以实现特定目标。从早期的STRIPS系统到现代的大语言模型驱动规划,任务规划理论与技术经历了半个多世纪的发展,已广泛应用于机器人、物流、交通、军事、游戏等众多领域。

本书旨在为本科生和研究生提供一本系统、全面的任务规划教材。全书共分四篇十六章:

\begin{itemize}
    \item \textbf{第一篇:基础篇}(第1--6章):介绍任务规划的基本概念、状态空间搜索、经典规划理论、启发式方法、时态规划和层次任务网络规划,构建扎实的理论基础。

    \item \textbf{第二篇:方法篇}(第7--10章):深入讨论约束满足与规划、不确定性规划、多智能体规划和运动规划等进阶方法。

    \item \textbf{第三篇:应用篇}(第11--13章):通过交通运输规划、通信与编码传输规划、军事任务规划三个应用领域,展示任务规划技术的实际应用。

    \item \textbf{第四篇:前沿篇}(第14--16章,标注\grad):介绍基于学习的规划方法、大语言模型与任务规划、规划系统的可解释性与安全性等前沿研究方向,适合研究生深入学习。
\end{itemize}

本书特色:
\begin{enumerate}
    \item \textbf{理论与实践并重}:既有严谨的数学定义和算法分析,又有丰富的应用示例。
    \item \textbf{示例驱动}:每章包含多个来自交通、通信、军事等领域的详细示例。
    \item \textbf{分层设计}:基础内容面向本科生,前沿内容(带\grad 标记)面向研究生。
    \item \textbf{配套资源}:提供PDDL参考手册、规划器使用指南和编程实验指导。
\end{enumerate}

\vspace{1cm}
\begin{flushright}
    编者\\
    \the\year 年
\end{flushright}

% 目录
\tableofcontents

% 正文
\mainmatter

% ==================== 第一篇:基础篇 ====================
\part{基础篇}

% 第1章 绪论
\chapter{绪论}
\label{chap:introduction}

\section{任务规划的基本概念}

\keyterm{任务规划}(Task Planning)是人工智能领域的核心研究方向之一,它研究如何让智能系统自主地制定行动序列以实现特定目标。简单来说,任务规划回答的是"做什么"和"怎么做"的问题。

\begin{definition}[任务规划问题]
    一个任务规划问题可以形式化地定义为一个四元组 $\mathcal{P} = \langle S, A, I, G \rangle$,其中:
    \begin{itemize}
        \item $S$ 是状态空间,表示系统所有可能状态的集合;
        \item $A$ 是动作集合,表示智能体可执行的所有动作;
        \item $I \in S$ 是初始状态;
        \item $G \subseteq S$ 是目标状态集合。
    \end{itemize}
    规划的任务是找到一个动作序列 $\pi = \langle a_1, a_2, \ldots, a_n \rangle$,使得从初始状态 $I$ 出发,依次执行这些动作后,系统能够到达某个目标状态 $g \in G$。
\end{definition}

任务规划与人类的日常决策密切相关。当我们计划一次旅行、安排一天的工作、或者思考如何完成一个复杂项目时,我们都在进行某种形式的规划。人工智能领域的任务规划研究旨在将这种能力赋予计算机系统。

\section{任务规划的发展历程}

任务规划的研究始于20世纪60年代末,至今已经历了半个多世纪的发展。

\subsection{早期探索(1969--1980)}

1969年,斯坦福研究院(Stanford Research Institute)的研究人员开发了\keyterm{STRIPS}(Stanford Research Institute Problem Solver)系统,这是第一个具有重要影响力的自动规划系统。STRIPS引入了用前提条件和效果来描述动作的方法,这一表示方法至今仍是规划领域的基础。

同一时期,\keyterm{积木世界}(Blocks World)成为规划研究的标准测试问题。在这个问题中,机器人需要通过移动积木来实现特定的堆叠配置。

\subsection{经典规划时期(1980--1995)}

20世纪80年代至90年代中期,研究者们发展了多种规划方法:

\begin{itemize}
    \item \textbf{偏序规划}(Partial-Order Planning):允许规划中的动作保持部分有序,提高了规划的灵活性。
    \item \textbf{层次任务网络规划}(HTN Planning):通过任务分解的方式,将复杂任务分解为简单子任务。
    \item \textbf{规划图方法}:Blum和Furst于1995年提出的GraphPlan算法,通过构建规划图来加速规划过程。
\end{itemize}

\subsection{现代规划时期(1995--2015)}

这一时期的重要进展包括:

\begin{itemize}
    \item \textbf{PDDL语言}:1998年,为了统一规划问题的描述,研究者们制定了\keyterm{规划领域定义语言}(PDDL),并成为国际规划竞赛的标准语言。
    \item \textbf{启发式搜索规划}:FF规划器和Fast Downward系统的出现,使得规划器能够高效处理大规模问题。
    \item \textbf{SAT规划}:将规划问题转换为布尔可满足性问题,利用SAT求解器的强大能力。
\end{itemize}

\subsection{智能时代(2015--至今)}

近年来,深度学习和大语言模型的发展为任务规划带来了新的机遇:

\begin{itemize}
    \item \textbf{强化学习与规划}:AlphaGo等系统展示了学习与规划结合的强大能力。
    \item \textbf{神经符号规划}:将神经网络的学习能力与符号规划的推理能力相结合。
    \item \textbf{大语言模型规划}:探索利用GPT、Claude等大语言模型进行任务规划。
\end{itemize}

\section{任务规划的应用领域}

任务规划技术已广泛应用于多个领域,本节通过三个典型示例来说明。

\begin{example}[交通物流调度问题]
\label{ex:logistics}
    某快递公司需要为5辆配送车规划当日的配送路线。已知:
    \begin{itemize}
        \item 配送中心位于城市中心,坐标为$(0, 0)$;
        \item 共有50个配送点,每个点有一定的货物需求量;
        \item 每辆车的载重限制为2吨,行驶里程限制为200公里;
        \item 部分配送点有时间窗要求(如"上午9点前送达")。
    \end{itemize}

    这是一个典型的\keyterm{带时间窗的车辆路径问题}(VRPTW)。规划系统需要确定:
    \begin{enumerate}
        \item 每辆车访问哪些配送点(任务分配);
        \item 每辆车的访问顺序(路径规划);
        \item 何时到达每个配送点(调度)。
    \end{enumerate}

    目标是在满足所有约束的前提下,最小化总行驶距离或总配送时间。
\end{example}

\begin{example}[机器人任务执行]
\label{ex:robot}
    考虑一个家庭服务机器人,用户发出指令:"把客厅桌上的杯子放到厨房水槽里。"

    机器人需要规划以下动作序列:
    \begin{enumerate}
        \item 导航到客厅;
        \item 定位桌子和杯子;
        \item 移动到桌子前的合适位置;
        \item 伸出机械臂抓取杯子;
        \item 导航到厨房(同时保持杯子稳定);
        \item 定位水槽;
        \item 将杯子放入水槽。
    \end{enumerate}

    这涉及到\keyterm{任务与运动规划}(TAMP)的结合:高层的任务规划决定"做什么",底层的运动规划决定"怎么移动"。
\end{example}

\begin{example}[军事作战任务分配]
\label{ex:military}
    在一次无人机侦察任务中,指挥中心需要为10架无人机分配侦察任务。已知:
    \begin{itemize}
        \item 有20个需要侦察的目标区域;
        \item 每架无人机的续航时间为4小时;
        \item 不同目标区域的优先级不同;
        \item 存在禁飞区和威胁区域;
        \item 部分目标需要多架无人机协同侦察。
    \end{itemize}

    规划系统需要考虑:
    \begin{enumerate}
        \item 目标分配:哪些无人机负责哪些目标;
        \item 航迹规划:每架无人机的飞行路线;
        \item 时间协调:多机协同侦察的时间安排;
        \item 应急处理:遇到突发情况时的重规划。
    \end{enumerate}

    这是\keyterm{多智能体任务规划}与\keyterm{路径规划}的综合问题。
\end{example}

\section{任务规划与相关学科的关系}

任务规划是一个跨学科的研究领域,与多个学科密切相关:

\begin{itemize}
    \item \textbf{人工智能}:任务规划是AI的核心研究方向之一,与知识表示、推理、搜索等紧密相关。

    \item \textbf{运筹学}:许多规划问题可以建模为优化问题,运筹学提供了丰富的求解方法。

    \item \textbf{控制理论}:在机器人规划中,控制理论为运动执行提供了理论基础。

    \item \textbf{计算机科学}:算法设计、复杂性理论为规划提供了分析工具。

    \item \textbf{认知科学}:人类规划行为的研究为人工规划系统提供了启发。
\end{itemize}

图\ref{fig:planning-relations}展示了任务规划与相关学科的关系。

\begin{figure}[htbp]
    \centering
    \begin{tikzpicture}[
        box/.style={rectangle, draw, rounded corners, minimum width=2.5cm, minimum height=1cm, align=center},
        arrow/.style={->, >=stealth, thick}
    ]
        \node[box, fill=blue!20] (planning) at (0,0) {任务规划};
        \node[box, fill=green!20] (ai) at (-3,2) {人工智能};
        \node[box, fill=green!20] (or) at (3,2) {运筹学};
        \node[box, fill=yellow!20] (control) at (-3,-2) {控制理论};
        \node[box, fill=yellow!20] (cs) at (3,-2) {计算机科学};
        \node[box, fill=orange!20] (cognitive) at (0,3) {认知科学};

        \draw[arrow, <->] (planning) -- (ai);
        \draw[arrow, <->] (planning) -- (or);
        \draw[arrow, <->] (planning) -- (control);
        \draw[arrow, <->] (planning) -- (cs);
        \draw[arrow, <->] (planning) -- (cognitive);
    \end{tikzpicture}
    \caption{任务规划与相关学科的关系}
    \label{fig:planning-relations}
\end{figure}

\section{本书结构与学习指南}

本书共分四篇十六章,内容安排如下:

\subsection{内容结构}

\begin{description}
    \item[第一篇:基础篇(第1--6章)] 建立任务规划的理论基础,包括状态空间搜索、经典规划理论、启发式方法、时态规划和层次任务网络规划。这部分内容是后续学习的基础,建议按顺序学习。

    \item[第二篇:方法篇(第7--10章)] 介绍进阶的规划方法,包括约束满足、不确定性规划、多智能体规划和运动规划。这些章节相对独立,可以根据兴趣选择性学习。

    \item[第三篇:应用篇(第11--13章)] 通过三个应用领域展示任务规划技术的实际应用。建议在学完基础篇后,选择感兴趣的应用领域深入学习。

    \item[第四篇:前沿篇(第14--16章)] 介绍最新的研究进展,适合研究生深入学习。这些章节标有\grad 符号。
\end{description}

\subsection{学习建议}

\begin{enumerate}
    \item \textbf{理论与实践结合}:在学习理论知识的同时,动手完成附录D中的编程实验。

    \item \textbf{循序渐进}:先掌握基础篇的内容,再学习进阶内容。

    \item \textbf{关注示例}:每章的示例都经过精心设计,建议仔细研读。

    \item \textbf{完成习题}:每章末尾的习题有助于检验学习效果。

    \item \textbf{查阅参考文献}:对感兴趣的主题,可以通过参考文献进一步深入学习。
\end{enumerate}

\section*{本章小结}

本章介绍了任务规划的基本概念、发展历程、应用领域以及与相关学科的关系。任务规划研究如何让智能系统自主制定行动序列以实现目标,是人工智能领域的核心研究方向。从1969年的STRIPS系统到现代的大语言模型规划,任务规划技术不断发展,应用领域日益广泛。

\section*{习题}

\begin{enumerate}
    \item 试用自己的语言解释什么是任务规划,并举出日常生活中的两个任务规划例子。

    \item 对于示例\ref{ex:logistics}中的快递配送问题,讨论以下问题:
    \begin{enumerate}
        \item 如果取消时间窗约束,问题会变得更简单还是更复杂?为什么?
        \item 如果允许配送点之间的货物中转,会对问题产生什么影响?
    \end{enumerate}

    \item 查阅资料,了解STRIPS系统的历史,回答:
    \begin{enumerate}
        \item STRIPS的名称是什么的缩写?
        \item STRIPS最初用于什么应用场景?
        \item STRIPS表示方法的核心思想是什么?
    \end{enumerate}

    \item 思考题:你认为任务规划与机器学习有什么区别和联系?

    \item 编程练习:用你熟悉的编程语言,实现一个简单的"积木世界"问题的状态表示,包括:
    \begin{enumerate}
        \item 定义状态的数据结构;
        \item 实现检查某个积木是否在另一个积木上方的函数;
        \item 实现打印当前状态的函数。
    \end{enumerate}
\end{enumerate}

% 第2章 状态空间与搜索基础
\chapter{状态空间与搜索基础}
\label{chap:search}

状态空间搜索是任务规划的理论基础。本章将介绍状态空间的形式化表示方法,以及各种搜索算法。

\section{状态空间表示}

\subsection{状态的形式化定义}

\begin{definition}[状态]
    状态(State)是对系统在某一时刻所处情况的完整描述。在规划问题中,状态通常用一组\keyterm{状态变量}(State Variables)或\keyterm{命题}(Propositions)来表示。
\end{definition}

以积木世界为例,假设有三个积木A、B、C和一个桌面Table。我们可以用以下谓词来描述状态:
\begin{itemize}
    \item $\text{On}(x, y)$:积木$x$在$y$上面
    \item $\text{OnTable}(x)$:积木$x$在桌面上
    \item $\text{Clear}(x)$:积木$x$上面没有其他积木
    \item $\text{Holding}(x)$:机械手正在抓取积木$x$
    \item $\text{ArmEmpty}$:机械手是空的
\end{itemize}

一个具体的状态可以表示为这些谓词的集合:
\begin{equation}
    s_0 = \{\text{OnTable}(A), \text{On}(B, A), \text{Clear}(B), \text{OnTable}(C), \text{Clear}(C), \text{ArmEmpty}\}
\end{equation}

\subsection{动作与状态转移}

\begin{definition}[动作]
    动作(Action)是智能体可以执行的操作,它将系统从一个状态转移到另一个状态。一个动作$a$可以用三元组$\langle \text{Pre}(a), \text{Add}(a), \text{Del}(a) \rangle$来描述:
    \begin{itemize}
        \item $\text{Pre}(a)$:前提条件,动作执行前必须满足的条件
        \item $\text{Add}(a)$:添加列表,动作执行后新增的事实
        \item $\text{Del}(a)$:删除列表,动作执行后删除的事实
    \end{itemize}
\end{definition}

\begin{example}[积木世界的动作定义]
    \textbf{拾取动作} $\text{PickUp}(x)$:从桌面上拾取积木$x$
    \begin{align}
        \text{Pre} &: \{\text{OnTable}(x), \text{Clear}(x), \text{ArmEmpty}\} \\
        \text{Add} &: \{\text{Holding}(x)\} \\
        \text{Del} &: \{\text{OnTable}(x), \text{ArmEmpty}\}
    \end{align}

    \textbf{放下动作} $\text{PutDown}(x)$:将手中的积木$x$放到桌面上
    \begin{align}
        \text{Pre} &: \{\text{Holding}(x)\} \\
        \text{Add} &: \{\text{OnTable}(x), \text{Clear}(x), \text{ArmEmpty}\} \\
        \text{Del} &: \{\text{Holding}(x)\}
    \end{align}

    \textbf{堆叠动作} $\text{Stack}(x, y)$:将手中的积木$x$放到积木$y$上
    \begin{align}
        \text{Pre} &: \{\text{Holding}(x), \text{Clear}(y)\} \\
        \text{Add} &: \{\text{On}(x, y), \text{Clear}(x), \text{ArmEmpty}\} \\
        \text{Del} &: \{\text{Holding}(x), \text{Clear}(y)\}
    \end{align}

    \textbf{拆卸动作} $\text{Unstack}(x, y)$:从积木$y$上取下积木$x$
    \begin{align}
        \text{Pre} &: \{\text{On}(x, y), \text{Clear}(x), \text{ArmEmpty}\} \\
        \text{Add} &: \{\text{Holding}(x), \text{Clear}(y)\} \\
        \text{Del} &: \{\text{On}(x, y), \text{ArmEmpty}\}
    \end{align}
\end{example}

\begin{definition}[状态转移函数]
    给定状态$s$和动作$a$,如果$\text{Pre}(a) \subseteq s$,则$a$在$s$中是\keyterm{可应用的}(Applicable)。状态转移函数$\gamma$定义为:
    \begin{equation}
        \gamma(s, a) = (s \setminus \text{Del}(a)) \cup \text{Add}(a)
    \end{equation}
\end{definition}

\subsection{目标状态与规划问题}

\begin{definition}[规划问题]
    一个\keyterm{经典规划问题}可以定义为三元组 $\mathcal{P} = \langle \mathcal{D}, s_0, g \rangle$,其中:
    \begin{itemize}
        \item $\mathcal{D}$是规划领域,包括状态变量和动作定义
        \item $s_0$是初始状态
        \item $g$是目标条件(一组必须满足的命题)
    \end{itemize}
\end{definition}

\begin{definition}[解(规划)]
    规划问题$\mathcal{P}$的解是一个动作序列$\pi = \langle a_1, a_2, \ldots, a_n \rangle$,使得:
    \begin{enumerate}
        \item $a_1$在$s_0$中可应用
        \item 对于$i = 1, \ldots, n-1$,$a_{i+1}$在$\gamma(\cdots\gamma(\gamma(s_0, a_1), a_2)\cdots, a_i)$中可应用
        \item $g \subseteq \gamma(\cdots\gamma(\gamma(s_0, a_1), a_2)\cdots, a_n)$
    \end{enumerate}
\end{definition}

\section{盲目搜索策略}

\keyterm{盲目搜索}(Uninformed Search)也称为无信息搜索,是指在搜索过程中不使用问题特定知识的搜索方法。

\subsection{广度优先搜索}

\keyterm{广度优先搜索}(Breadth-First Search,BFS)按层次顺序扩展节点,先扩展所有深度为$d$的节点,再扩展深度为$d+1$的节点。

\begin{algorithm}[H]
    \caption{广度优先搜索}
    \label{alg:bfs}
    \KwIn{规划问题 $\mathcal{P} = \langle \mathcal{D}, s_0, g \rangle$}
    \KwOut{解(动作序列)或失败}

    $\text{frontier} \gets$ 队列,初始包含 $(s_0, \langle\rangle)$\;
    $\text{explored} \gets \emptyset$\;

    \While{$\text{frontier}$ 非空}{
        $(s, \pi) \gets \text{frontier.dequeue}()$\;
        \If{$g \subseteq s$}{
            \KwRet{$\pi$}
        }
        $\text{explored} \gets \text{explored} \cup \{s\}$\;
        \ForEach{动作 $a$ 在 $s$ 中可应用}{
            $s' \gets \gamma(s, a)$\;
            \If{$s' \notin \text{explored}$ 且 $s' \notin \text{frontier}$}{
                $\text{frontier.enqueue}((s', \pi \cdot \langle a \rangle))$\;
            }
        }
    }
    \KwRet{失败}
\end{algorithm}

\textbf{性质分析}:
\begin{itemize}
    \item \textbf{完备性}:如果解存在,BFS一定能找到
    \item \textbf{最优性}:BFS找到的解是步数最少的(假设所有动作代价相同)
    \item \textbf{时间复杂度}:$O(b^d)$,其中$b$是分支因子,$d$是解的深度
    \item \textbf{空间复杂度}:$O(b^d)$
\end{itemize}

\subsection{深度优先搜索}

\keyterm{深度优先搜索}(Depth-First Search,DFS)总是扩展搜索树中最深的节点。

\begin{algorithm}[H]
    \caption{深度优先搜索}
    \label{alg:dfs}
    \KwIn{规划问题 $\mathcal{P} = \langle \mathcal{D}, s_0, g \rangle$}
    \KwOut{解(动作序列)或失败}

    $\text{frontier} \gets$ 栈,初始包含 $(s_0, \langle\rangle)$\;
    $\text{explored} \gets \emptyset$\;

    \While{$\text{frontier}$ 非空}{
        $(s, \pi) \gets \text{frontier.pop}()$\;
        \If{$g \subseteq s$}{
            \KwRet{$\pi$}
        }
        $\text{explored} \gets \text{explored} \cup \{s\}$\;
        \ForEach{动作 $a$ 在 $s$ 中可应用}{
            $s' \gets \gamma(s, a)$\;
            \If{$s' \notin \text{explored}$}{
                $\text{frontier.push}((s', \pi \cdot \langle a \rangle))$\;
            }
        }
    }
    \KwRet{失败}
\end{algorithm}

\textbf{性质分析}:
\begin{itemize}
    \item \textbf{完备性}:在有限状态空间中完备(需要避免重复状态)
    \item \textbf{最优性}:不保证最优
    \item \textbf{时间复杂度}:$O(b^m)$,其中$m$是状态空间的最大深度
    \item \textbf{空间复杂度}:$O(bm)$
\end{itemize}

\subsection{迭代加深搜索}

\keyterm{迭代加深搜索}(Iterative Deepening Search,IDS)结合了BFS的完备性和DFS的空间效率。

\begin{algorithm}[H]
    \caption{迭代加深搜索}
    \label{alg:ids}
    \KwIn{规划问题 $\mathcal{P}$}
    \KwOut{解或失败}

    \For{$\text{depth} \gets 0$ \KwTo $\infty$}{
        $\text{result} \gets \text{DepthLimitedSearch}(\mathcal{P}, \text{depth})$\;
        \If{$\text{result} \neq \text{cutoff}$}{
            \KwRet{$\text{result}$}
        }
    }
\end{algorithm}

\begin{example}[八数码问题求解]
\label{ex:8puzzle}
    八数码问题(8-Puzzle)是一个经典的搜索问题。在$3 \times 3$的棋盘上有8个编号为1--8的滑块和一个空格,目标是通过移动滑块将棋盘从初始状态变换到目标状态。

    初始状态:
    \begin{center}
    \begin{tabular}{|c|c|c|}
        \hline
        2 & 8 & 3 \\
        \hline
        1 & 6 & 4 \\
        \hline
        7 &   & 5 \\
        \hline
    \end{tabular}
    \end{center}

    目标状态:
    \begin{center}
    \begin{tabular}{|c|c|c|}
        \hline
        1 & 2 & 3 \\
        \hline
        8 &   & 4 \\
        \hline
        7 & 6 & 5 \\
        \hline
    \end{tabular}
    \end{center}

    \textbf{状态表示}:用9元组$(p_1, p_2, \ldots, p_9)$表示,其中$p_i$是位置$i$上的滑块编号(0表示空格)。

    \textbf{动作}:上、下、左、右(移动空格)。

    \textbf{搜索结果比较}:
    \begin{center}
    \begin{tabular}{lrr}
        \toprule
        算法 & 扩展节点数 & 解的长度 \\
        \midrule
        BFS & 约46,000 & 23 \\
        DFS & 可能很大 & 不确定 \\
        IDS & 约47,000 & 23 \\
        \bottomrule
    \end{tabular}
    \end{center}
\end{example}

\section{启发式搜索}

\keyterm{启发式搜索}(Heuristic Search)利用问题特定的知识来引导搜索方向,从而提高搜索效率。

\subsection{启发函数设计}

\begin{definition}[启发函数]
    启发函数$h: S \to \mathbb{R}_{\geq 0}$估计从状态$s$到达目标状态的代价。如果$h(s) = 0$当且仅当$s$满足目标条件,则$h$是\keyterm{目标感知的}(Goal-Aware)。
\end{definition}

\begin{definition}[可采纳启发式]
    如果对于所有状态$s$,$h(s) \leq h^*(s)$,其中$h^*(s)$是从$s$到目标的真实最优代价,则$h$是\keyterm{可采纳的}(Admissible)。
\end{definition}

\begin{definition}[一致启发式]
    如果对于所有状态$s$和动作$a$,$h(s) \leq c(s, a) + h(\gamma(s, a))$,其中$c(s, a)$是执行$a$的代价,则$h$是\keyterm{一致的}(Consistent)。
\end{definition}

\subsection{A*算法}

\keyterm{A*算法}是最著名的启发式搜索算法,它使用评估函数$f(s) = g(s) + h(s)$来选择下一个扩展的节点,其中$g(s)$是从初始状态到$s$的实际代价。

\begin{algorithm}[H]
    \caption{A*算法}
    \label{alg:astar}
    \KwIn{规划问题 $\mathcal{P}$,启发函数 $h$}
    \KwOut{最优解或失败}

    $\text{open} \gets$ 优先队列,按$f$值排序,初始包含 $(s_0, \langle\rangle, 0)$\;
    $\text{closed} \gets \emptyset$\;

    \While{$\text{open}$ 非空}{
        $(s, \pi, g) \gets \text{open.extractMin}()$\;
        \If{$g \subseteq s$}{
            \KwRet{$\pi$}
        }
        \If{$s \in \text{closed}$}{
            \textbf{continue}\;
        }
        $\text{closed} \gets \text{closed} \cup \{s\}$\;
        \ForEach{动作 $a$ 在 $s$ 中可应用}{
            $s' \gets \gamma(s, a)$\;
            $g' \gets g + c(s, a)$\;
            \If{$s' \notin \text{closed}$}{
                $\text{open.insert}((s', \pi \cdot \langle a \rangle, g'))$\;
            }
        }
    }
    \KwRet{失败}
\end{algorithm}

\begin{theorem}
    如果启发函数$h$是可采纳的,则A*算法返回的解是最优的。
\end{theorem}

\begin{theorem}
    如果启发函数$h$是一致的,则A*算法在扩展每个节点时,已经找到了到达该节点的最优路径。
\end{theorem}

\subsection{IDA*算法}

\keyterm{IDA*}(Iterative Deepening A*)是A*的迭代加深版本,用深度限制代替优先队列,节省了空间。

\begin{example}[路径规划中的A*应用]
\label{ex:pathfinding}
    考虑在一个$10 \times 10$的网格地图中,从起点$(1,1)$到达终点$(10,10)$,地图中有障碍物。

    \textbf{状态}:机器人当前位置$(x, y)$

    \textbf{动作}:上、下、左、右移动(代价为1)或对角线移动(代价为$\sqrt{2}$)

    \textbf{启发函数}:常用的启发函数包括:
    \begin{itemize}
        \item \textbf{曼哈顿距离}:$h_1(s) = |x - x_g| + |y - y_g|$(仅考虑四方向移动时可采纳)
        \item \textbf{欧几里得距离}:$h_2(s) = \sqrt{(x - x_g)^2 + (y - y_g)^2}$(总是可采纳)
        \item \textbf{切比雪夫距离}:$h_3(s) = \max(|x - x_g|, |y - y_g|)$(考虑对角移动时可采纳)
    \end{itemize}

    图\ref{fig:astar-example}展示了使用不同启发函数时A*算法的搜索过程。
\end{example}

\begin{figure}[htbp]
    \centering
    \begin{tikzpicture}[scale=0.5]
        % 网格
        \draw[step=1,gray,very thin] (0,0) grid (10,10);

        % 障碍物
        \fill[black!30] (3,2) rectangle (4,7);
        \fill[black!30] (6,4) rectangle (7,9);

        % 起点和终点
        \fill[green!60] (0,0) rectangle (1,1);
        \fill[red!60] (9,9) rectangle (10,10);

        % 路径
        \draw[blue,very thick,->] (0.5,0.5) -- (2.5,0.5) -- (2.5,1.5) -- (2.5,7.5) -- (5.5,7.5) -- (5.5,9.5) -- (9.5,9.5);

        % 标签
        \node at (0.5,0.5) {S};
        \node at (9.5,9.5) {G};
    \end{tikzpicture}
    \caption{网格地图中的A*路径规划示例}
    \label{fig:astar-example}
\end{figure}

\section{搜索算法的性能分析}

\subsection{完备性与最优性}

\begin{table}[htbp]
    \centering
    \caption{搜索算法性能比较}
    \label{tab:search-comparison}
    \begin{tabular}{lcccc}
        \toprule
        算法 & 完备性 & 最优性 & 时间复杂度 & 空间复杂度 \\
        \midrule
        BFS & 是 & 是$^*$ & $O(b^d)$ & $O(b^d)$ \\
        DFS & 否$^{**}$ & 否 & $O(b^m)$ & $O(bm)$ \\
        IDS & 是 & 是$^*$ & $O(b^d)$ & $O(bd)$ \\
        A* & 是 & 是$^{***}$ & $O(b^d)$ & $O(b^d)$ \\
        IDA* & 是 & 是$^{***}$ & $O(b^d)$ & $O(bd)$ \\
        \bottomrule
    \end{tabular}

    \footnotesize{$^*$假设所有动作代价相同;$^{**}$在无限空间中不完备;$^{***}$要求启发函数可采纳}
\end{table}

\subsection{时间复杂度与空间复杂度}

规划问题的计算复杂性是一个重要的理论问题。

\begin{theorem}
    经典规划问题(判定版本)是PSPACE完全的。
\end{theorem}

这意味着规划问题在最坏情况下非常困难。然而,实际问题往往具有特殊结构,可以用启发式方法高效求解。

\section*{本章小结}

本章介绍了状态空间搜索的基本概念和算法。状态空间由状态、动作和转移函数组成。盲目搜索算法(BFS、DFS、IDS)不使用问题特定知识,而启发式搜索(A*、IDA*)利用启发函数引导搜索。选择合适的搜索算法和启发函数对规划系统的效率至关重要。

\section*{习题}

\begin{enumerate}
    \item 对于示例\ref{ex:8puzzle}中的八数码问题,设计两种不同的启发函数,并分析其可采纳性。

    \item 证明:如果启发函数$h$是一致的,则$h$必定是可采纳的。

    \item 实现BFS和A*算法,求解15数码问题,比较两者的性能。

    \item 考虑一个机器人在$n \times n$网格中导航的问题,分析以下情况下A*算法的时间复杂度:
    \begin{enumerate}
        \item 使用$h(s) = 0$(退化为Dijkstra算法)
        \item 使用完美启发函数$h(s) = h^*(s)$
    \end{enumerate}

    \item 设计一个规划问题,使得DFS比BFS效率更高,并解释原因。
\end{enumerate}

% 第3章 经典规划理论
\chapter{经典规划理论}
\label{chap:classical}

本章介绍经典规划的核心理论,包括STRIPS表示语言、PDDL规划领域定义语言,以及基本的规划算法。

\section{STRIPS表示语言}

\keyterm{STRIPS}(Stanford Research Institute Problem Solver)是1969年由Fikes和Nilsson提出的规划系统,其表示方法至今仍是规划领域的基础。

\subsection{前提条件与效果}

STRIPS使用\keyterm{前提条件}(Preconditions)和\keyterm{效果}(Effects)来描述动作:

\begin{definition}[STRIPS动作]
    一个STRIPS动作$a$由三部分组成:
    \begin{itemize}
        \item \textbf{前提条件} $\text{Pre}(a)$:动作执行前必须满足的条件集合
        \item \textbf{添加效果} $\text{Add}(a)$:动作执行后变为真的事实集合
        \item \textbf{删除效果} $\text{Del}(a)$:动作执行后变为假的事实集合
    \end{itemize}
\end{definition}

\subsection{动作模式定义}

在实际应用中,我们通常定义\keyterm{动作模式}(Action Schema),即带参数的动作模板。

\begin{example}[积木世界问题]
\label{ex:blocksworld-strips}
    积木世界是规划领域的经典测试问题。假设有若干积木和一个机械手,目标是通过移动积木实现特定的堆叠配置。

    \textbf{谓词定义}:
    \begin{itemize}
        \item $\text{on}(x, y)$:积木$x$在积木$y$上面
        \item $\text{ontable}(x)$:积木$x$在桌面上
        \item $\text{clear}(x)$:积木$x$上面没有其他积木
        \item $\text{holding}(x)$:机械手正在抓取积木$x$
        \item $\text{arm-empty}$:机械手是空的
    \end{itemize}

    \textbf{动作模式}:

    \begin{center}
    \fbox{\parbox{0.8\textwidth}{
        \textbf{动作}:$\text{pick-up}(x)$ \\
        \textbf{前提}:$\text{ontable}(x) \land \text{clear}(x) \land \text{arm-empty}$ \\
        \textbf{效果}:$\text{holding}(x) \land \neg\text{ontable}(x) \land \neg\text{arm-empty}$
    }}
    \end{center}

    \begin{center}
    \fbox{\parbox{0.8\textwidth}{
        \textbf{动作}:$\text{put-down}(x)$ \\
        \textbf{前提}:$\text{holding}(x)$ \\
        \textbf{效果}:$\text{ontable}(x) \land \text{clear}(x) \land \text{arm-empty} \land \neg\text{holding}(x)$
    }}
    \end{center}

    \begin{center}
    \fbox{\parbox{0.8\textwidth}{
        \textbf{动作}:$\text{stack}(x, y)$ \\
        \textbf{前提}:$\text{holding}(x) \land \text{clear}(y)$ \\
        \textbf{效果}:$\text{on}(x,y) \land \text{clear}(x) \land \text{arm-empty} \land \neg\text{holding}(x) \land \neg\text{clear}(y)$
    }}
    \end{center}

    \begin{center}
    \fbox{\parbox{0.8\textwidth}{
        \textbf{动作}:$\text{unstack}(x, y)$ \\
        \textbf{前提}:$\text{on}(x,y) \land \text{clear}(x) \land \text{arm-empty}$ \\
        \textbf{效果}:$\text{holding}(x) \land \text{clear}(y) \land \neg\text{on}(x,y) \land \neg\text{arm-empty}$
    }}
    \end{center}
\end{example}

\section{PDDL规划领域定义语言}

\keyterm{PDDL}(Planning Domain Definition Language)是1998年为国际规划竞赛制定的标准规划语言。

\subsection{PDDL语法结构}

PDDL将规划问题分为两个文件:
\begin{itemize}
    \item \textbf{领域文件}(Domain File):定义谓词和动作模式
    \item \textbf{问题文件}(Problem File):定义具体实例的初始状态和目标
\end{itemize}

\subsection{领域文件与问题文件}

\begin{example}[物流运输领域建模]
\label{ex:logistics-pddl}
    考虑一个简化的物流问题:有若干城市、卡车和包裹,目标是将包裹运送到指定目的地。

    \textbf{领域文件}(logistics-domain.pddl):

\begin{lstlisting}[language=PDDL]
(define (domain logistics)
  (:requirements :strips :typing)

  (:types
    city location thing - object
    package vehicle - thing
    truck airplane - vehicle
  )

  (:predicates
    (in-city ?loc - location ?city - city)
    (at ?obj - thing ?loc - location)
    (in ?pkg - package ?veh - vehicle)
  )

  (:action load-truck
    :parameters (?pkg - package ?truck - truck ?loc - location)
    :precondition (and
      (at ?truck ?loc)
      (at ?pkg ?loc)
    )
    :effect (and
      (not (at ?pkg ?loc))
      (in ?pkg ?truck)
    )
  )

  (:action unload-truck
    :parameters (?pkg - package ?truck - truck ?loc - location)
    :precondition (and
      (at ?truck ?loc)
      (in ?pkg ?truck)
    )
    :effect (and
      (not (in ?pkg ?truck))
      (at ?pkg ?loc)
    )
  )

  (:action drive-truck
    :parameters (?truck - truck ?from ?to - location ?city - city)
    :precondition (and
      (at ?truck ?from)
      (in-city ?from ?city)
      (in-city ?to ?city)
    )
    :effect (and
      (not (at ?truck ?from))
      (at ?truck ?to)
    )
  )
)
\end{lstlisting}

    \textbf{问题文件}(logistics-problem.pddl):

\begin{lstlisting}[language=PDDL]
(define (problem logistics-1)
  (:domain logistics)

  (:objects
    city1 city2 - city
    loc1-1 loc1-2 loc2-1 loc2-2 - location
    truck1 truck2 - truck
    pkg1 pkg2 - package
  )

  (:init
    (in-city loc1-1 city1) (in-city loc1-2 city1)
    (in-city loc2-1 city2) (in-city loc2-2 city2)
    (at truck1 loc1-1)
    (at truck2 loc2-1)
    (at pkg1 loc1-2)
    (at pkg2 loc2-1)
  )

  (:goal (and
    (at pkg1 loc2-2)
    (at pkg2 loc1-1)
  ))
)
\end{lstlisting}
\end{example}

\subsection{PDDL扩展特性}

PDDL经历了多个版本的演进,引入了许多扩展特性:

\begin{table}[htbp]
    \centering
    \caption{PDDL版本演进}
    \label{tab:pddl-versions}
    \begin{tabular}{lll}
        \toprule
        版本 & 年份 & 主要特性 \\
        \midrule
        PDDL 1.2 & 1998 & STRIPS、类型、量词 \\
        PDDL 2.1 & 2002 & 数值fluents、持续性动作 \\
        PDDL 2.2 & 2004 & 派生谓词、时态初始文字 \\
        PDDL 3.0 & 2006 & 轨迹约束、偏好 \\
        PDDL 3.1 & 2008 & 对象fluents \\
        \bottomrule
    \end{tabular}
\end{table}

\section{前向状态空间规划}

\keyterm{前向规划}(Forward Planning)也称为\keyterm{前进规划}(Progression),从初始状态出发,逐步应用动作直到达到目标。

\subsection{基本算法}

前向规划的基本思想是将规划问题转化为图搜索问题,其中:
\begin{itemize}
    \item 节点是状态
    \item 边是动作
    \item 初始节点是$s_0$
    \item 目标节点是满足$g$的状态
\end{itemize}

\subsection{可达性分析}

\begin{definition}[可达状态]
    从初始状态$s_0$出发,通过执行一系列动作可以到达的状态称为\keyterm{可达状态}。所有可达状态的集合称为\keyterm{可达状态空间}。
\end{definition}

\section{后向状态空间规划}

\keyterm{后向规划}(Backward Planning)也称为\keyterm{回归规划}(Regression),从目标条件出发,逆向推导需要满足的前提条件。

\subsection{回归搜索}

\begin{definition}[回归]
    给定目标条件$g$和动作$a$,如果$a$可能实现$g$中的某些子目标,则$g$关于$a$的\keyterm{回归}定义为:
    \begin{equation}
        \text{regress}(g, a) = (g \setminus \text{Add}(a)) \cup \text{Pre}(a)
    \end{equation}
    前提是$\text{Add}(a) \cap g \neq \emptyset$且$\text{Del}(a) \cap g = \emptyset$。
\end{definition}

\subsection{相关性分析}

在后向搜索中,我们只考虑与目标\keyterm{相关}的动作,即那些能够实现至少一个目标子条件的动作。

\section{规划图与GraphPlan算法}

\keyterm{GraphPlan}是1995年由Blum和Furst提出的规划算法,它通过构建规划图来加速规划过程。

\subsection{规划图构造}

\begin{definition}[规划图]
    规划图是一个有向分层图,包含交替的\keyterm{命题层}和\keyterm{动作层}:
    \begin{itemize}
        \item 命题层包含在该时刻可能为真的命题
        \item 动作层包含在该时刻可能执行的动作
        \item 层之间通过前提条件和效果连接
    \end{itemize}
\end{definition}

\subsection{互斥关系}

\begin{definition}[互斥]
    两个动作(或两个命题)是\keyterm{互斥的}(Mutex),如果它们不能在同一规划步骤中同时出现。动作互斥的条件包括:
    \begin{enumerate}
        \item \textbf{不一致效果}:一个动作的效果否定另一个的效果
        \item \textbf{干扰}:一个动作的效果否定另一个的前提
        \item \textbf{竞争需求}:两个动作的前提条件互斥
    \end{enumerate}
\end{definition}

\subsection{解的提取}

在规划图构造完成后,使用后向搜索提取解。

\begin{example}[火箭运输问题]
\label{ex:rocket}
    考虑一个简化的火箭运输问题:
    \begin{itemize}
        \item 有两个地点:伦敦(London)和巴黎(Paris)
        \item 有一枚火箭R1,初始在伦敦
        \item 有两件货物A和B,初始都在伦敦
        \item 目标:将A和B都运送到巴黎
    \end{itemize}

    \textbf{动作}:
    \begin{itemize}
        \item $\text{load}(c, r, l)$:在地点$l$将货物$c$装载到火箭$r$
        \item $\text{unload}(c, r, l)$:在地点$l$将货物$c$从火箭$r$卸载
        \item $\text{move}(r, from, to)$:火箭$r$从$from$飞到$to$
    \end{itemize}

    \textbf{规划图}(部分):

    \begin{center}
    \begin{tikzpicture}[scale=0.9,
        prop/.style={rectangle, draw, minimum width=2cm, minimum height=0.6cm, font=\small},
        action/.style={ellipse, draw, minimum width=1.5cm, font=\small},
        mutex/.style={red, dashed, thick}
    ]
        % 第0层命题
        \node[prop] (at-a-l-0) at (0,3) {at(A,L)};
        \node[prop] (at-b-l-0) at (0,2) {at(B,L)};
        \node[prop] (at-r-l-0) at (0,1) {at(R,L)};
        \node[prop] (fuel-0) at (0,0) {fuel(R)};

        % 第1层动作
        \node[action] (load-a) at (3,2.5) {load(A)};
        \node[action] (load-b) at (3,1.5) {load(B)};
        \node[action] (noop1) at (3,0.5) {no-op};

        % 连接
        \draw[->] (at-a-l-0) -- (load-a);
        \draw[->] (at-r-l-0) -- (load-a);
        \draw[->] (at-b-l-0) -- (load-b);
        \draw[->] (at-r-l-0) -- (load-b);

        % 标签
        \node at (0,-1) {$P_0$};
        \node at (3,-1) {$A_1$};
    \end{tikzpicture}
    \end{center}

    \textbf{解}:
    \begin{enumerate}
        \item load(A, R1, London)
        \item load(B, R1, London)
        \item move(R1, London, Paris)
        \item unload(A, R1, Paris)
        \item unload(B, R1, Paris)
    \end{enumerate}
\end{example}

\section*{本章小结}

本章介绍了经典规划的核心理论。STRIPS是最早也是最基础的规划表示方法,通过前提条件和效果描述动作。PDDL是现代规划的标准语言,支持类型、数值、时态等扩展特性。前向规划和后向规划是两种基本的规划策略,GraphPlan通过构建规划图和分析互斥关系来加速规划过程。

\section*{习题}

\begin{enumerate}
    \item 用STRIPS格式定义一个"河内塔"(Tower of Hanoi)问题的动作。

    \item 编写PDDL领域文件,建模一个简单的"机器人取物"问题:机器人需要在房间之间移动并拾取物品。

    \item 对于示例\ref{ex:rocket}中的火箭运输问题,手工构造完整的规划图(直到所有目标第一次同时出现且不互斥)。

    \item 分析前向规划和后向规划各自的优缺点,讨论在什么情况下应该选择哪种方法。

    \item 实现一个简单的GraphPlan算法,求解积木世界问题。
\end{enumerate}

% 第4章 启发式规划方法
\chapter{启发式规划方法}
\label{chap:heuristic}

启发式方法是现代规划器成功的关键。本章介绍规划中常用的启发式函数设计方法。

\section{松弛启发式}

\keyterm{松弛}(Relaxation)是设计启发式的一种通用方法,通过简化原问题来获得代价的下界估计。

\subsection{删除松弛}

\begin{definition}[删除松弛]
    \keyterm{删除松弛}(Delete Relaxation)是指忽略所有动作的删除效果。在删除松弛下,一旦某个命题变为真,它将永远保持为真。
\end{definition}

\subsection{$h^{add}$与$h^{max}$启发式}

\begin{definition}[$h^{max}$启发式]
    $h^{max}$启发式估计达到目标集合中"最难"达到的子目标的代价:
    \begin{equation}
        h^{max}(s) = \max_{p \in g} h^{max}_p(s)
    \end{equation}
    其中$h^{max}_p(s)$是从状态$s$达到命题$p$的估计代价。
\end{definition}

\begin{definition}[$h^{add}$启发式]
    $h^{add}$启发式假设各子目标独立,将达到各子目标的代价相加:
    \begin{equation}
        h^{add}(s) = \sum_{p \in g} h^{add}_p(s)
    \end{equation}
\end{definition}

$h^{max}$是可采纳的但信息量较少;$h^{add}$信息量丰富但通常高估真实代价。

\subsection{$h^{FF}$启发式}

\keyterm{FF启发式}由Hoffmann和Nebel提出,是最成功的规划启发式之一。它基于松弛规划图,计算从当前状态到目标的松弛计划长度。

\begin{algorithm}[H]
    \caption{FF启发式计算}
    \label{alg:hff}
    \KwIn{当前状态 $s$,目标 $g$}
    \KwOut{启发式值 $h^{FF}(s)$}

    构建松弛规划图直到$g$中所有命题出现\;
    \If{$g$中存在命题从未出现}{
        \KwRet{$\infty$}
    }
    从最后一层开始,使用贪心方法提取松弛计划\;
    \KwRet{松弛计划中的动作数量}
\end{algorithm}

\section{抽象启发式}

\keyterm{抽象}(Abstraction)通过将多个状态映射到同一抽象状态来简化问题。

\subsection{模式数据库}

\begin{definition}[模式数据库]
    \keyterm{模式数据库}(Pattern Database,PDB)预先计算抽象空间中所有状态到目标的精确代价,然后在搜索时作为启发式查表使用。
\end{definition}

\subsection{合并与收缩}

\keyterm{合并与收缩}(Merge-and-Shrink)是一种系统化构建抽象的方法,通过合并变量和收缩状态空间来控制抽象的大小。

\section{地标启发式}

\keyterm{地标}(Landmark)是在任何解中都必须在某个时刻为真的命题或必须执行的动作。

\subsection{地标识别}

识别地标的方法包括:
\begin{itemize}
    \item 基于松弛规划图的方法
    \item 基于回归的方法
    \item 基于SAT/CSP的方法
\end{itemize}

\subsection{地标计数启发式}

\begin{definition}[地标计数启发式]
    地标计数启发式$h^{LM}$估计从当前状态到目标还需要达成的地标数量:
    \begin{equation}
        h^{LM}(s) = |\{l \in L : l \text{ 在 } s \text{ 中未被满足且尚未达成}\}|
    \end{equation}
\end{definition}

\section{现代规划器架构}

\subsection{Fast Downward系统}

\keyterm{Fast Downward}是最成功的现代规划系统之一,由Helmert于2006年提出。其主要特点包括:
\begin{itemize}
    \item 将PDDL转换为多值状态变量表示(SAS+)
    \item 支持多种启发式函数
    \item 支持多种搜索算法
    \item 模块化设计,易于扩展
\end{itemize}

\subsection{LAMA规划器}

\keyterm{LAMA}(Landmarks, Actions, and Multi-heuristics Anytime)规划器结合了地标启发式和FF启发式,在国际规划竞赛中表现优异。

\begin{example}[国际规划竞赛问题求解]
\label{ex:ipc}
    国际规划竞赛(IPC)是规划领域的重要评测平台。以下是LAMA在IPC 2011物流领域问题上的表现:

    \begin{center}
    \begin{tabular}{lrrr}
        \toprule
        问题规模 & Fast Downward & LAMA & 最优规划器 \\
        \midrule
        小(<100状态) & 0.1s & 0.1s & 0.5s \\
        中(<10000状态) & 2.3s & 1.8s & 超时 \\
        大(>100000状态) & 45s & 32s & 超时 \\
        \bottomrule
    \end{tabular}
    \end{center}

    这说明启发式搜索规划器在大规模问题上具有显著优势。
\end{example}

\section*{本章小结}

本章介绍了规划中的启发式方法。松弛启发式通过简化问题获得代价估计,其中$h^{FF}$是最成功的启发式之一。抽象启发式通过状态聚合来简化问题。地标启发式利用问题的结构信息。现代规划器如Fast Downward和LAMA结合了多种技术,在实际问题中表现出色。

\section*{习题}

\begin{enumerate}
    \item 对于积木世界问题,计算$h^{add}$和$h^{max}$启发式的值,并比较它们与真实代价的关系。

    \item 证明$h^{max}$是可采纳的启发式。

    \item 为八数码问题设计一个模式数据库,讨论如何选择模式以获得信息丰富的启发式。

    \item 在一个物流问题中,识别可能的地标,并解释为什么它们必须在任何解中出现。

    \item 下载并安装Fast Downward,使用不同的启发式配置求解IPC基准问题,比较性能差异。
\end{enumerate}

% 第5章 时态规划与调度
\chapter{时态规划与调度}
\label{chap:temporal}

本章介绍考虑时间约束的规划问题,包括时态规划和调度问题。

\section{时态逻辑基础}

\subsection{时态算子}

时态逻辑引入了描述时间的算子:
\begin{itemize}
    \item $\square \phi$(总是):$\phi$在所有未来时刻都为真
    \item $\Diamond \phi$(最终):$\phi$在某个未来时刻为真
    \item $\bigcirc \phi$(下一刻):$\phi$在下一时刻为真
    \item $\phi \mathcal{U} \psi$(直到):$\phi$保持为真直到$\psi$变为真
\end{itemize}

\subsection{时态约束}

\begin{definition}[时态约束]
    时态约束规定动作之间的时间关系,常见形式包括:
    \begin{itemize}
        \item 先后约束:动作$a$必须在动作$b$之前完成
        \item 同时约束:动作$a$和$b$必须同时执行
        \item 时间窗约束:动作必须在指定时间范围内执行
        \item 持续时间约束:动作执行需要一定的时间
    \end{itemize}
\end{definition}

\section{PDDL 2.1时态扩展}

\subsection{持续性动作}

\begin{definition}[持续性动作]
    \keyterm{持续性动作}(Durative Action)有明确的开始时刻、结束时刻和持续时间。其定义包括:
    \begin{itemize}
        \item \texttt{:duration}:动作的持续时间
        \item \texttt{:condition}:开始条件、结束条件和持续条件
        \item \texttt{:effect}:开始效果和结束效果
    \end{itemize}
\end{definition}

\begin{lstlisting}[language=PDDL]
(:durative-action drive
  :parameters (?truck - truck ?from ?to - location)
  :duration (= ?duration (travel-time ?from ?to))
  :condition (and
    (at start (at ?truck ?from))
    (over all (road ?from ?to))
  )
  :effect (and
    (at start (not (at ?truck ?from)))
    (at end (at ?truck ?to))
  )
)
\end{lstlisting}

\subsection{并发执行}

时态规划允许多个动作并发执行,但需要满足以下条件:
\begin{enumerate}
    \item 不存在资源冲突
    \item 不违反因果依赖
    \item 满足时间约束
\end{enumerate}

\section{时态规划算法}

\subsection{SAPA规划器}

SAPA是一个前向搜索时态规划器,使用时态启发式引导搜索。

\subsection{TFD规划器}

TFD(Temporal Fast Downward)扩展了Fast Downward以支持时态规划。

\section{调度问题}

\subsection{作业车间调度}

\begin{definition}[作业车间调度]
    \keyterm{作业车间调度}(Job Shop Scheduling)问题定义如下:
    \begin{itemize}
        \item 有$n$个作业(jobs)和$m$台机器(machines)
        \item 每个作业包含若干工序,工序之间有先后约束
        \item 每个工序需要在特定机器上处理特定时间
        \item 目标是最小化总完工时间(makespan)
    \end{itemize}
\end{definition}

\subsection{资源约束项目调度}

\begin{definition}[RCPSP]
    \keyterm{资源约束项目调度问题}(Resource-Constrained Project Scheduling Problem,RCPSP)是一类考虑有限资源约束的调度问题。
\end{definition}

\begin{example}[生产线调度优化]
\label{ex:production}
    某制造企业有3条生产线和10个生产订单。每个订单需要经过多道工序,部分工序可以在不同生产线上执行。约束条件包括:
    \begin{itemize}
        \item 每条生产线同一时刻只能处理一个工序
        \item 同一订单的工序之间有先后关系
        \item 部分订单有交货期限制
    \end{itemize}

    目标是最小化总延迟时间或最大化产能利用率。
\end{example}

\begin{example}[航班调度问题]
\label{ex:flight}
    机场需要为当日100架次航班分配停机位和登机口。约束条件包括:
    \begin{itemize}
        \item 每个停机位同一时刻只能停放一架飞机
        \item 登机口与停机位之间需要满足一定的对应关系
        \item 不同机型对停机位有不同要求
        \item 需要预留足够的周转时间
    \end{itemize}

    目标是最小化乘客步行距离或最大化机场运营效率。
\end{example}

\section*{本章小结}

本章介绍了时态规划与调度。时态规划扩展了经典规划以处理时间约束和持续性动作。PDDL 2.1引入了持续性动作和数值fluent。调度问题是规划的重要应用领域,包括作业车间调度和资源约束项目调度。

\section*{习题}

\begin{enumerate}
    \item 用PDDL 2.1编写一个简单的时态规划领域,包含至少两个持续性动作。

    \item 对于示例\ref{ex:production}中的生产调度问题,画出一个可行的甘特图。

    \item 比较规划问题和调度问题的异同点。

    \item 分析时态规划相比经典规划的计算复杂性。

    \item 设计一个启发式函数,用于评估时态规划中的部分计划质量。
\end{enumerate}

% 第6章 层次任务网络规划
\chapter{层次任务网络规划}
\label{chap:htn}

\keyterm{层次任务网络}(Hierarchical Task Network,HTN)规划是一种通过任务分解来求解规划问题的方法。

\section{HTN基本概念}

\subsection{任务与方法}

\begin{definition}[任务]
    在HTN规划中,\keyterm{任务}分为两类:
    \begin{itemize}
        \item \textbf{原子任务}(Primitive Task):可以直接执行的动作
        \item \textbf{复合任务}(Compound Task):需要进一步分解的抽象任务
    \end{itemize}
\end{definition}

\begin{definition}[方法]
    \keyterm{方法}(Method)定义了如何将复合任务分解为子任务序列。一个方法$m$包含:
    \begin{itemize}
        \item 任务头:要分解的复合任务
        \item 前提条件:方法适用的条件
        \item 任务网络:分解后的子任务及其约束
    \end{itemize}
\end{definition}

\subsection{任务分解}

HTN规划的核心思想是递归分解:从初始任务网络开始,不断选择复合任务并应用方法进行分解,直到所有任务都是原子任务。

\section{SHOP规划系统}

\keyterm{SHOP}(Simple Hierarchical Ordered Planner)是一个广泛使用的HTN规划系统。

\subsection{SHOP算法}

\begin{algorithm}[H]
    \caption{SHOP算法}
    \label{alg:shop}
    \KwIn{当前状态 $s$,任务列表 $T$}
    \KwOut{规划 $\pi$ 或失败}

    \If{$T$ 为空}{
        \KwRet{空规划 $\langle\rangle$}
    }
    $t \gets T$ 的第一个任务\;
    \eIf{$t$ 是原子任务}{
        \If{$t$ 的前提条件在 $s$ 中满足}{
            $s' \gets$ 执行 $t$ 后的状态\;
            $\pi \gets \text{SHOP}(s', T \setminus \{t\})$\;
            \If{$\pi \neq$ 失败}{
                \KwRet{$\langle t \rangle \cdot \pi$}
            }
        }
    }{
        \ForEach{方法 $m$ 可以分解 $t$}{
            \If{$m$ 的前提条件在 $s$ 中满足}{
                $T' \gets$ 用 $m$ 的子任务替换 $T$ 中的 $t$\;
                $\pi \gets \text{SHOP}(s, T')$\;
                \If{$\pi \neq$ 失败}{
                    \KwRet{$\pi$}
                }
            }
        }
    }
    \KwRet{失败}
\end{algorithm}

\subsection{SHOP2扩展}

SHOP2是SHOP的扩展版本,支持:
\begin{itemize}
    \item 偏序任务网络
    \item 分支和循环
    \item 数值计算
    \item 外部函数调用
\end{itemize}

\section{HDDL表示语言}

\keyterm{HDDL}(Hierarchical Domain Definition Language)是HTN规划的标准表示语言。

\subsection{全序与偏序HTN}

\begin{itemize}
    \item \textbf{全序HTN}:子任务之间有严格的顺序约束
    \item \textbf{偏序HTN}:子任务之间只有部分顺序约束
\end{itemize}

\subsection{HDDL语法}

\begin{lstlisting}[language=PDDL]
(:method deliver-package
  :parameters (?p - package ?from ?to - location)
  :task (deliver ?p ?to)
  :precondition (at ?p ?from)
  :ordered-subtasks (and
    (pick-up ?p ?from)
    (transport ?p ?from ?to)
    (put-down ?p ?to)
  )
)
\end{lstlisting}

\section{HTN应用案例}

\begin{example}[军事作战任务分解]
\label{ex:military-htn}
    考虑一个空中打击任务的层次分解:

    \textbf{顶层任务}:执行空中打击任务

    \textbf{分解方法}:
    \begin{enumerate}
        \item 任务规划阶段
        \begin{itemize}
            \item 情报收集与分析
            \item 目标确认
            \item 航线规划
        \end{itemize}
        \item 任务准备阶段
        \begin{itemize}
            \item 装备检查
            \item 燃料加注
            \item 弹药装载
        \end{itemize}
        \item 任务执行阶段
        \begin{itemize}
            \item 起飞
            \item 航渡
            \item 目标攻击
            \item 返航
        \end{itemize}
        \item 任务评估阶段
        \begin{itemize}
            \item 战果评估
            \item 任务总结
        \end{itemize}
    \end{enumerate}
\end{example}

\begin{example}[智能家居任务规划]
\label{ex:smarthome-htn}
    智能家居系统收到用户指令:"准备晚餐派对"

    \textbf{任务分解}:
    \begin{enumerate}
        \item 环境准备
        \begin{itemize}
            \item 调节室温至22度
            \item 调暗灯光
            \item 播放背景音乐
        \end{itemize}
        \item 餐饮准备
        \begin{itemize}
            \item 预热烤箱
            \item 准备食材(通知用户)
            \item 设置定时器
        \end{itemize}
        \item 安全检查
        \begin{itemize}
            \item 检查门锁状态
            \item 启动访客模式
        \end{itemize}
    \end{enumerate}
\end{example}

\begin{example}[游戏AI行为规划]
\label{ex:game-htn}
    在策略游戏中,AI需要规划"建造军事基地":

    \textbf{方法1}(资源充足时):
    \begin{enumerate}
        \item 选择建造地点
        \item 派遣工人
        \item 建造兵营
        \item 建造防御塔
        \item 训练士兵
    \end{enumerate}

    \textbf{方法2}(资源不足时):
    \begin{enumerate}
        \item 采集资源
        \item 递归调用"建造军事基地"
    \end{enumerate}
\end{example}

\section*{本章小结}

HTN规划通过任务分解的方式求解规划问题,更符合人类的规划思维方式。SHOP系列是最著名的HTN规划系统,HDDL是HTN规划的标准表示语言。HTN规划在军事、智能家居、游戏AI等领域有广泛应用。

\section*{习题}

\begin{enumerate}
    \item 设计一个HTN领域,对"组织一次旅行"任务进行分解。

    \item 比较HTN规划和经典规划的优缺点。

    \item 用HDDL编写示例\ref{ex:smarthome-htn}中的智能家居任务规划领域。

    \item 分析HTN规划的完备性:在什么条件下HTN规划能够找到解?

    \item 设计一个HTN规划问题,使得全序和偏序分解产生不同的结果。
\end{enumerate}


% ==================== 第二篇:方法篇 ====================
\part{方法篇}

% 第7章 约束满足与规划
\chapter{约束满足与规划}
\label{chap:csp}

本章介绍约束满足问题及其在规划中的应用。

\section{约束满足问题}

\subsection{CSP形式化}

\begin{definition}[约束满足问题]
    \keyterm{约束满足问题}(Constraint Satisfaction Problem,CSP)定义为三元组 $\langle X, D, C \rangle$:
    \begin{itemize}
        \item $X = \{x_1, x_2, \ldots, x_n\}$ 是变量集合
        \item $D = \{D_1, D_2, \ldots, D_n\}$ 是值域集合,$D_i$ 是 $x_i$ 的可能取值
        \item $C = \{c_1, c_2, \ldots, c_m\}$ 是约束集合
    \end{itemize}
\end{definition}

\subsection{约束传播}

\keyterm{约束传播}(Constraint Propagation)通过推理减少变量的值域。常用技术包括:
\begin{itemize}
    \item \textbf{节点一致性}:确保每个变量的值域满足一元约束
    \item \textbf{弧一致性}:确保每对变量满足二元约束
    \item \textbf{路径一致性}:确保任意三个变量的组合一致
\end{itemize}

\subsection{回溯搜索}

\begin{algorithm}[H]
    \caption{CSP回溯搜索}
    \label{alg:csp-backtrack}
    \KwIn{CSP问题 $\langle X, D, C \rangle$,部分赋值 $\sigma$}
    \KwOut{完整解或失败}

    \If{$\sigma$ 是完整赋值}{
        \KwRet{$\sigma$}
    }
    选择一个未赋值变量 $x$\;
    \ForEach{$v \in D_x$ 按某种顺序}{
        \If{$\sigma \cup \{x = v\}$ 满足所有约束}{
            $\text{result} \gets \text{Backtrack}(\sigma \cup \{x = v\})$\;
            \If{$\text{result} \neq$ 失败}{
                \KwRet{$\text{result}$}
            }
        }
    }
    \KwRet{失败}
\end{algorithm}

\section{规划问题的SAT编码}

\subsection{命题逻辑基础}

命题逻辑的基本概念:
\begin{itemize}
    \item \textbf{命题变量}:取值为真或假的变量
    \item \textbf{文字}:命题变量或其否定
    \item \textbf{子句}:文字的析取
    \item \textbf{CNF}:子句的合取
\end{itemize}

\subsection{规划到SAT的转换}

将规划问题编码为SAT的关键思想是引入时间步:
\begin{itemize}
    \item $p_t$:命题$p$在时刻$t$为真
    \item $a_t$:动作$a$在时刻$t$执行
\end{itemize}

\textbf{编码规则}:
\begin{enumerate}
    \item 初始状态:$\bigwedge_{p \in s_0} p_0 \land \bigwedge_{p \notin s_0} \neg p_0$
    \item 目标状态:$\bigwedge_{p \in g} p_T$
    \item 动作前提:$a_t \rightarrow \bigwedge_{p \in \text{Pre}(a)} p_t$
    \item 动作效果:$a_t \rightarrow \bigwedge_{p \in \text{Add}(a)} p_{t+1}$
    \item 帧公理:$(p_t \land \neg p_{t+1}) \rightarrow \bigvee_{a: p \in \text{Del}(a)} a_t$
\end{enumerate}

\subsection{SAT求解器应用}

\begin{example}[布尔可满足性规划]
\label{ex:sat-planning}
    考虑一个简单的积木问题,初始状态A在桌上,B在A上;目标是B在桌上,A在B上。

    \textbf{SAT编码}(假设最多2步):

    命题变量:
    \begin{itemize}
        \item $\text{on}(A,B)_t$, $\text{on}(B,A)_t$:积木位置
        \item $\text{ontable}(A)_t$, $\text{ontable}(B)_t$:在桌上
        \item $\text{unstack}(B,A)_t$, $\text{stack}(A,B)_t$, ...:动作
    \end{itemize}

    初始状态子句:
    \begin{align}
        &\text{ontable}(A)_0 \\
        &\text{on}(B,A)_0 \\
        &\neg\text{on}(A,B)_0 \\
        &\neg\text{ontable}(B)_0
    \end{align}

    目标子句:
    \begin{align}
        &\text{ontable}(B)_2 \\
        &\text{on}(A,B)_2
    \end{align}

    SAT求解器返回满足赋值,从中提取动作序列。
\end{example}

\section{规划问题的SMT编码}

\subsection{SMT理论}

\keyterm{SMT}(Satisfiability Modulo Theories)扩展了SAT,支持更丰富的理论:
\begin{itemize}
    \item 线性算术
    \item 数组理论
    \item 位向量
    \item 非线性算术
\end{itemize}

\subsection{时态规划的SMT方法}

SMT特别适合时态规划,因为可以直接处理连续时间变量和数值约束。

\section*{本章小结}

本章介绍了约束满足与规划的关系。CSP是一种强大的建模语言,规划问题可以编码为SAT或SMT问题。现代SAT/SMT求解器的强大能力使得这种方法在实践中非常有效。

\section*{习题}

\begin{enumerate}
    \item 将八皇后问题建模为CSP,并用回溯搜索求解。

    \item 为示例\ref{ex:sat-planning}中的积木问题写出完整的SAT编码。

    \item 比较SAT规划和启发式搜索规划的优缺点。

    \item 设计一个需要数值约束的规划问题,说明为什么SMT比SAT更适合。

    \item 实现一个简单的SAT规划器,测试其在小规模问题上的性能。
\end{enumerate}

% 第8章 不确定性规划
\chapter{不确定性规划}
\label{chap:uncertainty}

本章讨论在不确定环境下的规划问题。

\section{非确定性规划}

\subsection{条件规划}

\begin{definition}[条件规划]
    \keyterm{条件规划}(Contingent Planning)生成的计划包含条件分支,根据执行过程中观测到的信息选择不同的动作。
\end{definition}

条件计划可以表示为树结构:
\begin{itemize}
    \item 内部节点是观测或条件判断
    \item 叶节点是动作序列
\end{itemize}

\subsection{一致性规划}

\begin{definition}[一致性规划]
    \keyterm{一致性规划}(Conformant Planning)在没有任何观测能力的情况下进行规划。计划必须在所有可能的初始状态下都能达到目标。
\end{definition}

\section{部分可观测规划}

\subsection{信念状态}

\begin{definition}[信念状态]
    当智能体不能完全观测环境状态时,它维护一个\keyterm{信念状态}(Belief State),表示对当前可能状态的概率分布或集合。
\end{definition}

\subsection{信念空间搜索}

在信念空间中搜索,节点是信念状态,边是动作和观测的组合。

\begin{example}[传感器受限的机器人导航]
\label{ex:robot-navigation}
    一个机器人在$4 \times 4$网格中导航,但传感器只能检测相邻格子是否有障碍物,不能确定自己的精确位置。

    \textbf{信念状态}:可能位置的集合

    \textbf{动作}:上、下、左、右移动

    \textbf{观测}:相邻四个方向的障碍物情况

    机器人需要在不确定自己位置的情况下,规划到达目标区域的策略。通过执行动作和获取观测,逐渐缩小信念状态,最终确定位置并到达目标。
\end{example}

\section{概率规划}

\subsection{马尔可夫决策过程(MDP)}

\begin{definition}[MDP]
    \keyterm{马尔可夫决策过程}定义为五元组 $\langle S, A, T, R, \gamma \rangle$:
    \begin{itemize}
        \item $S$:状态空间
        \item $A$:动作空间
        \item $T: S \times A \times S \to [0,1]$:转移概率函数
        \item $R: S \times A \to \mathbb{R}$:奖励函数
        \item $\gamma \in [0,1)$:折扣因子
    \end{itemize}
\end{definition}

\subsection{部分可观测MDP(POMDP)}

\begin{definition}[POMDP]
    \keyterm{POMDP}在MDP基础上增加:
    \begin{itemize}
        \item $\Omega$:观测空间
        \item $O: S \times A \times \Omega \to [0,1]$:观测概率函数
    \end{itemize}
\end{definition}

\subsection{值迭代与策略迭代}

\begin{algorithm}[H]
    \caption{值迭代算法}
    \label{alg:value-iteration}
    \KwIn{MDP $\langle S, A, T, R, \gamma \rangle$}
    \KwOut{最优值函数 $V^*$}

    初始化 $V(s) = 0$ 对所有 $s \in S$\;
    \Repeat{收敛}{
        \ForEach{$s \in S$}{
            $V(s) \gets \max_{a \in A} \left[ R(s,a) + \gamma \sum_{s' \in S} T(s,a,s') V(s') \right]$\;
        }
    }
    \KwRet{$V$}
\end{algorithm}

\begin{example}[无人机任务规划中的不确定性处理]
\label{ex:uav-uncertainty}
    无人机侦察任务中存在多种不确定性:
    \begin{itemize}
        \item \textbf{天气不确定性}:风速、能见度可能变化
        \item \textbf{目标不确定性}:目标可能移动或隐藏
        \item \textbf{传感器不确定性}:侦察可能失败
        \item \textbf{威胁不确定性}:敌方防空系统位置不确定
    \end{itemize}

    使用POMDP建模:
    \begin{itemize}
        \item 状态包括:无人机位置、剩余燃料、目标状态、威胁状态
        \item 动作包括:飞向某区域、执行侦察、返航
        \item 观测包括:是否发现目标、是否被探测
        \item 奖励设计:发现目标获得正奖励,被击落获得大负奖励
    \end{itemize}
\end{example}

\section*{本章小结}

本章介绍了不确定性规划的主要方法。条件规划和一致性规划处理非确定性。信念状态用于表示部分可观测环境。MDP和POMDP提供了处理概率不确定性的数学框架。

\section*{习题}

\begin{enumerate}
    \item 设计一个需要条件规划的问题场景,画出条件计划树。

    \item 对于示例\ref{ex:robot-navigation},设计一个缩小信念状态的动作序列。

    \item 用MDP建模一个简单的路径规划问题,其中动作有时会失败。

    \item 比较MDP和POMDP的计算复杂性差异。

    \item 实现值迭代算法,求解一个$5 \times 5$网格世界MDP。
\end{enumerate}

% 第9章 多智能体规划
\chapter{多智能体规划}
\label{chap:multiagent}

本章讨论多个智能体协同完成任务的规划问题。

\section{多智能体系统基础}

\subsection{智能体架构}

\begin{definition}[智能体]
    \keyterm{智能体}(Agent)是一个能够感知环境并采取行动以实现目标的自主实体。智能体架构包括:
    \begin{itemize}
        \item 感知模块:获取环境信息
        \item 决策模块:选择行动
        \item 执行模块:执行选定的行动
        \item 通信模块:与其他智能体交互
    \end{itemize}
\end{definition}

\subsection{协调与通信}

多智能体系统的核心挑战是协调:
\begin{itemize}
    \item \textbf{集中式协调}:存在中央协调器
    \item \textbf{分布式协调}:智能体自主协商
    \item \textbf{隐式协调}:通过环境间接交互
\end{itemize}

\section{分布式规划}

\subsection{任务分配}

\begin{definition}[任务分配问题]
    给定$n$个智能体和$m$个任务,\keyterm{任务分配}问题是找到一个分配方案,使得总效用最大化或总代价最小化。
\end{definition}

常用算法:
\begin{itemize}
    \item 拍卖算法
    \item 匈牙利算法
    \item 合同网协议
\end{itemize}

\subsection{计划合并}

当每个智能体独立生成计划后,需要合并这些计划以避免冲突。

\section{多智能体路径规划(MAPF)}

\subsection{MAPF问题定义}

\begin{definition}[MAPF]
    \keyterm{多智能体路径规划}(Multi-Agent Path Finding)问题:给定$n$个智能体,每个智能体有起点和终点,在图上找到无冲突的路径使所有智能体到达各自目标。
\end{definition}

冲突类型:
\begin{itemize}
    \item \textbf{顶点冲突}:两个智能体同时占用同一顶点
    \item \textbf{边冲突}:两个智能体同时使用同一条边(相向移动)
\end{itemize}

\subsection{CBS算法}

\begin{definition}[冲突基搜索]
    \keyterm{CBS}(Conflict-Based Search)是一种两层搜索算法:
    \begin{itemize}
        \item 高层:搜索冲突树,识别和解决冲突
        \item 低层:为每个智能体规划满足约束的路径
    \end{itemize}
\end{definition}

\begin{algorithm}[H]
    \caption{CBS算法}
    \label{alg:cbs}
    \KwIn{MAPF问题}
    \KwOut{无冲突路径集合或失败}

    为每个智能体计算最短路径\;
    $\text{root} \gets$ 创建根节点,包含所有初始路径\;
    $\text{OPEN} \gets \{\text{root}\}$\;

    \While{$\text{OPEN}$ 非空}{
        $N \gets \text{OPEN}$ 中代价最小的节点\;
        验证 $N$ 中的路径是否有冲突\;
        \If{无冲突}{
            \KwRet{$N$ 中的路径}
        }
        选择第一个冲突 $(a_i, a_j, v, t)$\;
        \ForEach{智能体 $a \in \{a_i, a_j\}$}{
            创建子节点 $N'$,添加约束"$a$在时刻$t$不能在$v$"\;
            为 $a$ 重新规划满足新约束的路径\;
            \If{路径存在}{
                将 $N'$ 加入 $\text{OPEN}$\;
            }
        }
    }
    \KwRet{失败}
\end{algorithm}

\subsection{优先级规划}

\keyterm{优先级规划}按某种顺序依次为智能体规划路径,后规划的智能体需要避让先规划的智能体。

\begin{example}[仓储机器人协同调度]
\label{ex:warehouse}
    某自动化仓库有50台AGV(自动引导车)负责货物搬运。系统需要:
    \begin{itemize}
        \item 分配搬运任务给AGV
        \item 规划无冲突的移动路径
        \item 处理实时订单和突发情况
    \end{itemize}

    \textbf{挑战}:
    \begin{itemize}
        \item 狭窄通道可能造成死锁
        \item 需要实时重规划
        \item 充电站调度
    \end{itemize}

    \textbf{解决方案}:
    \begin{enumerate}
        \item 使用拍卖机制分配任务
        \item 使用CBS或其变体规划路径
        \item 实现滚动窗口重规划
    \end{enumerate}
\end{example}

\begin{example}[无人机蜂群编队规划]
\label{ex:swarm}
    20架无人机需要从分散位置汇聚形成特定编队,然后保持编队飞向目标。

    \textbf{阶段1:汇聚规划}
    \begin{itemize}
        \item 每架无人机分配编队中的目标位置
        \item 规划无碰撞的汇聚路径
        \item 同步到达时间
    \end{itemize}

    \textbf{阶段2:编队飞行}
    \begin{itemize}
        \item 领航无人机规划主航线
        \item 其他无人机保持相对位置
        \item 处理编队变换指令
    \end{itemize}

    \textbf{阶段3:障碍规避}
    \begin{itemize}
        \item 检测障碍物
        \item 编队收缩或分裂
        \item 通过后恢复编队
    \end{itemize}
\end{example}

\section{博弈论与规划}

\subsection{纳什均衡}

当智能体有各自的目标时,规划问题变成博弈。

\begin{definition}[纳什均衡]
    \keyterm{纳什均衡}是一个策略组合,其中没有智能体能通过单方面改变策略获得更高收益。
\end{definition}

\subsection{对抗规划}

在存在对手的情况下进行规划,需要考虑对手的可能反应。

\section*{本章小结}

本章介绍了多智能体规划的主要问题和方法。任务分配和计划合并是分布式规划的核心问题。MAPF是多智能体路径规划的标准形式化,CBS是求解MAPF的有效算法。博弈论为存在竞争关系的多智能体规划提供了理论基础。

\section*{习题}

\begin{enumerate}
    \item 设计一个3智能体的MAPF问题实例,手工用CBS算法求解。

    \item 比较集中式和分布式多智能体规划的优缺点。

    \item 对于示例\ref{ex:warehouse},讨论如何处理死锁情况。

    \item 实现一个简单的优先级规划算法,并分析其完备性。

    \item 将示例\ref{ex:swarm}建模为MAPF问题,讨论可能的简化假设。
\end{enumerate}

% 第10章 运动规划基础
\chapter{运动规划基础}
\label{chap:motion}

本章介绍机器人运动规划的基本概念和方法。

\section{构型空间}

\subsection{机器人构型表示}

\begin{definition}[构型]
    机器人的\keyterm{构型}(Configuration)是完全描述机器人位置和姿态所需的最小参数集合。构型的维度称为\keyterm{自由度}(DOF)。
\end{definition}

例如:
\begin{itemize}
    \item 平面移动机器人:3 DOF $(x, y, \theta)$
    \item 六轴机械臂:6 DOF $(\theta_1, \theta_2, \ldots, \theta_6)$
    \item 人形机器人:通常 > 20 DOF
\end{itemize}

\subsection{障碍物映射}

\begin{definition}[构型空间障碍物]
    工作空间中的障碍物在构型空间中形成\keyterm{C空间障碍物}(C-obstacle),机器人的任何构型如果导致与障碍物碰撞,则属于C空间障碍物。
\end{definition}

\section{采样规划方法}

\subsection{PRM(概率路线图)}

\begin{definition}[PRM]
    \keyterm{概率路线图}(Probabilistic Roadmap)方法分两个阶段:
    \begin{enumerate}
        \item \textbf{学习阶段}:随机采样构型,连接近邻形成路线图
        \item \textbf{查询阶段}:将起点和终点连接到路线图,搜索路径
    \end{enumerate}
\end{definition}

\subsection{RRT(快速探索随机树)}

\begin{definition}[RRT]
    \keyterm{快速探索随机树}(Rapidly-exploring Random Tree)通过增量式构建搜索树来探索构型空间。
\end{definition}

\begin{algorithm}[H]
    \caption{基本RRT算法}
    \label{alg:rrt}
    \KwIn{起点 $q_{\text{init}}$,终点 $q_{\text{goal}}$}
    \KwOut{从起点到终点的路径}

    $T \gets$ 初始化树,包含 $q_{\text{init}}$\;
    \For{$k = 1$ \KwTo $K$}{
        $q_{\text{rand}} \gets$ 随机采样(偶尔采样 $q_{\text{goal}}$)\;
        $q_{\text{near}} \gets$ 树中距离 $q_{\text{rand}}$ 最近的节点\;
        $q_{\text{new}} \gets$ 从 $q_{\text{near}}$ 向 $q_{\text{rand}}$ 方向扩展一步\;
        \If{路径 $(q_{\text{near}}, q_{\text{new}})$ 无碰撞}{
            将 $q_{\text{new}}$ 加入树\;
            \If{$q_{\text{new}}$ 接近 $q_{\text{goal}}$}{
                \KwRet{构造路径}
            }
        }
    }
    \KwRet{失败}
\end{algorithm}

\subsection{RRT*与渐进最优性}

\begin{definition}[RRT*]
    \keyterm{RRT*}是RRT的改进版本,通过重新布线(rewiring)操作保证渐进最优性。
\end{definition}

\begin{example}[机械臂运动规划]
\label{ex:manipulator}
    一个6自由度工业机械臂需要将工件从A点移动到B点,工作空间中有障碍物。

    \textbf{构型空间}:$\mathbb{R}^6$,每个维度对应一个关节角度

    \textbf{约束}:
    \begin{itemize}
        \item 关节角度限制
        \item 避免碰撞(与障碍物、与自身)
        \item 奇异点避免
    \end{itemize}

    \textbf{使用RRT规划}:
    \begin{enumerate}
        \item 在6维构型空间中随机采样
        \item 使用正向运动学计算末端位置
        \item 使用碰撞检测验证路径
        \item 路径平滑处理
    \end{enumerate}
\end{example}

\section{任务与运动规划集成(TAMP)}

\subsection{符号-几何接口}

\keyterm{TAMP}(Task and Motion Planning)的挑战在于连接符号层面的任务规划和几何层面的运动规划。

关键问题:
\begin{itemize}
    \item 符号动作的几何可行性检验
    \item 几何约束的符号化表示
    \item 规划层次之间的信息传递
\end{itemize}

\subsection{PDDLStream}

\begin{definition}[PDDLStream]
    \keyterm{PDDLStream}扩展了PDDL,引入了流(Stream)来生成连续参数(如位姿、轨迹)。
\end{definition}

\begin{example}[服务机器人抓取规划]
\label{ex:manipulation}
    服务机器人需要从桌上抓取一个杯子并放入橱柜。

    \textbf{任务层面}(符号规划):
    \begin{enumerate}
        \item 移动到桌子附近
        \item 抓取杯子
        \item 移动到橱柜附近
        \item 打开橱柜门
        \item 放入杯子
        \item 关闭橱柜门
    \end{enumerate}

    \textbf{运动层面}(几何规划):
    \begin{itemize}
        \item 导航路径规划
        \item 抓取位姿采样
        \item 机械臂运动规划
        \item 门把手操作轨迹
    \end{itemize}

    \textbf{集成挑战}:
    \begin{itemize}
        \item 某些抓取位姿可能因碰撞不可行
        \item 放置位置影响后续操作
        \item 需要同时考虑导航和操作
    \end{itemize}
\end{example}

\section*{本章小结}

本章介绍了运动规划的基本概念和方法。构型空间是运动规划的数学基础。PRM和RRT是两类主要的采样规划方法。TAMP集成了任务规划和运动规划,是机器人规划的前沿方向。

\section*{习题}

\begin{enumerate}
    \item 对于一个在$10 \times 10$网格中的移动机器人,比较A*和RRT的性能。

    \item 证明RRT*具有渐进最优性。

    \item 为示例\ref{ex:manipulator}设计碰撞检测算法。

    \item 讨论TAMP中符号规划和运动规划如何交互。

    \item 实现基本的RRT算法,在2D空间中规划路径。
\end{enumerate}


% ==================== 第三篇:应用篇 ====================
\part{应用篇}

% 第11章 交通运输规划
\chapter{交通运输规划}
\label{chap:transportation}

本章介绍任务规划在交通运输领域的应用。

\section{车辆路径问题(VRP)}

\subsection{基本VRP模型}

\begin{definition}[车辆路径问题]
    \keyterm{车辆路径问题}(Vehicle Routing Problem,VRP):给定一个配送中心和若干客户点,确定车辆的配送路线,使得所有客户的需求得到满足,同时优化某个目标(如总行驶距离最短)。
\end{definition}

VRP可以形式化为:
\begin{align}
    \min \quad & \sum_{i,j,k} c_{ij} x_{ijk} \\
    \text{s.t.} \quad & \sum_{k} \sum_{j} x_{ijk} = 1, \quad \forall i \text{(每个客户恰好被访问一次)} \\
    & \sum_{i} d_i \sum_{j} x_{ijk} \leq Q_k, \quad \forall k \text{(车辆容量约束)} \\
    & x_{ijk} \in \{0,1\}
\end{align}

\subsection{带时间窗的VRP}

\begin{definition}[VRPTW]
    \keyterm{带时间窗的VRP}(VRP with Time Windows)要求车辆在客户指定的时间范围内到达并提供服务。
\end{definition}

\subsection{带容量约束的VRP}

\keyterm{CVRP}(Capacitated VRP)考虑车辆的载重限制。

\section{求解方法}

\subsection{精确算法}

\begin{itemize}
    \item 分支定界法
    \item 分支切割法
    \item 列生成法
\end{itemize}

精确算法能够保证找到最优解,但对于大规模问题计算时间过长。

\subsection{启发式与元启发式}

\begin{itemize}
    \item \textbf{构造启发式}:最近邻、节约算法、插入法
    \item \textbf{改进启发式}:2-opt、3-opt、Or-opt
    \item \textbf{元启发式}:遗传算法、模拟退火、禁忌搜索、蚁群优化
\end{itemize}

\subsection{混合智能方法}

结合精确方法和启发式的优点,如:
\begin{itemize}
    \item 大规模邻域搜索(LNS)
    \item 自适应大规模邻域搜索(ALNS)
    \item 混合遗传算法
\end{itemize}

\begin{example}[快递配送路径优化]
\label{ex:delivery}
    某快递公司早班需要配送200个包裹到城区各处。

    \textbf{已知条件}:
    \begin{itemize}
        \item 配送中心位置
        \item 200个配送点的位置和包裹重量
        \item 10辆配送车,每辆载重500kg
        \item 部分配送点有时间要求
    \end{itemize}

    \textbf{优化目标}:最小化总行驶距离

    \textbf{求解过程}:
    \begin{enumerate}
        \item 使用聚类算法初步划分区域
        \item 为每个区域用节约算法构造初始路线
        \item 使用ALNS进行改进
        \item 检验时间窗约束,调整路线
    \end{enumerate}

    \textbf{优化结果}:相比人工规划,总行驶距离减少15\%,配送时间缩短20\%。
\end{example}

\begin{example}[公交线路规划]
\label{ex:bus}
    某城市需要设计新的公交线网。

    \textbf{输入数据}:
    \begin{itemize}
        \item 城市道路网络
        \item 居民出行OD矩阵
        \item 现有公交站点
        \item 车辆配置
    \end{itemize}

    \textbf{规划内容}:
    \begin{enumerate}
        \item 线路走向设计
        \item 站点设置
        \item 发车频率确定
        \item 车辆调度计划
    \end{enumerate}

    \textbf{优化目标}:最大化乘客覆盖率,最小化总运营成本。
\end{example}

\section{动态路径规划}

\subsection{实时重规划}

在实际运营中,经常需要根据实时情况调整路线:
\begin{itemize}
    \item 新订单插入
    \item 交通拥堵
    \item 车辆故障
    \item 客户取消
\end{itemize}

\subsection{交通流预测}

利用历史数据和实时数据预测交通流,提前调整路线。

\begin{example}[网约车调度系统]
\label{ex:ridesharing}
    网约车平台需要实时匹配乘客和司机。

    \textbf{核心问题}:
    \begin{itemize}
        \item 订单分配:哪个司机接哪个乘客
        \item 路径规划:如何到达乘客位置和目的地
        \item 拼车匹配:多个乘客如何共乘
        \item 供需调度:如何引导空闲车辆
    \end{itemize}

    \textbf{算法特点}:
    \begin{itemize}
        \item 毫秒级响应要求
        \item 考虑未来需求预测
        \item 平衡效率和公平性
    \end{itemize}
\end{example}

\section{多式联运规划}

\begin{example}[集装箱多式联运优化]
\label{ex:intermodal}
    某物流公司需要将货物从上海港运送到成都仓库。

    \textbf{可选方式}:
    \begin{itemize}
        \item 全程公路运输(快但贵)
        \item 铁路+公路(经济但慢)
        \item 水运+铁路+公路(最经济但最慢)
    \end{itemize}

    \textbf{决策因素}:
    \begin{itemize}
        \item 货物时效性要求
        \item 各环节成本
        \item 转运时间和成本
        \item 运力可用性
    \end{itemize}

    \textbf{优化模型}:多目标规划,平衡成本、时间和可靠性。
\end{example}

\section*{本章小结}

本章介绍了任务规划在交通运输领域的应用。VRP是物流配送的核心问题,有多种变体和求解方法。动态路径规划处理实时变化的情况。多式联运规划优化不同运输方式的组合。

\section*{习题}

\begin{enumerate}
    \item 用节约算法为一个10客户的VRP问题构造初始解。

    \item 实现2-opt改进算法,改进TSP问题的初始解。

    \item 对于示例\ref{ex:delivery},讨论如何处理配送过程中的新订单。

    \item 比较遗传算法和禁忌搜索在VRP问题上的性能。

    \item 设计一个简单的网约车匹配算法,考虑等待时间和绕路距离。
\end{enumerate}

% 第12章 通信与编码传输规划
\chapter{通信与编码传输规划}
\label{chap:communication}

本章介绍任务规划在通信网络领域的应用。

\section{通信网络资源调度}

\subsection{信道分配}

\begin{definition}[信道分配问题]
    在无线通信网络中,\keyterm{信道分配}问题是将有限的频谱资源分配给各用户或基站,以最大化系统容量并最小化干扰。
\end{definition}

信道分配可以建模为图着色问题:
\begin{itemize}
    \item 节点表示通信链路
    \item 边表示干扰关系
    \item 颜色表示信道
\end{itemize}

\subsection{功率控制}

功率控制问题是确定每个发射机的发射功率,以满足通信质量要求并最小化总功耗或干扰。

\section{无线传感器网络任务规划}

\subsection{数据采集调度}

\begin{definition}[数据采集调度]
    在无线传感器网络中,\keyterm{数据采集调度}确定每个传感器节点何时进行数据采集、处理和传输,以平衡数据时效性和能量消耗。
\end{definition}

\subsection{能量感知路由}

由于传感器节点通常由电池供电,路由协议需要考虑能量因素:
\begin{itemize}
    \item 最小能量路由
    \item 能量均衡路由
    \item 基于剩余能量的路由
\end{itemize}

\begin{example}[物联网数据汇聚规划]
\label{ex:iot-aggregation}
    某智慧农业系统部署了1000个土壤传感器,需要定期向网关汇报数据。

    \textbf{约束条件}:
    \begin{itemize}
        \item 每个传感器电池容量有限
        \item 传输距离影响能耗
        \item 数据需要在规定时间内到达
        \item 部分节点可以进行数据聚合
    \end{itemize}

    \textbf{规划内容}:
    \begin{enumerate}
        \item 确定数据汇聚树结构
        \item 调度每个节点的醒睡周期
        \item 确定数据聚合点
        \item 优化传输时隙分配
    \end{enumerate}

    \textbf{优化目标}:最大化网络寿命,保证数据时效性。
\end{example}

\section{卫星通信任务规划}

\subsection{卫星过境调度}

\begin{definition}[卫星过境调度]
    地面站需要与多颗卫星通信,但视窗时间有限。\keyterm{卫星过境调度}问题是确定地面站与哪颗卫星在何时通信。
\end{definition}

\subsection{星地链路规划}

\begin{example}[遥感卫星成像任务规划]
\label{ex:satellite-imaging}
    某遥感卫星星座需要完成100个地面目标的成像任务。

    \textbf{约束条件}:
    \begin{itemize}
        \item 每颗卫星的轨道决定了可观测时间窗
        \item 卫星姿态调整需要时间
        \item 星上存储容量有限
        \item 数据需要在一定时间内下传
    \end{itemize}

    \textbf{规划决策}:
    \begin{enumerate}
        \item 任务分配:哪颗卫星观测哪个目标
        \item 观测调度:何时进行观测
        \item 数据下传:何时通过哪个地面站下传
    \end{enumerate}

    \textbf{优化目标}:最大化完成的任务数,优先满足高优先级任务。

    \textbf{规划算法}:
    \begin{itemize}
        \item 基于约束满足的方法
        \item 启发式搜索
        \item 遗传算法
    \end{itemize}
\end{example}

\section{数据中心任务调度}

\begin{example}[云计算资源编排]
\label{ex:cloud-orchestration}
    云计算平台需要调度大量虚拟机和容器。

    \textbf{任务类型}:
    \begin{itemize}
        \item 批处理任务:可延迟执行
        \item 交互式任务:需要快速响应
        \item 流式任务:持续执行
    \end{itemize}

    \textbf{资源约束}:
    \begin{itemize}
        \item CPU、内存、存储、网络带宽
        \item 物理机的容量限制
        \item 数据局部性要求
        \item 服务质量(QoS)保证
    \end{itemize}

    \textbf{调度目标}:
    \begin{itemize}
        \item 最小化任务完成时间
        \item 最大化资源利用率
        \item 最小化能耗
        \item 保证服务等级协议(SLA)
    \end{itemize}

    \textbf{调度策略}:
    \begin{enumerate}
        \item 基于优先级的调度
        \item 基于公平性的调度
        \item 基于预测的调度
        \item 基于强化学习的调度
    \end{enumerate}
\end{example}

\section*{本章小结}

本章介绍了任务规划在通信领域的应用。信道分配和功率控制是无线网络的基本优化问题。无线传感器网络需要考虑能量约束。卫星通信和数据中心调度是大规模资源分配问题。

\section*{习题}

\begin{enumerate}
    \item 将一个简单的信道分配问题建模为图着色问题并求解。

    \item 设计一个能量感知的路由算法,分析其对网络寿命的影响。

    \item 对于示例\ref{ex:satellite-imaging},用整数规划建模并求解一个小规模实例。

    \item 比较不同云计算调度策略的优缺点。

    \item 设计一个物联网数据采集调度算法,考虑能量和时延约束。
\end{enumerate}

% 第13章 军事任务规划
\chapter{军事任务规划}
\label{chap:military}

本章介绍任务规划在军事领域的应用。

\section{军事规划概述}

\subsection{作战规划层次}

军事规划通常分为多个层次:
\begin{itemize}
    \item \textbf{战略层}:国家级战略目标和资源分配
    \item \textbf{战役层}:战区级作战计划
    \item \textbf{战术层}:具体作战行动规划
    \item \textbf{技术层}:武器系统和平台控制
\end{itemize}

\subsection{C4ISR系统}

\begin{definition}[C4ISR]
    \keyterm{C4ISR}代表指挥(Command)、控制(Control)、通信(Communication)、计算机(Computer)、情报(Intelligence)、监视(Surveillance)和侦察(Reconnaissance)。
\end{definition}

任务规划系统是C4ISR的核心组件之一。

\section{兵力部署规划}

\subsection{力量配置优化}

\begin{definition}[兵力部署问题]
    给定作战目标和可用兵力,确定各作战单元的部署位置和任务分配,以最大化作战效能。
\end{definition}

\subsection{后勤保障规划}

后勤保障规划包括:
\begin{itemize}
    \item 物资补给调度
    \item 运输路线规划
    \item 维修保障安排
    \item 医疗后送计划
\end{itemize}

\begin{example}[两栖登陆作战兵力部署]
\label{ex:amphibious}
    某两栖作战需要在指定海岸登陆并建立滩头阵地。

    \textbf{可用兵力}:
    \begin{itemize}
        \item 3个海军陆战队营
        \item 2个装甲连
        \item 空中支援力量
        \item 舰炮火力支援
    \end{itemize}

    \textbf{规划内容}:
    \begin{enumerate}
        \item 登陆点选择
        \item 各波次兵力编组
        \item 火力支援计划
        \item 后续梯队跟进时机
    \end{enumerate}

    \textbf{约束条件}:
    \begin{itemize}
        \item 滩头地形限制
        \item 敌方防御部署
        \item 潮汐和天气窗口
        \item 登陆舰艇容量
    \end{itemize}
\end{example}

\section{无人系统任务规划}

\subsection{无人机航迹规划}

\begin{definition}[航迹规划]
    为无人机确定从起点到目标再返回的飞行路径,需要考虑威胁规避、燃料约束、任务时限等因素。
\end{definition}

航迹规划方法:
\begin{itemize}
    \item Voronoi图方法
    \item A*及其变体
    \item RRT方法
    \item 势场法
\end{itemize}

\subsection{无人车任务分配}

无人地面车辆(UGV)任务规划包括:
\begin{itemize}
    \item 巡逻路线规划
    \item 侦察任务分配
    \item 物资运输调度
\end{itemize}

\subsection{有人-无人协同}

\keyterm{MUM-T}(Manned-Unmanned Teaming)是现代军事的重要作战概念:
\begin{itemize}
    \item 有人平台指挥控制
    \item 无人平台执行危险任务
    \item 协同态势感知
    \item 任务动态分配
\end{itemize}

\begin{example}[无人机侦察监视任务]
\label{ex:uav-reconnaissance}
    使用4架侦察无人机对某区域进行持续监视。

    \textbf{任务要求}:
    \begin{itemize}
        \item 监视区域面积:$100 \times 100$ km$^2$
        \item 监视周期:24小时
        \item 重点目标需要持续监视
        \item 发现可疑目标需要详细侦察
    \end{itemize}

    \textbf{规划内容}:
    \begin{enumerate}
        \item 巡逻航线设计
        \item 换班和加油调度
        \item 应急任务响应
        \item 通信中继安排
    \end{enumerate}

    \textbf{优化目标}:最大化区域覆盖率,最小化目标发现延迟。
\end{example}

\begin{example}[无人机蜂群协同打击]
\label{ex:swarm-attack}
    无人机蜂群对多个目标实施协同打击。

    \textbf{任务场景}:
    \begin{itemize}
        \item 30架察打一体无人机
        \item 10个已知目标
        \item 未知的防空威胁
        \item 通信可能被干扰
    \end{itemize}

    \textbf{规划层次}:
    \begin{enumerate}
        \item \textbf{任务分配}:确定每架无人机的目标
        \item \textbf{航迹规划}:规划进入和撤离航线
        \item \textbf{协同时序}:同步攻击时间
        \item \textbf{应急处置}:处理损失和新发现目标
    \end{enumerate}

    \textbf{分布式规划}:
    \begin{itemize}
        \item 中央规划初始方案
        \item 各无人机本地调整
        \item 基于共识的协调
        \item 自主重规划能力
    \end{itemize}
\end{example}

\section{电子战任务规划}

\begin{example}[电子干扰任务规划]
\label{ex:ew}
    电子战飞机需要对敌方雷达网络实施干扰。

    \textbf{情报输入}:
    \begin{itemize}
        \item 敌方雷达位置和类型
        \item 雷达工作参数
        \item 我方干扰设备能力
    \end{itemize}

    \textbf{规划内容}:
    \begin{enumerate}
        \item 干扰阵位选择
        \item 干扰样式和时序
        \item 功率和频率分配
        \item 协同配合计划
    \end{enumerate}

    \textbf{效能评估}:
    \begin{itemize}
        \item 雷达探测概率降低程度
        \item 掩护目标的暴露风险
        \item 干扰资源消耗
    \end{itemize}
\end{example}

\section*{本章小结}

本章介绍了任务规划在军事领域的应用。军事规划涵盖多个层次,从战略到技术层面。无人系统任务规划是当前的研究热点,包括航迹规划、任务分配和有人-无人协同。电子战规划是信息化战争的重要组成部分。

\section*{习题}

\begin{enumerate}
    \item 设计一个简单的无人机航迹规划算法,考虑禁飞区和威胁区。

    \item 对于示例\ref{ex:amphibious},分析规划中需要考虑的不确定性因素。

    \item 比较集中式和分布式无人机蜂群规划的优缺点。

    \item 用HTN方法建模一个军事任务分解过程。

    \item 讨论人工智能在军事任务规划中的伦理问题。
\end{enumerate}


% ==================== 第四篇:前沿篇 ====================
\part{前沿篇}

% 第14章 基于学习的规划方法
\chapter{基于学习的规划方法 \grad}
\label{chap:learning}

本章介绍机器学习与规划的结合,属于研究生拓展内容。

\section{强化学习与规划}

\subsection{模型规划RL}

\begin{definition}[基于模型的强化学习]
    \keyterm{基于模型的强化学习}(Model-Based RL)通过学习环境动力学模型,然后在模型上进行规划来做出决策。
\end{definition}

\textbf{基本框架}:
\begin{enumerate}
    \item 与环境交互收集数据
    \item 学习转移模型 $\hat{T}(s'|s,a)$ 和奖励模型 $\hat{R}(s,a)$
    \item 在学习的模型上规划
    \item 执行规划得到的动作
\end{enumerate}

\subsection{AlphaGo与蒙特卡洛树搜索}

AlphaGo结合了深度学习和蒙特卡洛树搜索(MCTS):
\begin{itemize}
    \item \textbf{策略网络}:估计动作概率分布
    \item \textbf{价值网络}:估计局面价值
    \item \textbf{MCTS}:使用网络引导搜索
\end{itemize}

\begin{algorithm}[H]
    \caption{MCTS基本算法}
    \label{alg:mcts}
    \KwIn{根节点状态 $s$}
    \KwOut{最佳动作}

    \For{$i = 1$ \KwTo $N$}{
        $\text{node} \gets \text{root}$\;
        \tcp{选择阶段}
        \While{$\text{node}$ 完全展开且非叶节点}{
            $\text{node} \gets$ UCB1选择的子节点\;
        }
        \tcp{展开阶段}
        \If{$\text{node}$ 未完全展开}{
            $\text{node} \gets$ 展开一个未尝试的动作\;
        }
        \tcp{模拟阶段}
        $v \gets$ 随机模拟到终局\;
        \tcp{回传阶段}
        回传 $v$ 更新路径上所有节点的统计\;
    }
    \KwRet{访问次数最多的动作}
\end{algorithm}

\section{模仿学习}

\subsection{行为克隆}

\begin{definition}[行为克隆]
    \keyterm{行为克隆}(Behavior Cloning)直接从专家演示中学习策略,将状态-动作对作为监督学习问题。
\end{definition}

问题:分布漂移(Distribution Shift)——训练时的状态分布与执行时不同。

\subsection{逆强化学习}

\begin{definition}[逆强化学习]
    \keyterm{逆强化学习}(Inverse Reinforcement Learning,IRL)从专家演示中推断奖励函数,然后用标准RL方法学习策略。
\end{definition}

\section{神经网络规划器}

\subsection{神经符号规划}

\keyterm{神经符号规划}结合神经网络的学习能力和符号规划的推理能力:
\begin{itemize}
    \item 神经网络处理感知和特征提取
    \item 符号系统处理推理和规划
    \item 接口层连接两个组件
\end{itemize}

\subsection{图神经网络在规划中的应用}

\keyterm{图神经网络}(GNN)可以处理规划问题的结构化表示:
\begin{itemize}
    \item 状态表示为图结构
    \item GNN学习状态特征
    \item 用于启发式估计或策略学习
\end{itemize}

\section{迁移学习与元学习}

\subsection{领域迁移}

在源领域学习的规划知识可以迁移到目标领域:
\begin{itemize}
    \item 启发式函数迁移
    \item 动作模式迁移
    \item 领域知识迁移
\end{itemize}

\subsection{少样本规划}

\keyterm{元学习}使规划器能够快速适应新领域:
\begin{itemize}
    \item 学习如何学习规划
    \item 从少量样本中提取领域特征
    \item 快速调整规划策略
\end{itemize}

\section*{本章小结}

本章介绍了基于学习的规划方法。强化学习与规划的结合是重要的研究方向,AlphaGo是成功案例。模仿学习从专家演示中学习。神经符号规划结合了深度学习和符号推理的优势。迁移学习和元学习帮助规划器泛化到新问题。

\section*{习题}

\begin{enumerate}
    \item 解释为什么AlphaGo需要结合策略网络和价值网络。

    \item 比较行为克隆和逆强化学习的优缺点。

    \item 设计一个用GNN表示规划问题状态的方案。

    \item 讨论神经网络规划器的可解释性问题。

    \item 阅读一篇神经符号规划的最新论文,总结其主要贡献。
\end{enumerate}

% 第15章 大语言模型与任务规划
\chapter{大语言模型与任务规划 \grad}
\label{chap:llm}

本章介绍大语言模型在任务规划中的应用,属于研究生拓展内容。

\section{LLM规划能力分析}

\subsection{LLM的推理能力}

大语言模型(如GPT-4、Claude等)展示了一定的推理和规划能力:
\begin{itemize}
    \item 常识推理
    \item 多步骤问题求解
    \item 代码生成和调试
    \item 任务分解
\end{itemize}

\subsection{规划基准测试(PlanBench)}

\begin{definition}[PlanBench]
    \keyterm{PlanBench}是评估LLM规划能力的基准测试套件,包含基于国际规划竞赛领域的问题。
\end{definition}

PlanBench的测试类别:
\begin{itemize}
    \item 计划生成(零样本/少样本)
    \item 计划验证
    \item 目标识别
    \item 代价优化
\end{itemize}

\subsection{LLM规划的局限性}

研究发现LLM在规划任务上存在系统性不足:
\begin{enumerate}
    \item \textbf{长时域规划困难}:步数增加时性能显著下降
    \item \textbf{计划有效性问题}:生成的计划可能违反约束
    \item \textbf{缺乏形式化保证}:无法确保解的正确性
    \item \textbf{结构化推理不足}:对复杂依赖关系的处理能力有限
\end{enumerate}

\section{LLM作为规划组件}

鉴于LLM作为独立规划器的局限性,研究者探索将LLM作为规划系统的组件。

\subsection{LLM+P:自然语言到PDDL}

\begin{definition}[LLM+P]
    \keyterm{LLM+P}框架使用LLM将自然语言问题描述翻译为PDDL格式,然后由经典规划器求解。
\end{definition}

工作流程:
\begin{enumerate}
    \item 用户用自然语言描述问题
    \item LLM生成PDDL领域和问题文件
    \item 经典规划器(如Fast Downward)求解
    \item LLM将计划翻译回自然语言
\end{enumerate}

\subsection{LLM生成启发式函数}

LLM可以为规划问题生成领域特定的启发式函数:
\begin{itemize}
    \item 输入:PDDL领域描述
    \item 输出:Python实现的启发式函数
    \item 优势:保持规划器的正确性保证
\end{itemize}

\subsection{LLM作为世界模型}

LLM可以作为世界模型,预测动作的效果:
\begin{itemize}
    \item 处理开放领域问题
    \item 利用常识知识
    \item 但可能产生幻觉
\end{itemize}

\section{具身智能规划}

\subsection{SayCan框架}

\begin{definition}[SayCan]
    \keyterm{SayCan}框架结合LLM的语言理解能力和机器人的可供性(affordance)函数:
    \begin{equation}
        \pi(a|s, i) \propto p_{\text{LLM}}(a|i) \cdot p_{\text{affordance}}(a|s)
    \end{equation}
    其中$i$是自然语言指令,$s$是当前状态。
\end{definition}

\subsection{SayPlan与3D场景图}

\begin{definition}[SayPlan]
    \keyterm{SayPlan}使用3D场景图(3DSG)来提供环境的结构化表示,使LLM能够在大规模环境中进行可扩展的任务规划。
\end{definition}

SayPlan的关键创新:
\begin{itemize}
    \item 场景图作为环境表示
    \item 多层次抽象
    \item 语义搜索缩小相关范围
\end{itemize}

\subsection{Inner Monologue}

\begin{definition}[Inner Monologue]
    \keyterm{Inner Monologue}通过环境反馈形成"内心独白",包括:
    \begin{itemize}
        \item 被动场景描述
        \item 主动场景描述
        \item 成功检测反馈
    \end{itemize}
\end{definition}

这种闭环反馈显著提升了机器人任务执行的成功率。

\section{多智能体LLM系统}

\subsection{LLM-MAS架构}

多智能体LLM系统的典型架构:
\begin{itemize}
    \item 每个智能体由一个LLM驱动
    \item 智能体具有不同的角色和专长
    \item 通过对话进行协调
    \item 共同完成复杂任务
\end{itemize}

\subsection{智能体协作框架}

代表性框架:
\begin{itemize}
    \item \textbf{AutoGen}:微软的多智能体对话框架
    \item \textbf{CAMEL}:角色扮演对话框架
    \item \textbf{MetaGPT}:软件开发多智能体系统
\end{itemize}

\section*{本章小结}

本章介绍了大语言模型与任务规划的结合。LLM展示了一定的规划能力,但存在系统性局限。将LLM作为规划系统的组件(如翻译器、启发式生成器)是更有效的方法。具身智能规划框架(SayCan、SayPlan)展示了LLM在机器人领域的应用。多智能体LLM系统是新兴的研究方向。

\section*{习题}

\begin{enumerate}
    \item 分析LLM在长时域规划中性能下降的原因。

    \item 用LLM将一个简单的规划问题翻译为PDDL,并验证翻译的正确性。

    \item 比较LLM+P和纯LLM规划的优缺点。

    \item 讨论SayCan中可供性函数的作用。

    \item 阅读一篇多智能体LLM系统的论文,总结其协调机制。
\end{enumerate}

% 第16章 规划系统的可解释性与安全性
\chapter{规划系统的可解释性与安全性 \grad}
\label{chap:safety}

本章讨论规划系统的可解释性和安全性问题,属于研究生拓展内容。

\section{可解释规划}

\subsection{计划解释生成}

\begin{definition}[计划解释]
    \keyterm{计划解释}是向用户说明为什么规划器生成了特定的计划,以及为什么没有选择其他方案。
\end{definition}

解释类型:
\begin{itemize}
    \item \textbf{对比解释}:为什么选择动作A而不是B?
    \item \textbf{因果解释}:这个动作导致了什么结果?
    \item \textbf{目标解释}:这个动作如何帮助实现目标?
\end{itemize}

\subsection{人机协同规划}

人机协同规划中,系统需要:
\begin{itemize}
    \item 理解人类的意图和偏好
    \item 解释自己的决策
    \item 接受人类的反馈和修正
    \item 保持适当的自主性
\end{itemize}

\begin{definition}[混合主动规划]
    \keyterm{混合主动规划}(Mixed-Initiative Planning)中,人类和AI系统共同参与规划过程,各自贡献专长。
\end{definition}

\section{规划系统验证}

\subsection{形式化验证方法}

\begin{definition}[形式化验证]
    \keyterm{形式化验证}使用数学方法证明规划系统满足特定性质。
\end{definition}

验证目标包括:
\begin{itemize}
    \item \textbf{正确性}:计划能够达到目标
    \item \textbf{完整性}:如果解存在,系统能找到
    \item \textbf{最优性}:找到的解是最优的
    \item \textbf{安全性}:执行过程中不会进入危险状态
\end{itemize}

\subsection{模型检测}

\begin{definition}[模型检测]
    \keyterm{模型检测}(Model Checking)自动验证系统模型是否满足给定的时态逻辑性质。
\end{definition}

在规划中的应用:
\begin{itemize}
    \item 验证计划满足安全约束
    \item 检测死锁和活锁
    \item 验证时序性质
\end{itemize}

\section{安全关键系统规划}

\subsection{安全约束规划}

在安全关键系统中,规划必须满足硬性安全约束:
\begin{itemize}
    \item 机器人必须避免碰撞
    \item 车辆必须遵守交通规则
    \item 医疗系统必须避免有害操作
\end{itemize}

\begin{definition}[安全规划]
    \keyterm{安全规划}生成的计划保证在执行过程中永远不会进入不安全状态。
\end{definition}

方法:
\begin{itemize}
    \item 将安全约束编码到规划问题中
    \item 使用屏障函数保证安全
    \item 在线监控和干预
\end{itemize}

\subsection{鲁棒性分析}

分析规划系统在以下情况下的表现:
\begin{itemize}
    \item 模型不准确
    \item 执行误差
    \item 环境变化
    \item 对抗性干扰
\end{itemize}

\section{伦理与法规}

\subsection{自主系统伦理}

自主规划系统面临的伦理问题:
\begin{itemize}
    \item \textbf{责任归属}:当自主系统造成损害时,谁负责?
    \item \textbf{透明度}:用户是否知道系统如何做决策?
    \item \textbf{公平性}:决策是否存在偏见?
    \item \textbf{自主性边界}:系统何时应该请求人类干预?
\end{itemize}

\subsection{军事AI治理}

军事AI规划系统的特殊考虑:
\begin{itemize}
    \item \textbf{人类控制}:致命性决策必须有人类参与
    \item \textbf{国际法遵从}:遵守武装冲突法
    \item \textbf{责任链}:明确指挥和责任关系
    \item \textbf{可预测性}:行为应当可预测和可解释
\end{itemize}

\begin{example}[自主驾驶的伦理困境]
    自动驾驶车辆面临不可避免的碰撞时,应该如何决策?

    \textbf{场景}:车辆前方突然出现行人,左侧是护栏,右侧是其他行人。

    \textbf{伦理问题}:
    \begin{itemize}
        \item 是否可以"选择"伤害谁?
        \item 是否应该保护乘客优先?
        \item 如何量化不同选择的"代价"?
    \end{itemize}

    \textbf{技术考虑}:
    \begin{itemize}
        \item 规划系统如何表示这类约束?
        \item 决策过程如何做到可解释?
        \item 如何满足法规要求?
    \end{itemize}
\end{example}

\section*{本章小结}

本章讨论了规划系统的可解释性和安全性。可解释规划帮助用户理解和信任系统决策。形式化验证和模型检测为规划系统提供正确性保证。安全关键系统需要特别的规划方法。伦理和法规问题是自主系统部署的重要考虑。

\section*{习题}

\begin{enumerate}
    \item 为一个简单的规划问题设计解释生成方法。

    \item 讨论人机协同规划中的信任问题。

    \item 用模型检测验证一个简单规划系统的安全性质。

    \item 分析自动驾驶规划系统的伦理设计原则。

    \item 讨论军事AI规划系统应该遵循的原则。
\end{enumerate}


% ==================== 附录 ====================
\appendix
\part*{附录}
\addcontentsline{toc}{part}{附录}

% 附录A PDDL语言参考手册
\chapter{PDDL语言参考手册}
\label{app:pddl}

本附录提供PDDL(Planning Domain Definition Language)的语法参考。

\section{PDDL基本结构}

\subsection{领域文件结构}

\begin{lstlisting}[language=PDDL]
(define (domain <domain-name>)
  (:requirements <requirement-flags>)
  (:types <type-definitions>)
  (:constants <constant-definitions>)
  (:predicates <predicate-definitions>)
  (:functions <function-definitions>)
  (:action <action-definition>)*
)
\end{lstlisting}

\subsection{问题文件结构}

\begin{lstlisting}[language=PDDL]
(define (problem <problem-name>)
  (:domain <domain-name>)
  (:objects <object-definitions>)
  (:init <initial-state>)
  (:goal <goal-specification>)
  (:metric <metric-specification>)
)
\end{lstlisting}

\section{需求标志}

常用需求标志:
\begin{itemize}
    \item \texttt{:strips} -- 基本STRIPS功能
    \item \texttt{:typing} -- 类型系统
    \item \texttt{:negative-preconditions} -- 负前提条件
    \item \texttt{:disjunctive-preconditions} -- 析取前提条件
    \item \texttt{:equality} -- 等式判断
    \item \texttt{:existential-preconditions} -- 存在量词
    \item \texttt{:universal-preconditions} -- 全称量词
    \item \texttt{:conditional-effects} -- 条件效果
    \item \texttt{:numeric-fluents} -- 数值变量
    \item \texttt{:durative-actions} -- 持续性动作
    \item \texttt{:duration-inequalities} -- 持续时间不等式
    \item \texttt{:continuous-effects} -- 连续效果
\end{itemize}

\section{类型定义}

\begin{lstlisting}[language=PDDL]
(:types
  location city - object
  truck airplane - vehicle
  package - object
)
\end{lstlisting}

\section{谓词定义}

\begin{lstlisting}[language=PDDL]
(:predicates
  (at ?obj - object ?loc - location)
  (in ?pkg - package ?veh - vehicle)
  (connected ?from ?to - location)
)
\end{lstlisting}

\section{函数定义}

\begin{lstlisting}[language=PDDL]
(:functions
  (distance ?from ?to - location)
  (fuel ?v - vehicle)
  (total-cost)
)
\end{lstlisting}

\section{动作定义}

\subsection{基本动作}

\begin{lstlisting}[language=PDDL]
(:action drive
  :parameters (?t - truck ?from ?to - location)
  :precondition (and
    (at ?t ?from)
    (connected ?from ?to)
  )
  :effect (and
    (not (at ?t ?from))
    (at ?t ?to)
    (increase (total-cost) (distance ?from ?to))
  )
)
\end{lstlisting}

\subsection{持续性动作(PDDL 2.1)}

\begin{lstlisting}[language=PDDL]
(:durative-action fly
  :parameters (?a - airplane ?from ?to - city)
  :duration (= ?duration (/ (distance ?from ?to) (speed ?a)))
  :condition (and
    (at start (at ?a ?from))
    (at start (>= (fuel ?a) (fuel-required ?from ?to)))
    (over all (available ?to))
  )
  :effect (and
    (at start (not (at ?a ?from)))
    (at end (at ?a ?to))
    (at start (decrease (fuel ?a) (fuel-required ?from ?to)))
  )
)
\end{lstlisting}

\section{目标规范}

\subsection{简单目标}

\begin{lstlisting}[language=PDDL]
(:goal (and
  (at pkg1 loc-b)
  (at pkg2 loc-c)
))
\end{lstlisting}

\subsection{带偏好的目标(PDDL 3.0)}

\begin{lstlisting}[language=PDDL]
(:goal (and
  (at pkg1 loc-b)
  (preference p1 (at pkg2 loc-c))
))

(:metric minimize (+ (total-cost) (* 100 (is-violated p1))))
\end{lstlisting}

\section{度量规范}

\begin{lstlisting}[language=PDDL]
(:metric minimize (total-cost))
(:metric maximize (packages-delivered))
(:metric minimize (total-time))
\end{lstlisting}

\section{完整示例}

\subsection{物流领域}

\begin{lstlisting}[language=PDDL]
(define (domain logistics)
  (:requirements :strips :typing)

  (:types
    city location thing - object
    package vehicle - thing
    truck airplane - vehicle
    airport - location
  )

  (:predicates
    (in-city ?loc - location ?city - city)
    (at ?obj - thing ?loc - location)
    (in ?pkg - package ?veh - vehicle)
  )

  (:action load-truck
    :parameters (?pkg - package ?truck - truck ?loc - location)
    :precondition (and (at ?truck ?loc) (at ?pkg ?loc))
    :effect (and (not (at ?pkg ?loc)) (in ?pkg ?truck))
  )

  (:action unload-truck
    :parameters (?pkg - package ?truck - truck ?loc - location)
    :precondition (and (at ?truck ?loc) (in ?pkg ?truck))
    :effect (and (not (in ?pkg ?truck)) (at ?pkg ?loc))
  )

  (:action drive-truck
    :parameters (?truck - truck ?from ?to - location ?city - city)
    :precondition (and
      (at ?truck ?from)
      (in-city ?from ?city)
      (in-city ?to ?city)
    )
    :effect (and (not (at ?truck ?from)) (at ?truck ?to))
  )

  (:action load-airplane
    :parameters (?pkg - package ?airplane - airplane ?loc - airport)
    :precondition (and (at ?pkg ?loc) (at ?airplane ?loc))
    :effect (and (not (at ?pkg ?loc)) (in ?pkg ?airplane))
  )

  (:action unload-airplane
    :parameters (?pkg - package ?airplane - airplane ?loc - airport)
    :precondition (and (in ?pkg ?airplane) (at ?airplane ?loc))
    :effect (and (not (in ?pkg ?airplane)) (at ?pkg ?loc))
  )

  (:action fly-airplane
    :parameters (?airplane - airplane ?from ?to - airport)
    :precondition (at ?airplane ?from)
    :effect (and (not (at ?airplane ?from)) (at ?airplane ?to))
  )
)
\end{lstlisting}

% 附录B 常用规划器安装与使用指南
\chapter{常用规划器安装与使用指南}
\label{app:planners}

本附录介绍几种常用规划器的安装和使用方法。

\section{Fast Downward}

\subsection{简介}

Fast Downward是最成功的现代规划系统之一,支持多种启发式和搜索算法。

\subsection{安装}

\textbf{Linux/macOS}:
\begin{lstlisting}[language=bash]
# 克隆仓库
git clone https://github.com/aibasel/downward.git
cd downward

# 编译
./build.py
\end{lstlisting}

\textbf{依赖}:
\begin{itemize}
    \item Python 3.6+
    \item C++20编译器(GCC 10+或Clang 12+)
    \item CMake 3.16+
\end{itemize}

\subsection{基本使用}

\begin{lstlisting}[language=bash]
# 使用A*搜索和LM-cut启发式
./fast-downward.py domain.pddl problem.pddl \
    --search "astar(lmcut())"

# 使用贪婪最佳优先搜索和FF启发式
./fast-downward.py domain.pddl problem.pddl \
    --search "eager_greedy([ff()])"

# LAMA配置
./fast-downward.py domain.pddl problem.pddl --alias lama
\end{lstlisting}

\subsection{常用配置}

\begin{table}[htbp]
    \centering
    \begin{tabular}{ll}
        \toprule
        配置 & 说明 \\
        \midrule
        \texttt{--alias lama} & LAMA配置,平衡质量和速度 \\
        \texttt{--alias lama-first} & 快速找到第一个解 \\
        \texttt{--alias seq-opt-lmcut} & 最优规划,使用LM-cut \\
        \texttt{--alias seq-sat-lama} & 满足性规划,LAMA风格 \\
        \bottomrule
    \end{tabular}
\end{table}

\section{PDDL4J}

\subsection{简介}

PDDL4J是一个Java库,提供PDDL解析和多种规划算法。

\subsection{安装}

\begin{lstlisting}[language=bash]
# 使用Maven
<dependency>
    <groupId>fr.uga</groupId>
    <artifactId>pddl4j</artifactId>
    <version>4.0.0</version>
</dependency>
\end{lstlisting}

\subsection{命令行使用}

\begin{lstlisting}[language=bash]
java -jar pddl4j.jar -o domain.pddl -f problem.pddl
\end{lstlisting}

\section{SHOP2}

\subsection{简介}

SHOP2是一个HTN规划系统,使用Lisp实现。

\subsection{安装}

\begin{lstlisting}[language=bash]
# 下载SHOP2
# 需要Common Lisp环境(如SBCL)

# 在SBCL中加载
(load "shop2.lisp")
\end{lstlisting}

\subsection{基本使用}

\begin{lstlisting}[language=lisp]
;; 定义领域
(defdomain logistics
  (:method (deliver ?pkg ?dest)
    ((at ?pkg ?loc))
    ((transport ?pkg ?loc ?dest)))
  ...
)

;; 求解问题
(find-plans 'logistics-problem :verbose t)
\end{lstlisting}

\section{Pyperplan}

\subsection{简介}

Pyperplan是一个教学用途的Python规划器,代码简洁易读。

\subsection{安装}

\begin{lstlisting}[language=bash]
pip install pyperplan
\end{lstlisting}

\subsection{使用}

\begin{lstlisting}[language=bash]
# 命令行
pyperplan domain.pddl problem.pddl

# Python API
from pyperplan import planner
solution = planner.search(domain, problem, "astar", "hadd")
\end{lstlisting}

\section{在线规划工具}

\subsection{Planning.Domains}

网址:\url{http://planning.domains/}

提供在线PDDL编辑器和多个规划器。

\subsection{Web Planner}

网址:\url{http://editor.planning.domains/}

功能:
\begin{itemize}
    \item 在线编辑PDDL
    \item 语法高亮和检查
    \item 多规划器支持
    \item 可视化执行
\end{itemize}

\section{调试技巧}

\subsection{常见错误}

\begin{enumerate}
    \item \textbf{语法错误}:使用在线编辑器检查
    \item \textbf{类型不匹配}:检查参数类型定义
    \item \textbf{无解}:简化问题或检查目标可达性
    \item \textbf{内存不足}:使用更高效的启发式
\end{enumerate}

\subsection{性能优化}

\begin{itemize}
    \item 选择合适的启发式函数
    \item 使用类型系统减少搜索空间
    \item 添加领域特定的约束
    \item 考虑问题分解
\end{itemize}

% 附录C 国际规划竞赛(IPC)介绍
\chapter{国际规划竞赛(IPC)介绍}
\label{app:ipc}

本附录介绍国际规划竞赛的背景和主要赛道。

\section{IPC历史}

\keyterm{国际规划竞赛}(International Planning Competition,IPC)是自动规划领域的重要评测平台,始于1998年。

\begin{table}[htbp]
    \centering
    \caption{IPC历史}
    \begin{tabular}{lll}
        \toprule
        年份 & 届次 & 主要创新 \\
        \midrule
        1998 & IPC-1 & 引入PDDL \\
        2000 & IPC-2 & ADL特性 \\
        2002 & IPC-3 & 时态规划赛道 \\
        2004 & IPC-4 & 数值规划 \\
        2006 & IPC-5 & 偏好和约束 \\
        2008 & IPC-6 & 不确定性赛道 \\
        2011 & IPC-7 & 学习赛道 \\
        2014 & IPC-8 & 确定性和学习 \\
        2018 & IPC-9 & 多个子竞赛 \\
        2023 & IPC-10 & HTN赛道、学习赛道 \\
        \bottomrule
    \end{tabular}
\end{table}

\section{竞赛赛道}

\subsection{经典赛道}

\textbf{特点}:
\begin{itemize}
    \item 完全可观测
    \item 确定性动作
    \item 有限状态空间
    \item 使用PDDL描述
\end{itemize}

\textbf{评价指标}:
\begin{itemize}
    \item 解的数量
    \item 解的质量(代价)
    \item 求解时间
    \item IPC得分
\end{itemize}

\subsection{最优规划赛道}

要求找到代价最小的解,规划器必须证明解的最优性。

\subsection{满足性规划赛道}

只要求找到可行解,不要求最优。

\subsection{HTN赛道}

使用HDDL语言,评估层次任务网络规划器。

\subsection{学习赛道}

规划器可以在相似问题上学习,然后在新问题上测试。

\section{基准领域}

\subsection{经典领域}

\begin{description}
    \item[Blocksworld] 积木世界,最经典的测试领域
    \item[Logistics] 物流运输,多城市、多车辆
    \item[Gripper] 机器人搬运球
    \item[Satellite] 卫星观测调度
    \item[Rovers] 火星探测器规划
    \item[Airport] 机场地面交通控制
    \item[Pipesworld] 管道输油调度
\end{description}

\subsection{时态领域}

\begin{description}
    \item[Zenotravel] 旅行规划,考虑燃料
    \item[Depots] 仓库物流
    \item[DriverLog] 司机和卡车调度
\end{description}

\section{参与IPC}

\subsection{获取基准问题}

\begin{lstlisting}[language=bash]
# 克隆IPC基准仓库
git clone https://github.com/aibasel/downward-benchmarks.git
\end{lstlisting}

\subsection{评测工具}

\begin{itemize}
    \item \textbf{Lab}:实验框架,用于运行和分析规划实验
    \item \textbf{VAL}:计划验证工具
    \item \textbf{IPC得分计算}:标准化的评分方法
\end{itemize}

\subsection{IPC得分}

对于问题$p$和规划器$P$,IPC得分定义为:
\begin{equation}
    \text{Score}(P, p) = \frac{C^*}{C(P, p)}
\end{equation}
其中$C^*$是已知最优解代价,$C(P, p)$是$P$找到的解的代价。

总得分是所有问题得分之和。

\section{优秀规划器}

历年IPC优秀规划器:
\begin{itemize}
    \item \textbf{Fast Downward}:多次获奖
    \item \textbf{LAMA}:IPC 2011冠军
    \item \textbf{Scorpion}:最优规划赛道优秀表现
    \item \textbf{BFWS}:最佳优先宽度搜索
    \item \textbf{Ragnarok}:IPC 2023经典赛道第一
\end{itemize}

\section{相关资源}

\begin{itemize}
    \item IPC官网:\url{https://www.icaps-conference.org/competitions/}
    \item IPC 2023:\url{https://ipc2023.github.io/}
    \item PDDL参考:\url{https://planning.wiki/}
    \item Downward基准:\url{https://github.com/aibasel/downward-benchmarks}
\end{itemize}

% 附录D 编程实验指导
\chapter{编程实验指导}
\label{app:experiments}

本附录提供配套的编程实验指导。

\section{实验1:状态空间搜索实现}

\subsection{实验目的}

\begin{enumerate}
    \item 理解状态空间搜索的基本原理
    \item 实现BFS、DFS和A*算法
    \item 比较不同搜索策略的性能
\end{enumerate}

\subsection{实验内容}

实现八数码问题(8-puzzle)求解器。

\textbf{任务1}:实现状态表示
\begin{lstlisting}[language=Python]
class PuzzleState:
    def __init__(self, board):
        """board是3x3的列表,0表示空格"""
        self.board = board

    def get_neighbors(self):
        """返回所有可能的后继状态"""
        pass

    def is_goal(self):
        """判断是否为目标状态"""
        pass
\end{lstlisting}

\textbf{任务2}:实现BFS
\begin{lstlisting}[language=Python]
def bfs(initial_state):
    """广度优先搜索"""
    frontier = deque([initial_state])
    explored = set()
    # 实现搜索逻辑
    pass
\end{lstlisting}

\textbf{任务3}:实现A*算法
\begin{lstlisting}[language=Python]
def astar(initial_state, heuristic):
    """A*搜索"""
    # 使用优先队列
    # 实现f = g + h的评估
    pass
\end{lstlisting}

\textbf{任务4}:实现启发式函数
\begin{itemize}
    \item 曼哈顿距离
    \item 错位数
    \item 线性冲突
\end{itemize}

\subsection{实验报告要求}

\begin{enumerate}
    \item 比较BFS和A*的扩展节点数
    \item 分析不同启发式的效果
    \item 讨论算法的时间和空间复杂度
\end{enumerate}

\section{实验2:PDDL建模练习}

\subsection{实验目的}

\begin{enumerate}
    \item 掌握PDDL语法
    \item 学会建模实际问题
    \item 使用规划器求解
\end{enumerate}

\subsection{实验内容}

为"机器人仓库"问题编写PDDL描述。

\textbf{问题描述}:
\begin{itemize}
    \item 仓库是$n \times m$的网格
    \item 有若干机器人和货物
    \item 机器人可以移动、拾取和放下货物
    \item 目标是将货物运送到指定位置
\end{itemize}

\textbf{任务1}:编写领域文件
\begin{itemize}
    \item 定义类型:location, robot, package
    \item 定义谓词:at, holding, empty, adjacent
    \item 定义动作:move, pick, place
\end{itemize}

\textbf{任务2}:编写问题文件
\begin{itemize}
    \item 定义具体的仓库布局
    \item 设置初始状态
    \item 指定目标条件
\end{itemize}

\textbf{任务3}:使用规划器求解
\begin{lstlisting}[language=bash]
# 使用Fast Downward
./fast-downward.py warehouse-domain.pddl warehouse-problem.pddl \
    --search "astar(ff())"
\end{lstlisting}

\section{实验3:HTN规划系统使用}

\subsection{实验目的}

\begin{enumerate}
    \item 理解HTN规划的原理
    \item 学会使用SHOP2或Pyhop
    \item 设计任务分解方法
\end{enumerate}

\subsection{实验内容}

使用Pyhop实现"旅行规划"系统。

\textbf{安装Pyhop}:
\begin{lstlisting}[language=bash]
git clone https://github.com/dananau/pyhop.git
\end{lstlisting}

\textbf{任务}:
\begin{enumerate}
    \item 定义原子任务:taxi, walk, fly
    \item 定义复合任务:travel
    \item 定义分解方法
    \item 测试不同场景
\end{enumerate}

\begin{lstlisting}[language=Python]
import pyhop

def travel_by_taxi(state, person, origin, dest):
    """打车方法"""
    if state.cash[person] >= taxi_fare(origin, dest):
        return [('taxi', person, origin, dest)]
    return False

pyhop.declare_methods('travel', travel_by_taxi, travel_by_foot)
\end{lstlisting}

\section{实验4:多智能体路径规划仿真}

\subsection{实验目的}

\begin{enumerate}
    \item 理解MAPF问题
    \item 实现基本的MAPF算法
    \item 可视化多智能体路径
\end{enumerate}

\subsection{实验内容}

实现优先级规划和CBS算法。

\textbf{任务1}:实现地图和智能体表示

\textbf{任务2}:实现优先级规划
\begin{lstlisting}[language=Python]
def priority_planning(agents, obstacles):
    """按优先级顺序规划"""
    paths = []
    for agent in sorted(agents, key=priority):
        path = astar_with_constraints(agent, paths)
        paths.append(path)
    return paths
\end{lstlisting}

\textbf{任务3}:实现冲突检测和CBS框架

\textbf{任务4}:可视化
\begin{itemize}
    \item 使用matplotlib或pygame
    \item 动态展示智能体移动
    \item 标注冲突位置
\end{itemize}

\section{实验5:VRP求解器开发}

\subsection{实验目的}

\begin{enumerate}
    \item 理解VRP问题
    \item 实现构造启发式
    \item 实现改进启发式
\end{enumerate}

\subsection{实验内容}

为小规模VRP问题开发求解器。

\textbf{任务1}:实现最近邻构造法
\begin{lstlisting}[language=Python]
def nearest_neighbor(depot, customers, vehicle_capacity):
    """最近邻启发式"""
    routes = []
    unvisited = set(customers)
    while unvisited:
        route = [depot]
        load = 0
        while unvisited:
            nearest = find_nearest(route[-1], unvisited)
            if load + demand[nearest] <= vehicle_capacity:
                route.append(nearest)
                load += demand[nearest]
                unvisited.remove(nearest)
            else:
                break
        route.append(depot)
        routes.append(route)
    return routes
\end{lstlisting}

\textbf{任务2}:实现2-opt改进

\textbf{任务3}:测试和比较
\begin{itemize}
    \item 在标准测试集上测试
    \item 比较不同方法的解质量
    \item 分析计算时间
\end{itemize}

\section{提交要求}

每个实验需要提交:
\begin{enumerate}
    \item 源代码(带注释)
    \item 实验报告(PDF格式)
    \item 运行结果截图或日志
\end{enumerate}

实验报告应包含:
\begin{itemize}
    \item 实验目的
    \item 算法描述
    \item 实现细节
    \item 结果分析
    \item 问题和思考
\end{itemize}


% ==================== 参考文献 ====================
\backmatter
\chapter*{参考文献}
\addcontentsline{toc}{chapter}{参考文献}

\begin{enumerate}[label={[\arabic*]}]
    \item Ghallab M, Nau D, Traverso P. Automated Planning: Theory and Practice[M]. Morgan Kaufmann, 2004.
    \item Russell S, Norvig P. Artificial Intelligence: A Modern Approach (4th Edition)[M]. Pearson, 2020.
    \item Geffner H, Bonet B. A Concise Introduction to Models and Methods for Automated Planning[M]. Morgan \& Claypool Publishers, 2013.
    \item 蔡自兴, 刘丽珏, 陈白帆, 等. 人工智能及其应用(第7版)[M]. 清华大学出版社, 2024.
    \item 王万良. 人工智能导论(第5版)[M]. 高等教育出版社, 2020.
    \item LaValle S M. Planning Algorithms[M]. Cambridge University Press, 2006.
    \item Wooldridge M. An Introduction to MultiAgent Systems (2nd Edition)[M]. John Wiley \& Sons, 2009.
    \item Sutton R S, Barto A G. Reinforcement Learning: An Introduction (2nd Edition)[M]. MIT Press, 2018.
    \item Kautz H, Selman B. Planning as satisfiability[C]. ECAI, 1992: 359-363.
    \item Hoffmann J, Nebel B. The FF planning system: Fast plan generation through heuristic search[J]. Journal of Artificial Intelligence Research, 2001, 14: 253-302.
    \item Helmert M. The Fast Downward Planning System[J]. Journal of Artificial Intelligence Research, 2006, 26: 191-246.
    \item Nau D, Au T C, Ilghami O, et al. SHOP2: An HTN planning system[J]. Journal of Artificial Intelligence Research, 2003, 20: 379-404.
    \item Kaelbling L P, Littman M L, Cassandra A R. Planning and acting in partially observable stochastic domains[J]. Artificial Intelligence, 1998, 101(1-2): 99-134.
    \item Stern R, Sturtevant N, Felner A, et al. Multi-agent pathfinding: Definitions, variants, and benchmarks[C]. SOCS, 2019.
    \item Garrett C R, Lozano-Pérez T, Kaelbling L P. PDDLStream: Integrating symbolic planners and blackbox samplers via optimistic adaptive planning[C]. ICAPS, 2020.
\end{enumerate}

% ==================== 术语表 ====================
\chapter*{术语表}
\addcontentsline{toc}{chapter}{术语表}

\begin{longtable}{p{4cm}p{4cm}p{6cm}}
    \toprule
    \textbf{中文术语} & \textbf{英文术语} & \textbf{说明} \\
    \midrule
    \endfirsthead
    \toprule
    \textbf{中文术语} & \textbf{英文术语} & \textbf{说明} \\
    \midrule
    \endhead
    \bottomrule
    \endfoot
    任务规划 & Task Planning & 制定行动序列以实现目标的过程 \\
    自动规划 & Automated Planning & 由计算机自动生成规划的技术 \\
    状态空间 & State Space & 所有可能状态的集合 \\
    动作 & Action & 改变系统状态的操作 \\
    前提条件 & Precondition & 动作执行前必须满足的条件 \\
    效果 & Effect & 动作执行后产生的状态变化 \\
    启发式 & Heuristic & 引导搜索的估计函数 \\
    STRIPS & STRIPS & Stanford Research Institute Problem Solver \\
    PDDL & PDDL & Planning Domain Definition Language \\
    HTN & HTN & Hierarchical Task Network \\
    时态规划 & Temporal Planning & 考虑时间约束的规划 \\
    调度 & Scheduling & 为任务分配时间和资源 \\
    多智能体系统 & Multi-Agent System & 多个智能体协作的系统 \\
    MDP & MDP & Markov Decision Process \\
    POMDP & POMDP & Partially Observable MDP \\
    运动规划 & Motion Planning & 规划物理运动轨迹 \\
    TAMP & TAMP & Task and Motion Planning \\
    VRP & VRP & Vehicle Routing Problem \\
\end{longtable}

\end{document}
