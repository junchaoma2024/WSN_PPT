% 附录D 编程实验指导
\chapter{编程实验指导}
\label{app:experiments}

本附录提供配套的编程实验指导。

\section{实验1:状态空间搜索实现}

\subsection{实验目的}

\begin{enumerate}
    \item 理解状态空间搜索的基本原理
    \item 实现BFS、DFS和A*算法
    \item 比较不同搜索策略的性能
\end{enumerate}

\subsection{实验内容}

实现八数码问题(8-puzzle)求解器。

\textbf{任务1}:实现状态表示
\begin{lstlisting}[language=Python]
class PuzzleState:
    def __init__(self, board):
        """board是3x3的列表,0表示空格"""
        self.board = board

    def get_neighbors(self):
        """返回所有可能的后继状态"""
        pass

    def is_goal(self):
        """判断是否为目标状态"""
        pass
\end{lstlisting}

\textbf{任务2}:实现BFS
\begin{lstlisting}[language=Python]
def bfs(initial_state):
    """广度优先搜索"""
    frontier = deque([initial_state])
    explored = set()
    # 实现搜索逻辑
    pass
\end{lstlisting}

\textbf{任务3}:实现A*算法
\begin{lstlisting}[language=Python]
def astar(initial_state, heuristic):
    """A*搜索"""
    # 使用优先队列
    # 实现f = g + h的评估
    pass
\end{lstlisting}

\textbf{任务4}:实现启发式函数
\begin{itemize}
    \item 曼哈顿距离
    \item 错位数
    \item 线性冲突
\end{itemize}

\subsection{实验报告要求}

\begin{enumerate}
    \item 比较BFS和A*的扩展节点数
    \item 分析不同启发式的效果
    \item 讨论算法的时间和空间复杂度
\end{enumerate}

\section{实验2:PDDL建模练习}

\subsection{实验目的}

\begin{enumerate}
    \item 掌握PDDL语法
    \item 学会建模实际问题
    \item 使用规划器求解
\end{enumerate}

\subsection{实验内容}

为"机器人仓库"问题编写PDDL描述。

\textbf{问题描述}:
\begin{itemize}
    \item 仓库是$n \times m$的网格
    \item 有若干机器人和货物
    \item 机器人可以移动、拾取和放下货物
    \item 目标是将货物运送到指定位置
\end{itemize}

\textbf{任务1}:编写领域文件
\begin{itemize}
    \item 定义类型:location, robot, package
    \item 定义谓词:at, holding, empty, adjacent
    \item 定义动作:move, pick, place
\end{itemize}

\textbf{任务2}:编写问题文件
\begin{itemize}
    \item 定义具体的仓库布局
    \item 设置初始状态
    \item 指定目标条件
\end{itemize}

\textbf{任务3}:使用规划器求解
\begin{lstlisting}[language=bash]
# 使用Fast Downward
./fast-downward.py warehouse-domain.pddl warehouse-problem.pddl \
    --search "astar(ff())"
\end{lstlisting}

\section{实验3:HTN规划系统使用}

\subsection{实验目的}

\begin{enumerate}
    \item 理解HTN规划的原理
    \item 学会使用SHOP2或Pyhop
    \item 设计任务分解方法
\end{enumerate}

\subsection{实验内容}

使用Pyhop实现"旅行规划"系统。

\textbf{安装Pyhop}:
\begin{lstlisting}[language=bash]
git clone https://github.com/dananau/pyhop.git
\end{lstlisting}

\textbf{任务}:
\begin{enumerate}
    \item 定义原子任务:taxi, walk, fly
    \item 定义复合任务:travel
    \item 定义分解方法
    \item 测试不同场景
\end{enumerate}

\begin{lstlisting}[language=Python]
import pyhop

def travel_by_taxi(state, person, origin, dest):
    """打车方法"""
    if state.cash[person] >= taxi_fare(origin, dest):
        return [('taxi', person, origin, dest)]
    return False

pyhop.declare_methods('travel', travel_by_taxi, travel_by_foot)
\end{lstlisting}

\section{实验4:多智能体路径规划仿真}

\subsection{实验目的}

\begin{enumerate}
    \item 理解MAPF问题
    \item 实现基本的MAPF算法
    \item 可视化多智能体路径
\end{enumerate}

\subsection{实验内容}

实现优先级规划和CBS算法。

\textbf{任务1}:实现地图和智能体表示

\textbf{任务2}:实现优先级规划
\begin{lstlisting}[language=Python]
def priority_planning(agents, obstacles):
    """按优先级顺序规划"""
    paths = []
    for agent in sorted(agents, key=priority):
        path = astar_with_constraints(agent, paths)
        paths.append(path)
    return paths
\end{lstlisting}

\textbf{任务3}:实现冲突检测和CBS框架

\textbf{任务4}:可视化
\begin{itemize}
    \item 使用matplotlib或pygame
    \item 动态展示智能体移动
    \item 标注冲突位置
\end{itemize}

\section{实验5:VRP求解器开发}

\subsection{实验目的}

\begin{enumerate}
    \item 理解VRP问题
    \item 实现构造启发式
    \item 实现改进启发式
\end{enumerate}

\subsection{实验内容}

为小规模VRP问题开发求解器。

\textbf{任务1}:实现最近邻构造法
\begin{lstlisting}[language=Python]
def nearest_neighbor(depot, customers, vehicle_capacity):
    """最近邻启发式"""
    routes = []
    unvisited = set(customers)
    while unvisited:
        route = [depot]
        load = 0
        while unvisited:
            nearest = find_nearest(route[-1], unvisited)
            if load + demand[nearest] <= vehicle_capacity:
                route.append(nearest)
                load += demand[nearest]
                unvisited.remove(nearest)
            else:
                break
        route.append(depot)
        routes.append(route)
    return routes
\end{lstlisting}

\textbf{任务2}:实现2-opt改进

\textbf{任务3}:测试和比较
\begin{itemize}
    \item 在标准测试集上测试
    \item 比较不同方法的解质量
    \item 分析计算时间
\end{itemize}

\section{提交要求}

每个实验需要提交:
\begin{enumerate}
    \item 源代码(带注释)
    \item 实验报告(PDF格式)
    \item 运行结果截图或日志
\end{enumerate}

实验报告应包含:
\begin{itemize}
    \item 实验目的
    \item 算法描述
    \item 实现细节
    \item 结果分析
    \item 问题和思考
\end{itemize}
