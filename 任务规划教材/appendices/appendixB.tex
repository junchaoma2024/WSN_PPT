% 附录B 常用规划器安装与使用指南
\chapter{常用规划器安装与使用指南}
\label{app:planners}

本附录介绍几种常用规划器的安装和使用方法。

\section{Fast Downward}

\subsection{简介}

Fast Downward是最成功的现代规划系统之一,支持多种启发式和搜索算法。

\subsection{安装}

\textbf{Linux/macOS}:
\begin{lstlisting}[language=bash]
# 克隆仓库
git clone https://github.com/aibasel/downward.git
cd downward

# 编译
./build.py
\end{lstlisting}

\textbf{依赖}:
\begin{itemize}
    \item Python 3.6+
    \item C++20编译器(GCC 10+或Clang 12+)
    \item CMake 3.16+
\end{itemize}

\subsection{基本使用}

\begin{lstlisting}[language=bash]
# 使用A*搜索和LM-cut启发式
./fast-downward.py domain.pddl problem.pddl \
    --search "astar(lmcut())"

# 使用贪婪最佳优先搜索和FF启发式
./fast-downward.py domain.pddl problem.pddl \
    --search "eager_greedy([ff()])"

# LAMA配置
./fast-downward.py domain.pddl problem.pddl --alias lama
\end{lstlisting}

\subsection{常用配置}

\begin{table}[htbp]
    \centering
    \begin{tabular}{ll}
        \toprule
        配置 & 说明 \\
        \midrule
        \texttt{--alias lama} & LAMA配置,平衡质量和速度 \\
        \texttt{--alias lama-first} & 快速找到第一个解 \\
        \texttt{--alias seq-opt-lmcut} & 最优规划,使用LM-cut \\
        \texttt{--alias seq-sat-lama} & 满足性规划,LAMA风格 \\
        \bottomrule
    \end{tabular}
\end{table}

\section{PDDL4J}

\subsection{简介}

PDDL4J是一个Java库,提供PDDL解析和多种规划算法。

\subsection{安装}

\begin{lstlisting}[language=bash]
# 使用Maven
<dependency>
    <groupId>fr.uga</groupId>
    <artifactId>pddl4j</artifactId>
    <version>4.0.0</version>
</dependency>
\end{lstlisting}

\subsection{命令行使用}

\begin{lstlisting}[language=bash]
java -jar pddl4j.jar -o domain.pddl -f problem.pddl
\end{lstlisting}

\section{SHOP2}

\subsection{简介}

SHOP2是一个HTN规划系统,使用Lisp实现。

\subsection{安装}

\begin{lstlisting}[language=bash]
# 下载SHOP2
# 需要Common Lisp环境(如SBCL)

# 在SBCL中加载
(load "shop2.lisp")
\end{lstlisting}

\subsection{基本使用}

\begin{lstlisting}[language=lisp]
;; 定义领域
(defdomain logistics
  (:method (deliver ?pkg ?dest)
    ((at ?pkg ?loc))
    ((transport ?pkg ?loc ?dest)))
  ...
)

;; 求解问题
(find-plans 'logistics-problem :verbose t)
\end{lstlisting}

\section{Pyperplan}

\subsection{简介}

Pyperplan是一个教学用途的Python规划器,代码简洁易读。

\subsection{安装}

\begin{lstlisting}[language=bash]
pip install pyperplan
\end{lstlisting}

\subsection{使用}

\begin{lstlisting}[language=bash]
# 命令行
pyperplan domain.pddl problem.pddl

# Python API
from pyperplan import planner
solution = planner.search(domain, problem, "astar", "hadd")
\end{lstlisting}

\section{在线规划工具}

\subsection{Planning.Domains}

网址:\url{http://planning.domains/}

提供在线PDDL编辑器和多个规划器。

\subsection{Web Planner}

网址:\url{http://editor.planning.domains/}

功能:
\begin{itemize}
    \item 在线编辑PDDL
    \item 语法高亮和检查
    \item 多规划器支持
    \item 可视化执行
\end{itemize}

\section{调试技巧}

\subsection{常见错误}

\begin{enumerate}
    \item \textbf{语法错误}:使用在线编辑器检查
    \item \textbf{类型不匹配}:检查参数类型定义
    \item \textbf{无解}:简化问题或检查目标可达性
    \item \textbf{内存不足}:使用更高效的启发式
\end{enumerate}

\subsection{性能优化}

\begin{itemize}
    \item 选择合适的启发式函数
    \item 使用类型系统减少搜索空间
    \item 添加领域特定的约束
    \item 考虑问题分解
\end{itemize}
