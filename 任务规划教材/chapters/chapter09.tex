% 第9章 多智能体规划
\chapter{多智能体规划}
\label{chap:multiagent}

本章讨论多个智能体协同完成任务的规划问题。

\section{多智能体系统基础}

\subsection{智能体架构}

\begin{definition}[智能体]
    \keyterm{智能体}(Agent)是一个能够感知环境并采取行动以实现目标的自主实体。智能体架构包括:
    \begin{itemize}
        \item 感知模块:获取环境信息
        \item 决策模块:选择行动
        \item 执行模块:执行选定的行动
        \item 通信模块:与其他智能体交互
    \end{itemize}
\end{definition}

\subsection{协调与通信}

多智能体系统的核心挑战是协调:
\begin{itemize}
    \item \textbf{集中式协调}:存在中央协调器
    \item \textbf{分布式协调}:智能体自主协商
    \item \textbf{隐式协调}:通过环境间接交互
\end{itemize}

\section{分布式规划}

\subsection{任务分配}

\begin{definition}[任务分配问题]
    给定$n$个智能体和$m$个任务,\keyterm{任务分配}问题是找到一个分配方案,使得总效用最大化或总代价最小化。
\end{definition}

常用算法:
\begin{itemize}
    \item 拍卖算法
    \item 匈牙利算法
    \item 合同网协议
\end{itemize}

\subsection{计划合并}

当每个智能体独立生成计划后,需要合并这些计划以避免冲突。

\section{多智能体路径规划(MAPF)}

\subsection{MAPF问题定义}

\begin{definition}[MAPF]
    \keyterm{多智能体路径规划}(Multi-Agent Path Finding)问题:给定$n$个智能体,每个智能体有起点和终点,在图上找到无冲突的路径使所有智能体到达各自目标。
\end{definition}

冲突类型:
\begin{itemize}
    \item \textbf{顶点冲突}:两个智能体同时占用同一顶点
    \item \textbf{边冲突}:两个智能体同时使用同一条边(相向移动)
\end{itemize}

\subsection{CBS算法}

\begin{definition}[冲突基搜索]
    \keyterm{CBS}(Conflict-Based Search)是一种两层搜索算法:
    \begin{itemize}
        \item 高层:搜索冲突树,识别和解决冲突
        \item 低层:为每个智能体规划满足约束的路径
    \end{itemize}
\end{definition}

\begin{algorithm}[H]
    \caption{CBS算法}
    \label{alg:cbs}
    \KwIn{MAPF问题}
    \KwOut{无冲突路径集合或失败}

    为每个智能体计算最短路径\;
    $\text{root} \gets$ 创建根节点,包含所有初始路径\;
    $\text{OPEN} \gets \{\text{root}\}$\;

    \While{$\text{OPEN}$ 非空}{
        $N \gets \text{OPEN}$ 中代价最小的节点\;
        验证 $N$ 中的路径是否有冲突\;
        \If{无冲突}{
            \KwRet{$N$ 中的路径}
        }
        选择第一个冲突 $(a_i, a_j, v, t)$\;
        \ForEach{智能体 $a \in \{a_i, a_j\}$}{
            创建子节点 $N'$,添加约束"$a$在时刻$t$不能在$v$"\;
            为 $a$ 重新规划满足新约束的路径\;
            \If{路径存在}{
                将 $N'$ 加入 $\text{OPEN}$\;
            }
        }
    }
    \KwRet{失败}
\end{algorithm}

\subsection{优先级规划}

\keyterm{优先级规划}按某种顺序依次为智能体规划路径,后规划的智能体需要避让先规划的智能体。

\begin{example}[仓储机器人协同调度]
\label{ex:warehouse}
    某自动化仓库有50台AGV(自动引导车)负责货物搬运。系统需要:
    \begin{itemize}
        \item 分配搬运任务给AGV
        \item 规划无冲突的移动路径
        \item 处理实时订单和突发情况
    \end{itemize}

    \textbf{挑战}:
    \begin{itemize}
        \item 狭窄通道可能造成死锁
        \item 需要实时重规划
        \item 充电站调度
    \end{itemize}

    \textbf{解决方案}:
    \begin{enumerate}
        \item 使用拍卖机制分配任务
        \item 使用CBS或其变体规划路径
        \item 实现滚动窗口重规划
    \end{enumerate}
\end{example}

\begin{example}[无人机蜂群编队规划]
\label{ex:swarm}
    20架无人机需要从分散位置汇聚形成特定编队,然后保持编队飞向目标。

    \textbf{阶段1:汇聚规划}
    \begin{itemize}
        \item 每架无人机分配编队中的目标位置
        \item 规划无碰撞的汇聚路径
        \item 同步到达时间
    \end{itemize}

    \textbf{阶段2:编队飞行}
    \begin{itemize}
        \item 领航无人机规划主航线
        \item 其他无人机保持相对位置
        \item 处理编队变换指令
    \end{itemize}

    \textbf{阶段3:障碍规避}
    \begin{itemize}
        \item 检测障碍物
        \item 编队收缩或分裂
        \item 通过后恢复编队
    \end{itemize}
\end{example}

\section{博弈论与规划}

\subsection{纳什均衡}

当智能体有各自的目标时,规划问题变成博弈。

\begin{definition}[纳什均衡]
    \keyterm{纳什均衡}是一个策略组合,其中没有智能体能通过单方面改变策略获得更高收益。
\end{definition}

\subsection{对抗规划}

在存在对手的情况下进行规划,需要考虑对手的可能反应。

\section*{本章小结}

本章介绍了多智能体规划的主要问题和方法。任务分配和计划合并是分布式规划的核心问题。MAPF是多智能体路径规划的标准形式化,CBS是求解MAPF的有效算法。博弈论为存在竞争关系的多智能体规划提供了理论基础。

\section*{习题}

\begin{enumerate}
    \item 设计一个3智能体的MAPF问题实例,手工用CBS算法求解。

    \item 比较集中式和分布式多智能体规划的优缺点。

    \item 对于示例\ref{ex:warehouse},讨论如何处理死锁情况。

    \item 实现一个简单的优先级规划算法,并分析其完备性。

    \item 将示例\ref{ex:swarm}建模为MAPF问题,讨论可能的简化假设。
\end{enumerate}
