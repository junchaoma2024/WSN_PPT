% 第15章 大语言模型与任务规划
\chapter{大语言模型与任务规划 \grad}
\label{chap:llm}

本章介绍大语言模型在任务规划中的应用,属于研究生拓展内容。

\section{LLM规划能力分析}

\subsection{LLM的推理能力}

大语言模型(如GPT-4、Claude等)展示了一定的推理和规划能力:
\begin{itemize}
    \item 常识推理
    \item 多步骤问题求解
    \item 代码生成和调试
    \item 任务分解
\end{itemize}

\subsection{规划基准测试(PlanBench)}

\begin{definition}[PlanBench]
    \keyterm{PlanBench}是评估LLM规划能力的基准测试套件,包含基于国际规划竞赛领域的问题。
\end{definition}

PlanBench的测试类别:
\begin{itemize}
    \item 计划生成(零样本/少样本)
    \item 计划验证
    \item 目标识别
    \item 代价优化
\end{itemize}

\subsection{LLM规划的局限性}

研究发现LLM在规划任务上存在系统性不足:
\begin{enumerate}
    \item \textbf{长时域规划困难}:步数增加时性能显著下降
    \item \textbf{计划有效性问题}:生成的计划可能违反约束
    \item \textbf{缺乏形式化保证}:无法确保解的正确性
    \item \textbf{结构化推理不足}:对复杂依赖关系的处理能力有限
\end{enumerate}

\section{LLM作为规划组件}

鉴于LLM作为独立规划器的局限性,研究者探索将LLM作为规划系统的组件。

\subsection{LLM+P:自然语言到PDDL}

\begin{definition}[LLM+P]
    \keyterm{LLM+P}框架使用LLM将自然语言问题描述翻译为PDDL格式,然后由经典规划器求解。
\end{definition}

工作流程:
\begin{enumerate}
    \item 用户用自然语言描述问题
    \item LLM生成PDDL领域和问题文件
    \item 经典规划器(如Fast Downward)求解
    \item LLM将计划翻译回自然语言
\end{enumerate}

\subsection{LLM生成启发式函数}

LLM可以为规划问题生成领域特定的启发式函数:
\begin{itemize}
    \item 输入:PDDL领域描述
    \item 输出:Python实现的启发式函数
    \item 优势:保持规划器的正确性保证
\end{itemize}

\subsection{LLM作为世界模型}

LLM可以作为世界模型,预测动作的效果:
\begin{itemize}
    \item 处理开放领域问题
    \item 利用常识知识
    \item 但可能产生幻觉
\end{itemize}

\section{具身智能规划}

\subsection{SayCan框架}

\begin{definition}[SayCan]
    \keyterm{SayCan}框架结合LLM的语言理解能力和机器人的可供性(affordance)函数:
    \begin{equation}
        \pi(a|s, i) \propto p_{\text{LLM}}(a|i) \cdot p_{\text{affordance}}(a|s)
    \end{equation}
    其中$i$是自然语言指令,$s$是当前状态。
\end{definition}

\subsection{SayPlan与3D场景图}

\begin{definition}[SayPlan]
    \keyterm{SayPlan}使用3D场景图(3DSG)来提供环境的结构化表示,使LLM能够在大规模环境中进行可扩展的任务规划。
\end{definition}

SayPlan的关键创新:
\begin{itemize}
    \item 场景图作为环境表示
    \item 多层次抽象
    \item 语义搜索缩小相关范围
\end{itemize}

\subsection{Inner Monologue}

\begin{definition}[Inner Monologue]
    \keyterm{Inner Monologue}通过环境反馈形成"内心独白",包括:
    \begin{itemize}
        \item 被动场景描述
        \item 主动场景描述
        \item 成功检测反馈
    \end{itemize}
\end{definition}

这种闭环反馈显著提升了机器人任务执行的成功率。

\section{多智能体LLM系统}

\subsection{LLM-MAS架构}

多智能体LLM系统的典型架构:
\begin{itemize}
    \item 每个智能体由一个LLM驱动
    \item 智能体具有不同的角色和专长
    \item 通过对话进行协调
    \item 共同完成复杂任务
\end{itemize}

\subsection{智能体协作框架}

代表性框架:
\begin{itemize}
    \item \textbf{AutoGen}:微软的多智能体对话框架
    \item \textbf{CAMEL}:角色扮演对话框架
    \item \textbf{MetaGPT}:软件开发多智能体系统
\end{itemize}

\section*{本章小结}

本章介绍了大语言模型与任务规划的结合。LLM展示了一定的规划能力,但存在系统性局限。将LLM作为规划系统的组件(如翻译器、启发式生成器)是更有效的方法。具身智能规划框架(SayCan、SayPlan)展示了LLM在机器人领域的应用。多智能体LLM系统是新兴的研究方向。

\section*{习题}

\begin{enumerate}
    \item 分析LLM在长时域规划中性能下降的原因。

    \item 用LLM将一个简单的规划问题翻译为PDDL,并验证翻译的正确性。

    \item 比较LLM+P和纯LLM规划的优缺点。

    \item 讨论SayCan中可供性函数的作用。

    \item 阅读一篇多智能体LLM系统的论文,总结其协调机制。
\end{enumerate}
