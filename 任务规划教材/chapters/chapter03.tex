% 第3章 经典规划理论
\chapter{经典规划理论}
\label{chap:classical}

本章介绍经典规划的核心理论,包括STRIPS表示语言、PDDL规划领域定义语言,以及基本的规划算法。

\section{STRIPS表示语言}

\keyterm{STRIPS}(Stanford Research Institute Problem Solver)是1969年由Fikes和Nilsson提出的规划系统,其表示方法至今仍是规划领域的基础。

\subsection{前提条件与效果}

STRIPS使用\keyterm{前提条件}(Preconditions)和\keyterm{效果}(Effects)来描述动作:

\begin{definition}[STRIPS动作]
    一个STRIPS动作$a$由三部分组成:
    \begin{itemize}
        \item \textbf{前提条件} $\text{Pre}(a)$:动作执行前必须满足的条件集合
        \item \textbf{添加效果} $\text{Add}(a)$:动作执行后变为真的事实集合
        \item \textbf{删除效果} $\text{Del}(a)$:动作执行后变为假的事实集合
    \end{itemize}
\end{definition}

\subsection{动作模式定义}

在实际应用中,我们通常定义\keyterm{动作模式}(Action Schema),即带参数的动作模板。

\begin{example}[积木世界问题]
\label{ex:blocksworld-strips}
    积木世界是规划领域的经典测试问题。假设有若干积木和一个机械手,目标是通过移动积木实现特定的堆叠配置。

    \textbf{谓词定义}:
    \begin{itemize}
        \item $\text{on}(x, y)$:积木$x$在积木$y$上面
        \item $\text{ontable}(x)$:积木$x$在桌面上
        \item $\text{clear}(x)$:积木$x$上面没有其他积木
        \item $\text{holding}(x)$:机械手正在抓取积木$x$
        \item $\text{arm-empty}$:机械手是空的
    \end{itemize}

    \textbf{动作模式}:

    \begin{center}
    \fbox{\parbox{0.8\textwidth}{
        \textbf{动作}:$\text{pick-up}(x)$ \\
        \textbf{前提}:$\text{ontable}(x) \land \text{clear}(x) \land \text{arm-empty}$ \\
        \textbf{效果}:$\text{holding}(x) \land \neg\text{ontable}(x) \land \neg\text{arm-empty}$
    }}
    \end{center}

    \begin{center}
    \fbox{\parbox{0.8\textwidth}{
        \textbf{动作}:$\text{put-down}(x)$ \\
        \textbf{前提}:$\text{holding}(x)$ \\
        \textbf{效果}:$\text{ontable}(x) \land \text{clear}(x) \land \text{arm-empty} \land \neg\text{holding}(x)$
    }}
    \end{center}

    \begin{center}
    \fbox{\parbox{0.8\textwidth}{
        \textbf{动作}:$\text{stack}(x, y)$ \\
        \textbf{前提}:$\text{holding}(x) \land \text{clear}(y)$ \\
        \textbf{效果}:$\text{on}(x,y) \land \text{clear}(x) \land \text{arm-empty} \land \neg\text{holding}(x) \land \neg\text{clear}(y)$
    }}
    \end{center}

    \begin{center}
    \fbox{\parbox{0.8\textwidth}{
        \textbf{动作}:$\text{unstack}(x, y)$ \\
        \textbf{前提}:$\text{on}(x,y) \land \text{clear}(x) \land \text{arm-empty}$ \\
        \textbf{效果}:$\text{holding}(x) \land \text{clear}(y) \land \neg\text{on}(x,y) \land \neg\text{arm-empty}$
    }}
    \end{center}
\end{example}

\section{PDDL规划领域定义语言}

\keyterm{PDDL}(Planning Domain Definition Language)是1998年为国际规划竞赛制定的标准规划语言。

\subsection{PDDL语法结构}

PDDL将规划问题分为两个文件:
\begin{itemize}
    \item \textbf{领域文件}(Domain File):定义谓词和动作模式
    \item \textbf{问题文件}(Problem File):定义具体实例的初始状态和目标
\end{itemize}

\subsection{领域文件与问题文件}

\begin{example}[物流运输领域建模]
\label{ex:logistics-pddl}
    考虑一个简化的物流问题:有若干城市、卡车和包裹,目标是将包裹运送到指定目的地。

    \textbf{领域文件}(logistics-domain.pddl):

\begin{lstlisting}[language=PDDL]
(define (domain logistics)
  (:requirements :strips :typing)

  (:types
    city location thing - object
    package vehicle - thing
    truck airplane - vehicle
  )

  (:predicates
    (in-city ?loc - location ?city - city)
    (at ?obj - thing ?loc - location)
    (in ?pkg - package ?veh - vehicle)
  )

  (:action load-truck
    :parameters (?pkg - package ?truck - truck ?loc - location)
    :precondition (and
      (at ?truck ?loc)
      (at ?pkg ?loc)
    )
    :effect (and
      (not (at ?pkg ?loc))
      (in ?pkg ?truck)
    )
  )

  (:action unload-truck
    :parameters (?pkg - package ?truck - truck ?loc - location)
    :precondition (and
      (at ?truck ?loc)
      (in ?pkg ?truck)
    )
    :effect (and
      (not (in ?pkg ?truck))
      (at ?pkg ?loc)
    )
  )

  (:action drive-truck
    :parameters (?truck - truck ?from ?to - location ?city - city)
    :precondition (and
      (at ?truck ?from)
      (in-city ?from ?city)
      (in-city ?to ?city)
    )
    :effect (and
      (not (at ?truck ?from))
      (at ?truck ?to)
    )
  )
)
\end{lstlisting}

    \textbf{问题文件}(logistics-problem.pddl):

\begin{lstlisting}[language=PDDL]
(define (problem logistics-1)
  (:domain logistics)

  (:objects
    city1 city2 - city
    loc1-1 loc1-2 loc2-1 loc2-2 - location
    truck1 truck2 - truck
    pkg1 pkg2 - package
  )

  (:init
    (in-city loc1-1 city1) (in-city loc1-2 city1)
    (in-city loc2-1 city2) (in-city loc2-2 city2)
    (at truck1 loc1-1)
    (at truck2 loc2-1)
    (at pkg1 loc1-2)
    (at pkg2 loc2-1)
  )

  (:goal (and
    (at pkg1 loc2-2)
    (at pkg2 loc1-1)
  ))
)
\end{lstlisting}
\end{example}

\subsection{PDDL扩展特性}

PDDL经历了多个版本的演进,引入了许多扩展特性:

\begin{table}[htbp]
    \centering
    \caption{PDDL版本演进}
    \label{tab:pddl-versions}
    \begin{tabular}{lll}
        \toprule
        版本 & 年份 & 主要特性 \\
        \midrule
        PDDL 1.2 & 1998 & STRIPS、类型、量词 \\
        PDDL 2.1 & 2002 & 数值fluents、持续性动作 \\
        PDDL 2.2 & 2004 & 派生谓词、时态初始文字 \\
        PDDL 3.0 & 2006 & 轨迹约束、偏好 \\
        PDDL 3.1 & 2008 & 对象fluents \\
        \bottomrule
    \end{tabular}
\end{table}

\section{前向状态空间规划}

\keyterm{前向规划}(Forward Planning)也称为\keyterm{前进规划}(Progression),从初始状态出发,逐步应用动作直到达到目标。

\subsection{基本算法}

前向规划的基本思想是将规划问题转化为图搜索问题,其中:
\begin{itemize}
    \item 节点是状态
    \item 边是动作
    \item 初始节点是$s_0$
    \item 目标节点是满足$g$的状态
\end{itemize}

\subsection{可达性分析}

\begin{definition}[可达状态]
    从初始状态$s_0$出发,通过执行一系列动作可以到达的状态称为\keyterm{可达状态}。所有可达状态的集合称为\keyterm{可达状态空间}。
\end{definition}

\section{后向状态空间规划}

\keyterm{后向规划}(Backward Planning)也称为\keyterm{回归规划}(Regression),从目标条件出发,逆向推导需要满足的前提条件。

\subsection{回归搜索}

\begin{definition}[回归]
    给定目标条件$g$和动作$a$,如果$a$可能实现$g$中的某些子目标,则$g$关于$a$的\keyterm{回归}定义为:
    \begin{equation}
        \text{regress}(g, a) = (g \setminus \text{Add}(a)) \cup \text{Pre}(a)
    \end{equation}
    前提是$\text{Add}(a) \cap g \neq \emptyset$且$\text{Del}(a) \cap g = \emptyset$。
\end{definition}

\subsection{相关性分析}

在后向搜索中,我们只考虑与目标\keyterm{相关}的动作,即那些能够实现至少一个目标子条件的动作。

\section{规划图与GraphPlan算法}

\keyterm{GraphPlan}是1995年由Blum和Furst提出的规划算法,它通过构建规划图来加速规划过程。

\subsection{规划图构造}

\begin{definition}[规划图]
    规划图是一个有向分层图,包含交替的\keyterm{命题层}和\keyterm{动作层}:
    \begin{itemize}
        \item 命题层包含在该时刻可能为真的命题
        \item 动作层包含在该时刻可能执行的动作
        \item 层之间通过前提条件和效果连接
    \end{itemize}
\end{definition}

\subsection{互斥关系}

\begin{definition}[互斥]
    两个动作(或两个命题)是\keyterm{互斥的}(Mutex),如果它们不能在同一规划步骤中同时出现。动作互斥的条件包括:
    \begin{enumerate}
        \item \textbf{不一致效果}:一个动作的效果否定另一个的效果
        \item \textbf{干扰}:一个动作的效果否定另一个的前提
        \item \textbf{竞争需求}:两个动作的前提条件互斥
    \end{enumerate}
\end{definition}

\subsection{解的提取}

在规划图构造完成后,使用后向搜索提取解。

\begin{example}[火箭运输问题]
\label{ex:rocket}
    考虑一个简化的火箭运输问题:
    \begin{itemize}
        \item 有两个地点:伦敦(London)和巴黎(Paris)
        \item 有一枚火箭R1,初始在伦敦
        \item 有两件货物A和B,初始都在伦敦
        \item 目标:将A和B都运送到巴黎
    \end{itemize}

    \textbf{动作}:
    \begin{itemize}
        \item $\text{load}(c, r, l)$:在地点$l$将货物$c$装载到火箭$r$
        \item $\text{unload}(c, r, l)$:在地点$l$将货物$c$从火箭$r$卸载
        \item $\text{move}(r, from, to)$:火箭$r$从$from$飞到$to$
    \end{itemize}

    \textbf{规划图}(部分):

    \begin{center}
    \begin{tikzpicture}[scale=0.9,
        prop/.style={rectangle, draw, minimum width=2cm, minimum height=0.6cm, font=\small},
        action/.style={ellipse, draw, minimum width=1.5cm, font=\small},
        mutex/.style={red, dashed, thick}
    ]
        % 第0层命题
        \node[prop] (at-a-l-0) at (0,3) {at(A,L)};
        \node[prop] (at-b-l-0) at (0,2) {at(B,L)};
        \node[prop] (at-r-l-0) at (0,1) {at(R,L)};
        \node[prop] (fuel-0) at (0,0) {fuel(R)};

        % 第1层动作
        \node[action] (load-a) at (3,2.5) {load(A)};
        \node[action] (load-b) at (3,1.5) {load(B)};
        \node[action] (noop1) at (3,0.5) {no-op};

        % 连接
        \draw[->] (at-a-l-0) -- (load-a);
        \draw[->] (at-r-l-0) -- (load-a);
        \draw[->] (at-b-l-0) -- (load-b);
        \draw[->] (at-r-l-0) -- (load-b);

        % 标签
        \node at (0,-1) {$P_0$};
        \node at (3,-1) {$A_1$};
    \end{tikzpicture}
    \end{center}

    \textbf{解}:
    \begin{enumerate}
        \item load(A, R1, London)
        \item load(B, R1, London)
        \item move(R1, London, Paris)
        \item unload(A, R1, Paris)
        \item unload(B, R1, Paris)
    \end{enumerate}
\end{example}

\section*{本章小结}

本章介绍了经典规划的核心理论。STRIPS是最早也是最基础的规划表示方法,通过前提条件和效果描述动作。PDDL是现代规划的标准语言,支持类型、数值、时态等扩展特性。前向规划和后向规划是两种基本的规划策略,GraphPlan通过构建规划图和分析互斥关系来加速规划过程。

\section*{习题}

\begin{enumerate}
    \item 用STRIPS格式定义一个"河内塔"(Tower of Hanoi)问题的动作。

    \item 编写PDDL领域文件,建模一个简单的"机器人取物"问题:机器人需要在房间之间移动并拾取物品。

    \item 对于示例\ref{ex:rocket}中的火箭运输问题,手工构造完整的规划图(直到所有目标第一次同时出现且不互斥)。

    \item 分析前向规划和后向规划各自的优缺点,讨论在什么情况下应该选择哪种方法。

    \item 实现一个简单的GraphPlan算法,求解积木世界问题。
\end{enumerate}
