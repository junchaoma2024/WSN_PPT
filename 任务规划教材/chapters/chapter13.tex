% 第13章 军事任务规划
\chapter{军事任务规划}
\label{chap:military}

本章介绍任务规划在军事领域的应用。

\section{军事规划概述}

\subsection{作战规划层次}

军事规划通常分为多个层次:
\begin{itemize}
    \item \textbf{战略层}:国家级战略目标和资源分配
    \item \textbf{战役层}:战区级作战计划
    \item \textbf{战术层}:具体作战行动规划
    \item \textbf{技术层}:武器系统和平台控制
\end{itemize}

\subsection{C4ISR系统}

\begin{definition}[C4ISR]
    \keyterm{C4ISR}代表指挥(Command)、控制(Control)、通信(Communication)、计算机(Computer)、情报(Intelligence)、监视(Surveillance)和侦察(Reconnaissance)。
\end{definition}

任务规划系统是C4ISR的核心组件之一。

\section{兵力部署规划}

\subsection{力量配置优化}

\begin{definition}[兵力部署问题]
    给定作战目标和可用兵力,确定各作战单元的部署位置和任务分配,以最大化作战效能。
\end{definition}

\subsection{后勤保障规划}

后勤保障规划包括:
\begin{itemize}
    \item 物资补给调度
    \item 运输路线规划
    \item 维修保障安排
    \item 医疗后送计划
\end{itemize}

\begin{example}[两栖登陆作战兵力部署]
\label{ex:amphibious}
    某两栖作战需要在指定海岸登陆并建立滩头阵地。

    \textbf{可用兵力}:
    \begin{itemize}
        \item 3个海军陆战队营
        \item 2个装甲连
        \item 空中支援力量
        \item 舰炮火力支援
    \end{itemize}

    \textbf{规划内容}:
    \begin{enumerate}
        \item 登陆点选择
        \item 各波次兵力编组
        \item 火力支援计划
        \item 后续梯队跟进时机
    \end{enumerate}

    \textbf{约束条件}:
    \begin{itemize}
        \item 滩头地形限制
        \item 敌方防御部署
        \item 潮汐和天气窗口
        \item 登陆舰艇容量
    \end{itemize}
\end{example}

\section{无人系统任务规划}

\subsection{无人机航迹规划}

\begin{definition}[航迹规划]
    为无人机确定从起点到目标再返回的飞行路径,需要考虑威胁规避、燃料约束、任务时限等因素。
\end{definition}

航迹规划方法:
\begin{itemize}
    \item Voronoi图方法
    \item A*及其变体
    \item RRT方法
    \item 势场法
\end{itemize}

\subsection{无人车任务分配}

无人地面车辆(UGV)任务规划包括:
\begin{itemize}
    \item 巡逻路线规划
    \item 侦察任务分配
    \item 物资运输调度
\end{itemize}

\subsection{有人-无人协同}

\keyterm{MUM-T}(Manned-Unmanned Teaming)是现代军事的重要作战概念:
\begin{itemize}
    \item 有人平台指挥控制
    \item 无人平台执行危险任务
    \item 协同态势感知
    \item 任务动态分配
\end{itemize}

\begin{example}[无人机侦察监视任务]
\label{ex:uav-reconnaissance}
    使用4架侦察无人机对某区域进行持续监视。

    \textbf{任务要求}:
    \begin{itemize}
        \item 监视区域面积:$100 \times 100$ km$^2$
        \item 监视周期:24小时
        \item 重点目标需要持续监视
        \item 发现可疑目标需要详细侦察
    \end{itemize}

    \textbf{规划内容}:
    \begin{enumerate}
        \item 巡逻航线设计
        \item 换班和加油调度
        \item 应急任务响应
        \item 通信中继安排
    \end{enumerate}

    \textbf{优化目标}:最大化区域覆盖率,最小化目标发现延迟。
\end{example}

\begin{example}[无人机蜂群协同打击]
\label{ex:swarm-attack}
    无人机蜂群对多个目标实施协同打击。

    \textbf{任务场景}:
    \begin{itemize}
        \item 30架察打一体无人机
        \item 10个已知目标
        \item 未知的防空威胁
        \item 通信可能被干扰
    \end{itemize}

    \textbf{规划层次}:
    \begin{enumerate}
        \item \textbf{任务分配}:确定每架无人机的目标
        \item \textbf{航迹规划}:规划进入和撤离航线
        \item \textbf{协同时序}:同步攻击时间
        \item \textbf{应急处置}:处理损失和新发现目标
    \end{enumerate}

    \textbf{分布式规划}:
    \begin{itemize}
        \item 中央规划初始方案
        \item 各无人机本地调整
        \item 基于共识的协调
        \item 自主重规划能力
    \end{itemize}
\end{example}

\section{电子战任务规划}

\begin{example}[电子干扰任务规划]
\label{ex:ew}
    电子战飞机需要对敌方雷达网络实施干扰。

    \textbf{情报输入}:
    \begin{itemize}
        \item 敌方雷达位置和类型
        \item 雷达工作参数
        \item 我方干扰设备能力
    \end{itemize}

    \textbf{规划内容}:
    \begin{enumerate}
        \item 干扰阵位选择
        \item 干扰样式和时序
        \item 功率和频率分配
        \item 协同配合计划
    \end{enumerate}

    \textbf{效能评估}:
    \begin{itemize}
        \item 雷达探测概率降低程度
        \item 掩护目标的暴露风险
        \item 干扰资源消耗
    \end{itemize}
\end{example}

\section*{本章小结}

本章介绍了任务规划在军事领域的应用。军事规划涵盖多个层次,从战略到技术层面。无人系统任务规划是当前的研究热点,包括航迹规划、任务分配和有人-无人协同。电子战规划是信息化战争的重要组成部分。

\section*{习题}

\begin{enumerate}
    \item 设计一个简单的无人机航迹规划算法,考虑禁飞区和威胁区。

    \item 对于示例\ref{ex:amphibious},分析规划中需要考虑的不确定性因素。

    \item 比较集中式和分布式无人机蜂群规划的优缺点。

    \item 用HTN方法建模一个军事任务分解过程。

    \item 讨论人工智能在军事任务规划中的伦理问题。
\end{enumerate}
