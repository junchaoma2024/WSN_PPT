% 第10章 运动规划基础
\chapter{运动规划基础}
\label{chap:motion}

本章介绍机器人运动规划的基本概念和方法。

\section{构型空间}

\subsection{机器人构型表示}

\begin{definition}[构型]
    机器人的\keyterm{构型}(Configuration)是完全描述机器人位置和姿态所需的最小参数集合。构型的维度称为\keyterm{自由度}(DOF)。
\end{definition}

例如:
\begin{itemize}
    \item 平面移动机器人:3 DOF $(x, y, \theta)$
    \item 六轴机械臂:6 DOF $(\theta_1, \theta_2, \ldots, \theta_6)$
    \item 人形机器人:通常 > 20 DOF
\end{itemize}

\subsection{障碍物映射}

\begin{definition}[构型空间障碍物]
    工作空间中的障碍物在构型空间中形成\keyterm{C空间障碍物}(C-obstacle),机器人的任何构型如果导致与障碍物碰撞,则属于C空间障碍物。
\end{definition}

\section{采样规划方法}

\subsection{PRM(概率路线图)}

\begin{definition}[PRM]
    \keyterm{概率路线图}(Probabilistic Roadmap)方法分两个阶段:
    \begin{enumerate}
        \item \textbf{学习阶段}:随机采样构型,连接近邻形成路线图
        \item \textbf{查询阶段}:将起点和终点连接到路线图,搜索路径
    \end{enumerate}
\end{definition}

\subsection{RRT(快速探索随机树)}

\begin{definition}[RRT]
    \keyterm{快速探索随机树}(Rapidly-exploring Random Tree)通过增量式构建搜索树来探索构型空间。
\end{definition}

\begin{algorithm}[H]
    \caption{基本RRT算法}
    \label{alg:rrt}
    \KwIn{起点 $q_{\text{init}}$,终点 $q_{\text{goal}}$}
    \KwOut{从起点到终点的路径}

    $T \gets$ 初始化树,包含 $q_{\text{init}}$\;
    \For{$k = 1$ \KwTo $K$}{
        $q_{\text{rand}} \gets$ 随机采样(偶尔采样 $q_{\text{goal}}$)\;
        $q_{\text{near}} \gets$ 树中距离 $q_{\text{rand}}$ 最近的节点\;
        $q_{\text{new}} \gets$ 从 $q_{\text{near}}$ 向 $q_{\text{rand}}$ 方向扩展一步\;
        \If{路径 $(q_{\text{near}}, q_{\text{new}})$ 无碰撞}{
            将 $q_{\text{new}}$ 加入树\;
            \If{$q_{\text{new}}$ 接近 $q_{\text{goal}}$}{
                \KwRet{构造路径}
            }
        }
    }
    \KwRet{失败}
\end{algorithm}

\subsection{RRT*与渐进最优性}

\begin{definition}[RRT*]
    \keyterm{RRT*}是RRT的改进版本,通过重新布线(rewiring)操作保证渐进最优性。
\end{definition}

\begin{example}[机械臂运动规划]
\label{ex:manipulator}
    一个6自由度工业机械臂需要将工件从A点移动到B点,工作空间中有障碍物。

    \textbf{构型空间}:$\mathbb{R}^6$,每个维度对应一个关节角度

    \textbf{约束}:
    \begin{itemize}
        \item 关节角度限制
        \item 避免碰撞(与障碍物、与自身)
        \item 奇异点避免
    \end{itemize}

    \textbf{使用RRT规划}:
    \begin{enumerate}
        \item 在6维构型空间中随机采样
        \item 使用正向运动学计算末端位置
        \item 使用碰撞检测验证路径
        \item 路径平滑处理
    \end{enumerate}
\end{example}

\section{任务与运动规划集成(TAMP)}

\subsection{符号-几何接口}

\keyterm{TAMP}(Task and Motion Planning)的挑战在于连接符号层面的任务规划和几何层面的运动规划。

关键问题:
\begin{itemize}
    \item 符号动作的几何可行性检验
    \item 几何约束的符号化表示
    \item 规划层次之间的信息传递
\end{itemize}

\subsection{PDDLStream}

\begin{definition}[PDDLStream]
    \keyterm{PDDLStream}扩展了PDDL,引入了流(Stream)来生成连续参数(如位姿、轨迹)。
\end{definition}

\begin{example}[服务机器人抓取规划]
\label{ex:manipulation}
    服务机器人需要从桌上抓取一个杯子并放入橱柜。

    \textbf{任务层面}(符号规划):
    \begin{enumerate}
        \item 移动到桌子附近
        \item 抓取杯子
        \item 移动到橱柜附近
        \item 打开橱柜门
        \item 放入杯子
        \item 关闭橱柜门
    \end{enumerate}

    \textbf{运动层面}(几何规划):
    \begin{itemize}
        \item 导航路径规划
        \item 抓取位姿采样
        \item 机械臂运动规划
        \item 门把手操作轨迹
    \end{itemize}

    \textbf{集成挑战}:
    \begin{itemize}
        \item 某些抓取位姿可能因碰撞不可行
        \item 放置位置影响后续操作
        \item 需要同时考虑导航和操作
    \end{itemize}
\end{example}

\section*{本章小结}

本章介绍了运动规划的基本概念和方法。构型空间是运动规划的数学基础。PRM和RRT是两类主要的采样规划方法。TAMP集成了任务规划和运动规划,是机器人规划的前沿方向。

\section*{习题}

\begin{enumerate}
    \item 对于一个在$10 \times 10$网格中的移动机器人,比较A*和RRT的性能。

    \item 证明RRT*具有渐进最优性。

    \item 为示例\ref{ex:manipulator}设计碰撞检测算法。

    \item 讨论TAMP中符号规划和运动规划如何交互。

    \item 实现基本的RRT算法,在2D空间中规划路径。
\end{enumerate}
