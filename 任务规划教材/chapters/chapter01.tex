% 第1章 绪论
\chapter{绪论}
\label{chap:introduction}

\section{任务规划的基本概念}

\keyterm{任务规划}(Task Planning)是人工智能领域的核心研究方向之一,它研究如何让智能系统自主地制定行动序列以实现特定目标。简单来说,任务规划回答的是"做什么"和"怎么做"的问题。

\begin{definition}[任务规划问题]
    一个任务规划问题可以形式化地定义为一个四元组 $\mathcal{P} = \langle S, A, I, G \rangle$,其中:
    \begin{itemize}
        \item $S$ 是状态空间,表示系统所有可能状态的集合;
        \item $A$ 是动作集合,表示智能体可执行的所有动作;
        \item $I \in S$ 是初始状态;
        \item $G \subseteq S$ 是目标状态集合。
    \end{itemize}
    规划的任务是找到一个动作序列 $\pi = \langle a_1, a_2, \ldots, a_n \rangle$,使得从初始状态 $I$ 出发,依次执行这些动作后,系统能够到达某个目标状态 $g \in G$。
\end{definition}

任务规划与人类的日常决策密切相关。当我们计划一次旅行、安排一天的工作、或者思考如何完成一个复杂项目时,我们都在进行某种形式的规划。人工智能领域的任务规划研究旨在将这种能力赋予计算机系统。

\section{任务规划的发展历程}

任务规划的研究始于20世纪60年代末,至今已经历了半个多世纪的发展。

\subsection{早期探索(1969--1980)}

1969年,斯坦福研究院(Stanford Research Institute)的研究人员开发了\keyterm{STRIPS}(Stanford Research Institute Problem Solver)系统,这是第一个具有重要影响力的自动规划系统。STRIPS引入了用前提条件和效果来描述动作的方法,这一表示方法至今仍是规划领域的基础。

同一时期,\keyterm{积木世界}(Blocks World)成为规划研究的标准测试问题。在这个问题中,机器人需要通过移动积木来实现特定的堆叠配置。

\subsection{经典规划时期(1980--1995)}

20世纪80年代至90年代中期,研究者们发展了多种规划方法:

\begin{itemize}
    \item \textbf{偏序规划}(Partial-Order Planning):允许规划中的动作保持部分有序,提高了规划的灵活性。
    \item \textbf{层次任务网络规划}(HTN Planning):通过任务分解的方式,将复杂任务分解为简单子任务。
    \item \textbf{规划图方法}:Blum和Furst于1995年提出的GraphPlan算法,通过构建规划图来加速规划过程。
\end{itemize}

\subsection{现代规划时期(1995--2015)}

这一时期的重要进展包括:

\begin{itemize}
    \item \textbf{PDDL语言}:1998年,为了统一规划问题的描述,研究者们制定了\keyterm{规划领域定义语言}(PDDL),并成为国际规划竞赛的标准语言。
    \item \textbf{启发式搜索规划}:FF规划器和Fast Downward系统的出现,使得规划器能够高效处理大规模问题。
    \item \textbf{SAT规划}:将规划问题转换为布尔可满足性问题,利用SAT求解器的强大能力。
\end{itemize}

\subsection{智能时代(2015--至今)}

近年来,深度学习和大语言模型的发展为任务规划带来了新的机遇:

\begin{itemize}
    \item \textbf{强化学习与规划}:AlphaGo等系统展示了学习与规划结合的强大能力。
    \item \textbf{神经符号规划}:将神经网络的学习能力与符号规划的推理能力相结合。
    \item \textbf{大语言模型规划}:探索利用GPT、Claude等大语言模型进行任务规划。
\end{itemize}

\section{任务规划的应用领域}

任务规划技术已广泛应用于多个领域,本节通过三个典型示例来说明。

\begin{example}[交通物流调度问题]
\label{ex:logistics}
    某快递公司需要为5辆配送车规划当日的配送路线。已知:
    \begin{itemize}
        \item 配送中心位于城市中心,坐标为$(0, 0)$;
        \item 共有50个配送点,每个点有一定的货物需求量;
        \item 每辆车的载重限制为2吨,行驶里程限制为200公里;
        \item 部分配送点有时间窗要求(如"上午9点前送达")。
    \end{itemize}

    这是一个典型的\keyterm{带时间窗的车辆路径问题}(VRPTW)。规划系统需要确定:
    \begin{enumerate}
        \item 每辆车访问哪些配送点(任务分配);
        \item 每辆车的访问顺序(路径规划);
        \item 何时到达每个配送点(调度)。
    \end{enumerate}

    目标是在满足所有约束的前提下,最小化总行驶距离或总配送时间。
\end{example}

\begin{example}[机器人任务执行]
\label{ex:robot}
    考虑一个家庭服务机器人,用户发出指令:"把客厅桌上的杯子放到厨房水槽里。"

    机器人需要规划以下动作序列:
    \begin{enumerate}
        \item 导航到客厅;
        \item 定位桌子和杯子;
        \item 移动到桌子前的合适位置;
        \item 伸出机械臂抓取杯子;
        \item 导航到厨房(同时保持杯子稳定);
        \item 定位水槽;
        \item 将杯子放入水槽。
    \end{enumerate}

    这涉及到\keyterm{任务与运动规划}(TAMP)的结合:高层的任务规划决定"做什么",底层的运动规划决定"怎么移动"。
\end{example}

\begin{example}[军事作战任务分配]
\label{ex:military}
    在一次无人机侦察任务中,指挥中心需要为10架无人机分配侦察任务。已知:
    \begin{itemize}
        \item 有20个需要侦察的目标区域;
        \item 每架无人机的续航时间为4小时;
        \item 不同目标区域的优先级不同;
        \item 存在禁飞区和威胁区域;
        \item 部分目标需要多架无人机协同侦察。
    \end{itemize}

    规划系统需要考虑:
    \begin{enumerate}
        \item 目标分配:哪些无人机负责哪些目标;
        \item 航迹规划:每架无人机的飞行路线;
        \item 时间协调:多机协同侦察的时间安排;
        \item 应急处理:遇到突发情况时的重规划。
    \end{enumerate}

    这是\keyterm{多智能体任务规划}与\keyterm{路径规划}的综合问题。
\end{example}

\section{任务规划与相关学科的关系}

任务规划是一个跨学科的研究领域,与多个学科密切相关:

\begin{itemize}
    \item \textbf{人工智能}:任务规划是AI的核心研究方向之一,与知识表示、推理、搜索等紧密相关。

    \item \textbf{运筹学}:许多规划问题可以建模为优化问题,运筹学提供了丰富的求解方法。

    \item \textbf{控制理论}:在机器人规划中,控制理论为运动执行提供了理论基础。

    \item \textbf{计算机科学}:算法设计、复杂性理论为规划提供了分析工具。

    \item \textbf{认知科学}:人类规划行为的研究为人工规划系统提供了启发。
\end{itemize}

图\ref{fig:planning-relations}展示了任务规划与相关学科的关系。

\begin{figure}[htbp]
    \centering
    \begin{tikzpicture}[
        box/.style={rectangle, draw, rounded corners, minimum width=2.5cm, minimum height=1cm, align=center},
        arrow/.style={->, >=stealth, thick}
    ]
        \node[box, fill=blue!20] (planning) at (0,0) {任务规划};
        \node[box, fill=green!20] (ai) at (-3,2) {人工智能};
        \node[box, fill=green!20] (or) at (3,2) {运筹学};
        \node[box, fill=yellow!20] (control) at (-3,-2) {控制理论};
        \node[box, fill=yellow!20] (cs) at (3,-2) {计算机科学};
        \node[box, fill=orange!20] (cognitive) at (0,3) {认知科学};

        \draw[arrow, <->] (planning) -- (ai);
        \draw[arrow, <->] (planning) -- (or);
        \draw[arrow, <->] (planning) -- (control);
        \draw[arrow, <->] (planning) -- (cs);
        \draw[arrow, <->] (planning) -- (cognitive);
    \end{tikzpicture}
    \caption{任务规划与相关学科的关系}
    \label{fig:planning-relations}
\end{figure}

\section{本书结构与学习指南}

本书共分四篇十六章,内容安排如下:

\subsection{内容结构}

\begin{description}
    \item[第一篇:基础篇(第1--6章)] 建立任务规划的理论基础,包括状态空间搜索、经典规划理论、启发式方法、时态规划和层次任务网络规划。这部分内容是后续学习的基础,建议按顺序学习。

    \item[第二篇:方法篇(第7--10章)] 介绍进阶的规划方法,包括约束满足、不确定性规划、多智能体规划和运动规划。这些章节相对独立,可以根据兴趣选择性学习。

    \item[第三篇:应用篇(第11--13章)] 通过三个应用领域展示任务规划技术的实际应用。建议在学完基础篇后,选择感兴趣的应用领域深入学习。

    \item[第四篇:前沿篇(第14--16章)] 介绍最新的研究进展,适合研究生深入学习。这些章节标有\grad 符号。
\end{description}

\subsection{学习建议}

\begin{enumerate}
    \item \textbf{理论与实践结合}:在学习理论知识的同时,动手完成附录D中的编程实验。

    \item \textbf{循序渐进}:先掌握基础篇的内容,再学习进阶内容。

    \item \textbf{关注示例}:每章的示例都经过精心设计,建议仔细研读。

    \item \textbf{完成习题}:每章末尾的习题有助于检验学习效果。

    \item \textbf{查阅参考文献}:对感兴趣的主题,可以通过参考文献进一步深入学习。
\end{enumerate}

\section*{本章小结}

本章介绍了任务规划的基本概念、发展历程、应用领域以及与相关学科的关系。任务规划研究如何让智能系统自主制定行动序列以实现目标,是人工智能领域的核心研究方向。从1969年的STRIPS系统到现代的大语言模型规划,任务规划技术不断发展,应用领域日益广泛。

\section*{习题}

\begin{enumerate}
    \item 试用自己的语言解释什么是任务规划,并举出日常生活中的两个任务规划例子。

    \item 对于示例\ref{ex:logistics}中的快递配送问题,讨论以下问题:
    \begin{enumerate}
        \item 如果取消时间窗约束,问题会变得更简单还是更复杂?为什么?
        \item 如果允许配送点之间的货物中转,会对问题产生什么影响?
    \end{enumerate}

    \item 查阅资料,了解STRIPS系统的历史,回答:
    \begin{enumerate}
        \item STRIPS的名称是什么的缩写?
        \item STRIPS最初用于什么应用场景?
        \item STRIPS表示方法的核心思想是什么?
    \end{enumerate}

    \item 思考题:你认为任务规划与机器学习有什么区别和联系?

    \item 编程练习:用你熟悉的编程语言,实现一个简单的"积木世界"问题的状态表示,包括:
    \begin{enumerate}
        \item 定义状态的数据结构;
        \item 实现检查某个积木是否在另一个积木上方的函数;
        \item 实现打印当前状态的函数。
    \end{enumerate}
\end{enumerate}
