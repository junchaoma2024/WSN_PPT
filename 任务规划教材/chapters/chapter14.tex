% 第14章 基于学习的规划方法
\chapter{基于学习的规划方法 \grad}
\label{chap:learning}

本章介绍机器学习与规划的结合,属于研究生拓展内容。

\section{强化学习与规划}

\subsection{模型规划RL}

\begin{definition}[基于模型的强化学习]
    \keyterm{基于模型的强化学习}(Model-Based RL)通过学习环境动力学模型,然后在模型上进行规划来做出决策。
\end{definition}

\textbf{基本框架}:
\begin{enumerate}
    \item 与环境交互收集数据
    \item 学习转移模型 $\hat{T}(s'|s,a)$ 和奖励模型 $\hat{R}(s,a)$
    \item 在学习的模型上规划
    \item 执行规划得到的动作
\end{enumerate}

\subsection{AlphaGo与蒙特卡洛树搜索}

AlphaGo结合了深度学习和蒙特卡洛树搜索(MCTS):
\begin{itemize}
    \item \textbf{策略网络}:估计动作概率分布
    \item \textbf{价值网络}:估计局面价值
    \item \textbf{MCTS}:使用网络引导搜索
\end{itemize}

\begin{algorithm}[H]
    \caption{MCTS基本算法}
    \label{alg:mcts}
    \KwIn{根节点状态 $s$}
    \KwOut{最佳动作}

    \For{$i = 1$ \KwTo $N$}{
        $\text{node} \gets \text{root}$\;
        \tcp{选择阶段}
        \While{$\text{node}$ 完全展开且非叶节点}{
            $\text{node} \gets$ UCB1选择的子节点\;
        }
        \tcp{展开阶段}
        \If{$\text{node}$ 未完全展开}{
            $\text{node} \gets$ 展开一个未尝试的动作\;
        }
        \tcp{模拟阶段}
        $v \gets$ 随机模拟到终局\;
        \tcp{回传阶段}
        回传 $v$ 更新路径上所有节点的统计\;
    }
    \KwRet{访问次数最多的动作}
\end{algorithm}

\section{模仿学习}

\subsection{行为克隆}

\begin{definition}[行为克隆]
    \keyterm{行为克隆}(Behavior Cloning)直接从专家演示中学习策略,将状态-动作对作为监督学习问题。
\end{definition}

问题:分布漂移(Distribution Shift)——训练时的状态分布与执行时不同。

\subsection{逆强化学习}

\begin{definition}[逆强化学习]
    \keyterm{逆强化学习}(Inverse Reinforcement Learning,IRL)从专家演示中推断奖励函数,然后用标准RL方法学习策略。
\end{definition}

\section{神经网络规划器}

\subsection{神经符号规划}

\keyterm{神经符号规划}结合神经网络的学习能力和符号规划的推理能力:
\begin{itemize}
    \item 神经网络处理感知和特征提取
    \item 符号系统处理推理和规划
    \item 接口层连接两个组件
\end{itemize}

\subsection{图神经网络在规划中的应用}

\keyterm{图神经网络}(GNN)可以处理规划问题的结构化表示:
\begin{itemize}
    \item 状态表示为图结构
    \item GNN学习状态特征
    \item 用于启发式估计或策略学习
\end{itemize}

\section{迁移学习与元学习}

\subsection{领域迁移}

在源领域学习的规划知识可以迁移到目标领域:
\begin{itemize}
    \item 启发式函数迁移
    \item 动作模式迁移
    \item 领域知识迁移
\end{itemize}

\subsection{少样本规划}

\keyterm{元学习}使规划器能够快速适应新领域:
\begin{itemize}
    \item 学习如何学习规划
    \item 从少量样本中提取领域特征
    \item 快速调整规划策略
\end{itemize}

\section*{本章小结}

本章介绍了基于学习的规划方法。强化学习与规划的结合是重要的研究方向,AlphaGo是成功案例。模仿学习从专家演示中学习。神经符号规划结合了深度学习和符号推理的优势。迁移学习和元学习帮助规划器泛化到新问题。

\section*{习题}

\begin{enumerate}
    \item 解释为什么AlphaGo需要结合策略网络和价值网络。

    \item 比较行为克隆和逆强化学习的优缺点。

    \item 设计一个用GNN表示规划问题状态的方案。

    \item 讨论神经网络规划器的可解释性问题。

    \item 阅读一篇神经符号规划的最新论文,总结其主要贡献。
\end{enumerate}
