% 第8章 不确定性规划
\chapter{不确定性规划}
\label{chap:uncertainty}

本章讨论在不确定环境下的规划问题。

\section{非确定性规划}

\subsection{条件规划}

\begin{definition}[条件规划]
    \keyterm{条件规划}(Contingent Planning)生成的计划包含条件分支,根据执行过程中观测到的信息选择不同的动作。
\end{definition}

条件计划可以表示为树结构:
\begin{itemize}
    \item 内部节点是观测或条件判断
    \item 叶节点是动作序列
\end{itemize}

\subsection{一致性规划}

\begin{definition}[一致性规划]
    \keyterm{一致性规划}(Conformant Planning)在没有任何观测能力的情况下进行规划。计划必须在所有可能的初始状态下都能达到目标。
\end{definition}

\section{部分可观测规划}

\subsection{信念状态}

\begin{definition}[信念状态]
    当智能体不能完全观测环境状态时,它维护一个\keyterm{信念状态}(Belief State),表示对当前可能状态的概率分布或集合。
\end{definition}

\subsection{信念空间搜索}

在信念空间中搜索,节点是信念状态,边是动作和观测的组合。

\begin{example}[传感器受限的机器人导航]
\label{ex:robot-navigation}
    一个机器人在$4 \times 4$网格中导航,但传感器只能检测相邻格子是否有障碍物,不能确定自己的精确位置。

    \textbf{信念状态}:可能位置的集合

    \textbf{动作}:上、下、左、右移动

    \textbf{观测}:相邻四个方向的障碍物情况

    机器人需要在不确定自己位置的情况下,规划到达目标区域的策略。通过执行动作和获取观测,逐渐缩小信念状态,最终确定位置并到达目标。
\end{example}

\section{概率规划}

\subsection{马尔可夫决策过程(MDP)}

\begin{definition}[MDP]
    \keyterm{马尔可夫决策过程}定义为五元组 $\langle S, A, T, R, \gamma \rangle$:
    \begin{itemize}
        \item $S$:状态空间
        \item $A$:动作空间
        \item $T: S \times A \times S \to [0,1]$:转移概率函数
        \item $R: S \times A \to \mathbb{R}$:奖励函数
        \item $\gamma \in [0,1)$:折扣因子
    \end{itemize}
\end{definition}

\subsection{部分可观测MDP(POMDP)}

\begin{definition}[POMDP]
    \keyterm{POMDP}在MDP基础上增加:
    \begin{itemize}
        \item $\Omega$:观测空间
        \item $O: S \times A \times \Omega \to [0,1]$:观测概率函数
    \end{itemize}
\end{definition}

\subsection{值迭代与策略迭代}

\begin{algorithm}[H]
    \caption{值迭代算法}
    \label{alg:value-iteration}
    \KwIn{MDP $\langle S, A, T, R, \gamma \rangle$}
    \KwOut{最优值函数 $V^*$}

    初始化 $V(s) = 0$ 对所有 $s \in S$\;
    \Repeat{收敛}{
        \ForEach{$s \in S$}{
            $V(s) \gets \max_{a \in A} \left[ R(s,a) + \gamma \sum_{s' \in S} T(s,a,s') V(s') \right]$\;
        }
    }
    \KwRet{$V$}
\end{algorithm}

\begin{example}[无人机任务规划中的不确定性处理]
\label{ex:uav-uncertainty}
    无人机侦察任务中存在多种不确定性:
    \begin{itemize}
        \item \textbf{天气不确定性}:风速、能见度可能变化
        \item \textbf{目标不确定性}:目标可能移动或隐藏
        \item \textbf{传感器不确定性}:侦察可能失败
        \item \textbf{威胁不确定性}:敌方防空系统位置不确定
    \end{itemize}

    使用POMDP建模:
    \begin{itemize}
        \item 状态包括:无人机位置、剩余燃料、目标状态、威胁状态
        \item 动作包括:飞向某区域、执行侦察、返航
        \item 观测包括:是否发现目标、是否被探测
        \item 奖励设计:发现目标获得正奖励,被击落获得大负奖励
    \end{itemize}
\end{example}

\section*{本章小结}

本章介绍了不确定性规划的主要方法。条件规划和一致性规划处理非确定性。信念状态用于表示部分可观测环境。MDP和POMDP提供了处理概率不确定性的数学框架。

\section*{习题}

\begin{enumerate}
    \item 设计一个需要条件规划的问题场景,画出条件计划树。

    \item 对于示例\ref{ex:robot-navigation},设计一个缩小信念状态的动作序列。

    \item 用MDP建模一个简单的路径规划问题,其中动作有时会失败。

    \item 比较MDP和POMDP的计算复杂性差异。

    \item 实现值迭代算法,求解一个$5 \times 5$网格世界MDP。
\end{enumerate}
