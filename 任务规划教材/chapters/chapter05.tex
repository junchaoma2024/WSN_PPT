% 第5章 时态规划与调度
\chapter{时态规划与调度}
\label{chap:temporal}

本章介绍考虑时间约束的规划问题,包括时态规划和调度问题。

\section{时态逻辑基础}

\subsection{时态算子}

时态逻辑引入了描述时间的算子:
\begin{itemize}
    \item $\square \phi$(总是):$\phi$在所有未来时刻都为真
    \item $\Diamond \phi$(最终):$\phi$在某个未来时刻为真
    \item $\bigcirc \phi$(下一刻):$\phi$在下一时刻为真
    \item $\phi \mathcal{U} \psi$(直到):$\phi$保持为真直到$\psi$变为真
\end{itemize}

\subsection{时态约束}

\begin{definition}[时态约束]
    时态约束规定动作之间的时间关系,常见形式包括:
    \begin{itemize}
        \item 先后约束:动作$a$必须在动作$b$之前完成
        \item 同时约束:动作$a$和$b$必须同时执行
        \item 时间窗约束:动作必须在指定时间范围内执行
        \item 持续时间约束:动作执行需要一定的时间
    \end{itemize}
\end{definition}

\section{PDDL 2.1时态扩展}

\subsection{持续性动作}

\begin{definition}[持续性动作]
    \keyterm{持续性动作}(Durative Action)有明确的开始时刻、结束时刻和持续时间。其定义包括:
    \begin{itemize}
        \item \texttt{:duration}:动作的持续时间
        \item \texttt{:condition}:开始条件、结束条件和持续条件
        \item \texttt{:effect}:开始效果和结束效果
    \end{itemize}
\end{definition}

\begin{lstlisting}[language=PDDL]
(:durative-action drive
  :parameters (?truck - truck ?from ?to - location)
  :duration (= ?duration (travel-time ?from ?to))
  :condition (and
    (at start (at ?truck ?from))
    (over all (road ?from ?to))
  )
  :effect (and
    (at start (not (at ?truck ?from)))
    (at end (at ?truck ?to))
  )
)
\end{lstlisting}

\subsection{并发执行}

时态规划允许多个动作并发执行,但需要满足以下条件:
\begin{enumerate}
    \item 不存在资源冲突
    \item 不违反因果依赖
    \item 满足时间约束
\end{enumerate}

\section{时态规划算法}

\subsection{SAPA规划器}

SAPA是一个前向搜索时态规划器,使用时态启发式引导搜索。

\subsection{TFD规划器}

TFD(Temporal Fast Downward)扩展了Fast Downward以支持时态规划。

\section{调度问题}

\subsection{作业车间调度}

\begin{definition}[作业车间调度]
    \keyterm{作业车间调度}(Job Shop Scheduling)问题定义如下:
    \begin{itemize}
        \item 有$n$个作业(jobs)和$m$台机器(machines)
        \item 每个作业包含若干工序,工序之间有先后约束
        \item 每个工序需要在特定机器上处理特定时间
        \item 目标是最小化总完工时间(makespan)
    \end{itemize}
\end{definition}

\subsection{资源约束项目调度}

\begin{definition}[RCPSP]
    \keyterm{资源约束项目调度问题}(Resource-Constrained Project Scheduling Problem,RCPSP)是一类考虑有限资源约束的调度问题。
\end{definition}

\begin{example}[生产线调度优化]
\label{ex:production}
    某制造企业有3条生产线和10个生产订单。每个订单需要经过多道工序,部分工序可以在不同生产线上执行。约束条件包括:
    \begin{itemize}
        \item 每条生产线同一时刻只能处理一个工序
        \item 同一订单的工序之间有先后关系
        \item 部分订单有交货期限制
    \end{itemize}

    目标是最小化总延迟时间或最大化产能利用率。
\end{example}

\begin{example}[航班调度问题]
\label{ex:flight}
    机场需要为当日100架次航班分配停机位和登机口。约束条件包括:
    \begin{itemize}
        \item 每个停机位同一时刻只能停放一架飞机
        \item 登机口与停机位之间需要满足一定的对应关系
        \item 不同机型对停机位有不同要求
        \item 需要预留足够的周转时间
    \end{itemize}

    目标是最小化乘客步行距离或最大化机场运营效率。
\end{example}

\section*{本章小结}

本章介绍了时态规划与调度。时态规划扩展了经典规划以处理时间约束和持续性动作。PDDL 2.1引入了持续性动作和数值fluent。调度问题是规划的重要应用领域,包括作业车间调度和资源约束项目调度。

\section*{习题}

\begin{enumerate}
    \item 用PDDL 2.1编写一个简单的时态规划领域,包含至少两个持续性动作。

    \item 对于示例\ref{ex:production}中的生产调度问题,画出一个可行的甘特图。

    \item 比较规划问题和调度问题的异同点。

    \item 分析时态规划相比经典规划的计算复杂性。

    \item 设计一个启发式函数,用于评估时态规划中的部分计划质量。
\end{enumerate}
