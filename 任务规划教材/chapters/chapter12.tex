% 第12章 通信与编码传输规划
\chapter{通信与编码传输规划}
\label{chap:communication}

本章介绍任务规划在通信网络领域的应用。

\section{通信网络资源调度}

\subsection{信道分配}

\begin{definition}[信道分配问题]
    在无线通信网络中,\keyterm{信道分配}问题是将有限的频谱资源分配给各用户或基站,以最大化系统容量并最小化干扰。
\end{definition}

信道分配可以建模为图着色问题:
\begin{itemize}
    \item 节点表示通信链路
    \item 边表示干扰关系
    \item 颜色表示信道
\end{itemize}

\subsection{功率控制}

功率控制问题是确定每个发射机的发射功率,以满足通信质量要求并最小化总功耗或干扰。

\section{无线传感器网络任务规划}

\subsection{数据采集调度}

\begin{definition}[数据采集调度]
    在无线传感器网络中,\keyterm{数据采集调度}确定每个传感器节点何时进行数据采集、处理和传输,以平衡数据时效性和能量消耗。
\end{definition}

\subsection{能量感知路由}

由于传感器节点通常由电池供电,路由协议需要考虑能量因素:
\begin{itemize}
    \item 最小能量路由
    \item 能量均衡路由
    \item 基于剩余能量的路由
\end{itemize}

\begin{example}[物联网数据汇聚规划]
\label{ex:iot-aggregation}
    某智慧农业系统部署了1000个土壤传感器,需要定期向网关汇报数据。

    \textbf{约束条件}:
    \begin{itemize}
        \item 每个传感器电池容量有限
        \item 传输距离影响能耗
        \item 数据需要在规定时间内到达
        \item 部分节点可以进行数据聚合
    \end{itemize}

    \textbf{规划内容}:
    \begin{enumerate}
        \item 确定数据汇聚树结构
        \item 调度每个节点的醒睡周期
        \item 确定数据聚合点
        \item 优化传输时隙分配
    \end{enumerate}

    \textbf{优化目标}:最大化网络寿命,保证数据时效性。
\end{example}

\section{卫星通信任务规划}

\subsection{卫星过境调度}

\begin{definition}[卫星过境调度]
    地面站需要与多颗卫星通信,但视窗时间有限。\keyterm{卫星过境调度}问题是确定地面站与哪颗卫星在何时通信。
\end{definition}

\subsection{星地链路规划}

\begin{example}[遥感卫星成像任务规划]
\label{ex:satellite-imaging}
    某遥感卫星星座需要完成100个地面目标的成像任务。

    \textbf{约束条件}:
    \begin{itemize}
        \item 每颗卫星的轨道决定了可观测时间窗
        \item 卫星姿态调整需要时间
        \item 星上存储容量有限
        \item 数据需要在一定时间内下传
    \end{itemize}

    \textbf{规划决策}:
    \begin{enumerate}
        \item 任务分配:哪颗卫星观测哪个目标
        \item 观测调度:何时进行观测
        \item 数据下传:何时通过哪个地面站下传
    \end{enumerate}

    \textbf{优化目标}:最大化完成的任务数,优先满足高优先级任务。

    \textbf{规划算法}:
    \begin{itemize}
        \item 基于约束满足的方法
        \item 启发式搜索
        \item 遗传算法
    \end{itemize}
\end{example}

\section{数据中心任务调度}

\begin{example}[云计算资源编排]
\label{ex:cloud-orchestration}
    云计算平台需要调度大量虚拟机和容器。

    \textbf{任务类型}:
    \begin{itemize}
        \item 批处理任务:可延迟执行
        \item 交互式任务:需要快速响应
        \item 流式任务:持续执行
    \end{itemize}

    \textbf{资源约束}:
    \begin{itemize}
        \item CPU、内存、存储、网络带宽
        \item 物理机的容量限制
        \item 数据局部性要求
        \item 服务质量(QoS)保证
    \end{itemize}

    \textbf{调度目标}:
    \begin{itemize}
        \item 最小化任务完成时间
        \item 最大化资源利用率
        \item 最小化能耗
        \item 保证服务等级协议(SLA)
    \end{itemize}

    \textbf{调度策略}:
    \begin{enumerate}
        \item 基于优先级的调度
        \item 基于公平性的调度
        \item 基于预测的调度
        \item 基于强化学习的调度
    \end{enumerate}
\end{example}

\section*{本章小结}

本章介绍了任务规划在通信领域的应用。信道分配和功率控制是无线网络的基本优化问题。无线传感器网络需要考虑能量约束。卫星通信和数据中心调度是大规模资源分配问题。

\section*{习题}

\begin{enumerate}
    \item 将一个简单的信道分配问题建模为图着色问题并求解。

    \item 设计一个能量感知的路由算法,分析其对网络寿命的影响。

    \item 对于示例\ref{ex:satellite-imaging},用整数规划建模并求解一个小规模实例。

    \item 比较不同云计算调度策略的优缺点。

    \item 设计一个物联网数据采集调度算法,考虑能量和时延约束。
\end{enumerate}
