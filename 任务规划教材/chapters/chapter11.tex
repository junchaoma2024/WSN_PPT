% 第11章 交通运输规划
\chapter{交通运输规划}
\label{chap:transportation}

本章介绍任务规划在交通运输领域的应用。

\section{车辆路径问题(VRP)}

\subsection{基本VRP模型}

\begin{definition}[车辆路径问题]
    \keyterm{车辆路径问题}(Vehicle Routing Problem,VRP):给定一个配送中心和若干客户点,确定车辆的配送路线,使得所有客户的需求得到满足,同时优化某个目标(如总行驶距离最短)。
\end{definition}

VRP可以形式化为:
\begin{align}
    \min \quad & \sum_{i,j,k} c_{ij} x_{ijk} \\
    \text{s.t.} \quad & \sum_{k} \sum_{j} x_{ijk} = 1, \quad \forall i \text{(每个客户恰好被访问一次)} \\
    & \sum_{i} d_i \sum_{j} x_{ijk} \leq Q_k, \quad \forall k \text{(车辆容量约束)} \\
    & x_{ijk} \in \{0,1\}
\end{align}

\subsection{带时间窗的VRP}

\begin{definition}[VRPTW]
    \keyterm{带时间窗的VRP}(VRP with Time Windows)要求车辆在客户指定的时间范围内到达并提供服务。
\end{definition}

\subsection{带容量约束的VRP}

\keyterm{CVRP}(Capacitated VRP)考虑车辆的载重限制。

\section{求解方法}

\subsection{精确算法}

\begin{itemize}
    \item 分支定界法
    \item 分支切割法
    \item 列生成法
\end{itemize}

精确算法能够保证找到最优解,但对于大规模问题计算时间过长。

\subsection{启发式与元启发式}

\begin{itemize}
    \item \textbf{构造启发式}:最近邻、节约算法、插入法
    \item \textbf{改进启发式}:2-opt、3-opt、Or-opt
    \item \textbf{元启发式}:遗传算法、模拟退火、禁忌搜索、蚁群优化
\end{itemize}

\subsection{混合智能方法}

结合精确方法和启发式的优点,如:
\begin{itemize}
    \item 大规模邻域搜索(LNS)
    \item 自适应大规模邻域搜索(ALNS)
    \item 混合遗传算法
\end{itemize}

\begin{example}[快递配送路径优化]
\label{ex:delivery}
    某快递公司早班需要配送200个包裹到城区各处。

    \textbf{已知条件}:
    \begin{itemize}
        \item 配送中心位置
        \item 200个配送点的位置和包裹重量
        \item 10辆配送车,每辆载重500kg
        \item 部分配送点有时间要求
    \end{itemize}

    \textbf{优化目标}:最小化总行驶距离

    \textbf{求解过程}:
    \begin{enumerate}
        \item 使用聚类算法初步划分区域
        \item 为每个区域用节约算法构造初始路线
        \item 使用ALNS进行改进
        \item 检验时间窗约束,调整路线
    \end{enumerate}

    \textbf{优化结果}:相比人工规划,总行驶距离减少15\%,配送时间缩短20\%。
\end{example}

\begin{example}[公交线路规划]
\label{ex:bus}
    某城市需要设计新的公交线网。

    \textbf{输入数据}:
    \begin{itemize}
        \item 城市道路网络
        \item 居民出行OD矩阵
        \item 现有公交站点
        \item 车辆配置
    \end{itemize}

    \textbf{规划内容}:
    \begin{enumerate}
        \item 线路走向设计
        \item 站点设置
        \item 发车频率确定
        \item 车辆调度计划
    \end{enumerate}

    \textbf{优化目标}:最大化乘客覆盖率,最小化总运营成本。
\end{example}

\section{动态路径规划}

\subsection{实时重规划}

在实际运营中,经常需要根据实时情况调整路线:
\begin{itemize}
    \item 新订单插入
    \item 交通拥堵
    \item 车辆故障
    \item 客户取消
\end{itemize}

\subsection{交通流预测}

利用历史数据和实时数据预测交通流,提前调整路线。

\begin{example}[网约车调度系统]
\label{ex:ridesharing}
    网约车平台需要实时匹配乘客和司机。

    \textbf{核心问题}:
    \begin{itemize}
        \item 订单分配:哪个司机接哪个乘客
        \item 路径规划:如何到达乘客位置和目的地
        \item 拼车匹配:多个乘客如何共乘
        \item 供需调度:如何引导空闲车辆
    \end{itemize}

    \textbf{算法特点}:
    \begin{itemize}
        \item 毫秒级响应要求
        \item 考虑未来需求预测
        \item 平衡效率和公平性
    \end{itemize}
\end{example}

\section{多式联运规划}

\begin{example}[集装箱多式联运优化]
\label{ex:intermodal}
    某物流公司需要将货物从上海港运送到成都仓库。

    \textbf{可选方式}:
    \begin{itemize}
        \item 全程公路运输(快但贵)
        \item 铁路+公路(经济但慢)
        \item 水运+铁路+公路(最经济但最慢)
    \end{itemize}

    \textbf{决策因素}:
    \begin{itemize}
        \item 货物时效性要求
        \item 各环节成本
        \item 转运时间和成本
        \item 运力可用性
    \end{itemize}

    \textbf{优化模型}:多目标规划,平衡成本、时间和可靠性。
\end{example}

\section*{本章小结}

本章介绍了任务规划在交通运输领域的应用。VRP是物流配送的核心问题,有多种变体和求解方法。动态路径规划处理实时变化的情况。多式联运规划优化不同运输方式的组合。

\section*{习题}

\begin{enumerate}
    \item 用节约算法为一个10客户的VRP问题构造初始解。

    \item 实现2-opt改进算法,改进TSP问题的初始解。

    \item 对于示例\ref{ex:delivery},讨论如何处理配送过程中的新订单。

    \item 比较遗传算法和禁忌搜索在VRP问题上的性能。

    \item 设计一个简单的网约车匹配算法,考虑等待时间和绕路距离。
\end{enumerate}
