% 第16章 规划系统的可解释性与安全性
\chapter{规划系统的可解释性与安全性 \grad}
\label{chap:safety}

本章讨论规划系统的可解释性和安全性问题,属于研究生拓展内容。

\section{可解释规划}

\subsection{计划解释生成}

\begin{definition}[计划解释]
    \keyterm{计划解释}是向用户说明为什么规划器生成了特定的计划,以及为什么没有选择其他方案。
\end{definition}

解释类型:
\begin{itemize}
    \item \textbf{对比解释}:为什么选择动作A而不是B?
    \item \textbf{因果解释}:这个动作导致了什么结果?
    \item \textbf{目标解释}:这个动作如何帮助实现目标?
\end{itemize}

\subsection{人机协同规划}

人机协同规划中,系统需要:
\begin{itemize}
    \item 理解人类的意图和偏好
    \item 解释自己的决策
    \item 接受人类的反馈和修正
    \item 保持适当的自主性
\end{itemize}

\begin{definition}[混合主动规划]
    \keyterm{混合主动规划}(Mixed-Initiative Planning)中,人类和AI系统共同参与规划过程,各自贡献专长。
\end{definition}

\section{规划系统验证}

\subsection{形式化验证方法}

\begin{definition}[形式化验证]
    \keyterm{形式化验证}使用数学方法证明规划系统满足特定性质。
\end{definition}

验证目标包括:
\begin{itemize}
    \item \textbf{正确性}:计划能够达到目标
    \item \textbf{完整性}:如果解存在,系统能找到
    \item \textbf{最优性}:找到的解是最优的
    \item \textbf{安全性}:执行过程中不会进入危险状态
\end{itemize}

\subsection{模型检测}

\begin{definition}[模型检测]
    \keyterm{模型检测}(Model Checking)自动验证系统模型是否满足给定的时态逻辑性质。
\end{definition}

在规划中的应用:
\begin{itemize}
    \item 验证计划满足安全约束
    \item 检测死锁和活锁
    \item 验证时序性质
\end{itemize}

\section{安全关键系统规划}

\subsection{安全约束规划}

在安全关键系统中,规划必须满足硬性安全约束:
\begin{itemize}
    \item 机器人必须避免碰撞
    \item 车辆必须遵守交通规则
    \item 医疗系统必须避免有害操作
\end{itemize}

\begin{definition}[安全规划]
    \keyterm{安全规划}生成的计划保证在执行过程中永远不会进入不安全状态。
\end{definition}

方法:
\begin{itemize}
    \item 将安全约束编码到规划问题中
    \item 使用屏障函数保证安全
    \item 在线监控和干预
\end{itemize}

\subsection{鲁棒性分析}

分析规划系统在以下情况下的表现:
\begin{itemize}
    \item 模型不准确
    \item 执行误差
    \item 环境变化
    \item 对抗性干扰
\end{itemize}

\section{伦理与法规}

\subsection{自主系统伦理}

自主规划系统面临的伦理问题:
\begin{itemize}
    \item \textbf{责任归属}:当自主系统造成损害时,谁负责?
    \item \textbf{透明度}:用户是否知道系统如何做决策?
    \item \textbf{公平性}:决策是否存在偏见?
    \item \textbf{自主性边界}:系统何时应该请求人类干预?
\end{itemize}

\subsection{军事AI治理}

军事AI规划系统的特殊考虑:
\begin{itemize}
    \item \textbf{人类控制}:致命性决策必须有人类参与
    \item \textbf{国际法遵从}:遵守武装冲突法
    \item \textbf{责任链}:明确指挥和责任关系
    \item \textbf{可预测性}:行为应当可预测和可解释
\end{itemize}

\begin{example}[自主驾驶的伦理困境]
    自动驾驶车辆面临不可避免的碰撞时,应该如何决策?

    \textbf{场景}:车辆前方突然出现行人,左侧是护栏,右侧是其他行人。

    \textbf{伦理问题}:
    \begin{itemize}
        \item 是否可以"选择"伤害谁?
        \item 是否应该保护乘客优先?
        \item 如何量化不同选择的"代价"?
    \end{itemize}

    \textbf{技术考虑}:
    \begin{itemize}
        \item 规划系统如何表示这类约束?
        \item 决策过程如何做到可解释?
        \item 如何满足法规要求?
    \end{itemize}
\end{example}

\section*{本章小结}

本章讨论了规划系统的可解释性和安全性。可解释规划帮助用户理解和信任系统决策。形式化验证和模型检测为规划系统提供正确性保证。安全关键系统需要特别的规划方法。伦理和法规问题是自主系统部署的重要考虑。

\section*{习题}

\begin{enumerate}
    \item 为一个简单的规划问题设计解释生成方法。

    \item 讨论人机协同规划中的信任问题。

    \item 用模型检测验证一个简单规划系统的安全性质。

    \item 分析自动驾驶规划系统的伦理设计原则。

    \item 讨论军事AI规划系统应该遵循的原则。
\end{enumerate}
