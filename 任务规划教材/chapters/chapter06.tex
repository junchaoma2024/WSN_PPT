% 第6章 层次任务网络规划
\chapter{层次任务网络规划}
\label{chap:htn}

\keyterm{层次任务网络}(Hierarchical Task Network,HTN)规划是一种通过任务分解来求解规划问题的方法。

\section{HTN基本概念}

\subsection{任务与方法}

\begin{definition}[任务]
    在HTN规划中,\keyterm{任务}分为两类:
    \begin{itemize}
        \item \textbf{原子任务}(Primitive Task):可以直接执行的动作
        \item \textbf{复合任务}(Compound Task):需要进一步分解的抽象任务
    \end{itemize}
\end{definition}

\begin{definition}[方法]
    \keyterm{方法}(Method)定义了如何将复合任务分解为子任务序列。一个方法$m$包含:
    \begin{itemize}
        \item 任务头:要分解的复合任务
        \item 前提条件:方法适用的条件
        \item 任务网络:分解后的子任务及其约束
    \end{itemize}
\end{definition}

\subsection{任务分解}

HTN规划的核心思想是递归分解:从初始任务网络开始,不断选择复合任务并应用方法进行分解,直到所有任务都是原子任务。

\section{SHOP规划系统}

\keyterm{SHOP}(Simple Hierarchical Ordered Planner)是一个广泛使用的HTN规划系统。

\subsection{SHOP算法}

\begin{algorithm}[H]
    \caption{SHOP算法}
    \label{alg:shop}
    \KwIn{当前状态 $s$,任务列表 $T$}
    \KwOut{规划 $\pi$ 或失败}

    \If{$T$ 为空}{
        \KwRet{空规划 $\langle\rangle$}
    }
    $t \gets T$ 的第一个任务\;
    \eIf{$t$ 是原子任务}{
        \If{$t$ 的前提条件在 $s$ 中满足}{
            $s' \gets$ 执行 $t$ 后的状态\;
            $\pi \gets \text{SHOP}(s', T \setminus \{t\})$\;
            \If{$\pi \neq$ 失败}{
                \KwRet{$\langle t \rangle \cdot \pi$}
            }
        }
    }{
        \ForEach{方法 $m$ 可以分解 $t$}{
            \If{$m$ 的前提条件在 $s$ 中满足}{
                $T' \gets$ 用 $m$ 的子任务替换 $T$ 中的 $t$\;
                $\pi \gets \text{SHOP}(s, T')$\;
                \If{$\pi \neq$ 失败}{
                    \KwRet{$\pi$}
                }
            }
        }
    }
    \KwRet{失败}
\end{algorithm}

\subsection{SHOP2扩展}

SHOP2是SHOP的扩展版本,支持:
\begin{itemize}
    \item 偏序任务网络
    \item 分支和循环
    \item 数值计算
    \item 外部函数调用
\end{itemize}

\section{HDDL表示语言}

\keyterm{HDDL}(Hierarchical Domain Definition Language)是HTN规划的标准表示语言。

\subsection{全序与偏序HTN}

\begin{itemize}
    \item \textbf{全序HTN}:子任务之间有严格的顺序约束
    \item \textbf{偏序HTN}:子任务之间只有部分顺序约束
\end{itemize}

\subsection{HDDL语法}

\begin{lstlisting}[language=PDDL]
(:method deliver-package
  :parameters (?p - package ?from ?to - location)
  :task (deliver ?p ?to)
  :precondition (at ?p ?from)
  :ordered-subtasks (and
    (pick-up ?p ?from)
    (transport ?p ?from ?to)
    (put-down ?p ?to)
  )
)
\end{lstlisting}

\section{HTN应用案例}

\begin{example}[军事作战任务分解]
\label{ex:military-htn}
    考虑一个空中打击任务的层次分解:

    \textbf{顶层任务}:执行空中打击任务

    \textbf{分解方法}:
    \begin{enumerate}
        \item 任务规划阶段
        \begin{itemize}
            \item 情报收集与分析
            \item 目标确认
            \item 航线规划
        \end{itemize}
        \item 任务准备阶段
        \begin{itemize}
            \item 装备检查
            \item 燃料加注
            \item 弹药装载
        \end{itemize}
        \item 任务执行阶段
        \begin{itemize}
            \item 起飞
            \item 航渡
            \item 目标攻击
            \item 返航
        \end{itemize}
        \item 任务评估阶段
        \begin{itemize}
            \item 战果评估
            \item 任务总结
        \end{itemize}
    \end{enumerate}
\end{example}

\begin{example}[智能家居任务规划]
\label{ex:smarthome-htn}
    智能家居系统收到用户指令:"准备晚餐派对"

    \textbf{任务分解}:
    \begin{enumerate}
        \item 环境准备
        \begin{itemize}
            \item 调节室温至22度
            \item 调暗灯光
            \item 播放背景音乐
        \end{itemize}
        \item 餐饮准备
        \begin{itemize}
            \item 预热烤箱
            \item 准备食材(通知用户)
            \item 设置定时器
        \end{itemize}
        \item 安全检查
        \begin{itemize}
            \item 检查门锁状态
            \item 启动访客模式
        \end{itemize}
    \end{enumerate}
\end{example}

\begin{example}[游戏AI行为规划]
\label{ex:game-htn}
    在策略游戏中,AI需要规划"建造军事基地":

    \textbf{方法1}(资源充足时):
    \begin{enumerate}
        \item 选择建造地点
        \item 派遣工人
        \item 建造兵营
        \item 建造防御塔
        \item 训练士兵
    \end{enumerate}

    \textbf{方法2}(资源不足时):
    \begin{enumerate}
        \item 采集资源
        \item 递归调用"建造军事基地"
    \end{enumerate}
\end{example}

\section*{本章小结}

HTN规划通过任务分解的方式求解规划问题,更符合人类的规划思维方式。SHOP系列是最著名的HTN规划系统,HDDL是HTN规划的标准表示语言。HTN规划在军事、智能家居、游戏AI等领域有广泛应用。

\section*{习题}

\begin{enumerate}
    \item 设计一个HTN领域,对"组织一次旅行"任务进行分解。

    \item 比较HTN规划和经典规划的优缺点。

    \item 用HDDL编写示例\ref{ex:smarthome-htn}中的智能家居任务规划领域。

    \item 分析HTN规划的完备性:在什么条件下HTN规划能够找到解?

    \item 设计一个HTN规划问题,使得全序和偏序分解产生不同的结果。
\end{enumerate}
