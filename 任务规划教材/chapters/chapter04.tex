% 第4章 启发式规划方法
\chapter{启发式规划方法}
\label{chap:heuristic}

启发式方法是现代规划器成功的关键。本章介绍规划中常用的启发式函数设计方法。

\section{松弛启发式}

\keyterm{松弛}(Relaxation)是设计启发式的一种通用方法,通过简化原问题来获得代价的下界估计。

\subsection{删除松弛}

\begin{definition}[删除松弛]
    \keyterm{删除松弛}(Delete Relaxation)是指忽略所有动作的删除效果。在删除松弛下,一旦某个命题变为真,它将永远保持为真。
\end{definition}

\subsection{$h^{add}$与$h^{max}$启发式}

\begin{definition}[$h^{max}$启发式]
    $h^{max}$启发式估计达到目标集合中"最难"达到的子目标的代价:
    \begin{equation}
        h^{max}(s) = \max_{p \in g} h^{max}_p(s)
    \end{equation}
    其中$h^{max}_p(s)$是从状态$s$达到命题$p$的估计代价。
\end{definition}

\begin{definition}[$h^{add}$启发式]
    $h^{add}$启发式假设各子目标独立,将达到各子目标的代价相加:
    \begin{equation}
        h^{add}(s) = \sum_{p \in g} h^{add}_p(s)
    \end{equation}
\end{definition}

$h^{max}$是可采纳的但信息量较少;$h^{add}$信息量丰富但通常高估真实代价。

\subsection{$h^{FF}$启发式}

\keyterm{FF启发式}由Hoffmann和Nebel提出,是最成功的规划启发式之一。它基于松弛规划图,计算从当前状态到目标的松弛计划长度。

\begin{algorithm}[H]
    \caption{FF启发式计算}
    \label{alg:hff}
    \KwIn{当前状态 $s$,目标 $g$}
    \KwOut{启发式值 $h^{FF}(s)$}

    构建松弛规划图直到$g$中所有命题出现\;
    \If{$g$中存在命题从未出现}{
        \KwRet{$\infty$}
    }
    从最后一层开始,使用贪心方法提取松弛计划\;
    \KwRet{松弛计划中的动作数量}
\end{algorithm}

\section{抽象启发式}

\keyterm{抽象}(Abstraction)通过将多个状态映射到同一抽象状态来简化问题。

\subsection{模式数据库}

\begin{definition}[模式数据库]
    \keyterm{模式数据库}(Pattern Database,PDB)预先计算抽象空间中所有状态到目标的精确代价,然后在搜索时作为启发式查表使用。
\end{definition}

\subsection{合并与收缩}

\keyterm{合并与收缩}(Merge-and-Shrink)是一种系统化构建抽象的方法,通过合并变量和收缩状态空间来控制抽象的大小。

\section{地标启发式}

\keyterm{地标}(Landmark)是在任何解中都必须在某个时刻为真的命题或必须执行的动作。

\subsection{地标识别}

识别地标的方法包括:
\begin{itemize}
    \item 基于松弛规划图的方法
    \item 基于回归的方法
    \item 基于SAT/CSP的方法
\end{itemize}

\subsection{地标计数启发式}

\begin{definition}[地标计数启发式]
    地标计数启发式$h^{LM}$估计从当前状态到目标还需要达成的地标数量:
    \begin{equation}
        h^{LM}(s) = |\{l \in L : l \text{ 在 } s \text{ 中未被满足且尚未达成}\}|
    \end{equation}
\end{definition}

\section{现代规划器架构}

\subsection{Fast Downward系统}

\keyterm{Fast Downward}是最成功的现代规划系统之一,由Helmert于2006年提出。其主要特点包括:
\begin{itemize}
    \item 将PDDL转换为多值状态变量表示(SAS+)
    \item 支持多种启发式函数
    \item 支持多种搜索算法
    \item 模块化设计,易于扩展
\end{itemize}

\subsection{LAMA规划器}

\keyterm{LAMA}(Landmarks, Actions, and Multi-heuristics Anytime)规划器结合了地标启发式和FF启发式,在国际规划竞赛中表现优异。

\begin{example}[国际规划竞赛问题求解]
\label{ex:ipc}
    国际规划竞赛(IPC)是规划领域的重要评测平台。以下是LAMA在IPC 2011物流领域问题上的表现:

    \begin{center}
    \begin{tabular}{lrrr}
        \toprule
        问题规模 & Fast Downward & LAMA & 最优规划器 \\
        \midrule
        小(<100状态) & 0.1s & 0.1s & 0.5s \\
        中(<10000状态) & 2.3s & 1.8s & 超时 \\
        大(>100000状态) & 45s & 32s & 超时 \\
        \bottomrule
    \end{tabular}
    \end{center}

    这说明启发式搜索规划器在大规模问题上具有显著优势。
\end{example}

\section*{本章小结}

本章介绍了规划中的启发式方法。松弛启发式通过简化问题获得代价估计,其中$h^{FF}$是最成功的启发式之一。抽象启发式通过状态聚合来简化问题。地标启发式利用问题的结构信息。现代规划器如Fast Downward和LAMA结合了多种技术,在实际问题中表现出色。

\section*{习题}

\begin{enumerate}
    \item 对于积木世界问题,计算$h^{add}$和$h^{max}$启发式的值,并比较它们与真实代价的关系。

    \item 证明$h^{max}$是可采纳的启发式。

    \item 为八数码问题设计一个模式数据库,讨论如何选择模式以获得信息丰富的启发式。

    \item 在一个物流问题中,识别可能的地标,并解释为什么它们必须在任何解中出现。

    \item 下载并安装Fast Downward,使用不同的启发式配置求解IPC基准问题,比较性能差异。
\end{enumerate}
