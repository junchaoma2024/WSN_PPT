% 第7章 约束满足与规划
\chapter{约束满足与规划}
\label{chap:csp}

本章介绍约束满足问题及其在规划中的应用。

\section{约束满足问题}

\subsection{CSP形式化}

\begin{definition}[约束满足问题]
    \keyterm{约束满足问题}(Constraint Satisfaction Problem,CSP)定义为三元组 $\langle X, D, C \rangle$:
    \begin{itemize}
        \item $X = \{x_1, x_2, \ldots, x_n\}$ 是变量集合
        \item $D = \{D_1, D_2, \ldots, D_n\}$ 是值域集合,$D_i$ 是 $x_i$ 的可能取值
        \item $C = \{c_1, c_2, \ldots, c_m\}$ 是约束集合
    \end{itemize}
\end{definition}

\subsection{约束传播}

\keyterm{约束传播}(Constraint Propagation)通过推理减少变量的值域。常用技术包括:
\begin{itemize}
    \item \textbf{节点一致性}:确保每个变量的值域满足一元约束
    \item \textbf{弧一致性}:确保每对变量满足二元约束
    \item \textbf{路径一致性}:确保任意三个变量的组合一致
\end{itemize}

\subsection{回溯搜索}

\begin{algorithm}[H]
    \caption{CSP回溯搜索}
    \label{alg:csp-backtrack}
    \KwIn{CSP问题 $\langle X, D, C \rangle$,部分赋值 $\sigma$}
    \KwOut{完整解或失败}

    \If{$\sigma$ 是完整赋值}{
        \KwRet{$\sigma$}
    }
    选择一个未赋值变量 $x$\;
    \ForEach{$v \in D_x$ 按某种顺序}{
        \If{$\sigma \cup \{x = v\}$ 满足所有约束}{
            $\text{result} \gets \text{Backtrack}(\sigma \cup \{x = v\})$\;
            \If{$\text{result} \neq$ 失败}{
                \KwRet{$\text{result}$}
            }
        }
    }
    \KwRet{失败}
\end{algorithm}

\section{规划问题的SAT编码}

\subsection{命题逻辑基础}

命题逻辑的基本概念:
\begin{itemize}
    \item \textbf{命题变量}:取值为真或假的变量
    \item \textbf{文字}:命题变量或其否定
    \item \textbf{子句}:文字的析取
    \item \textbf{CNF}:子句的合取
\end{itemize}

\subsection{规划到SAT的转换}

将规划问题编码为SAT的关键思想是引入时间步:
\begin{itemize}
    \item $p_t$:命题$p$在时刻$t$为真
    \item $a_t$:动作$a$在时刻$t$执行
\end{itemize}

\textbf{编码规则}:
\begin{enumerate}
    \item 初始状态:$\bigwedge_{p \in s_0} p_0 \land \bigwedge_{p \notin s_0} \neg p_0$
    \item 目标状态:$\bigwedge_{p \in g} p_T$
    \item 动作前提:$a_t \rightarrow \bigwedge_{p \in \text{Pre}(a)} p_t$
    \item 动作效果:$a_t \rightarrow \bigwedge_{p \in \text{Add}(a)} p_{t+1}$
    \item 帧公理:$(p_t \land \neg p_{t+1}) \rightarrow \bigvee_{a: p \in \text{Del}(a)} a_t$
\end{enumerate}

\subsection{SAT求解器应用}

\begin{example}[布尔可满足性规划]
\label{ex:sat-planning}
    考虑一个简单的积木问题,初始状态A在桌上,B在A上;目标是B在桌上,A在B上。

    \textbf{SAT编码}(假设最多2步):

    命题变量:
    \begin{itemize}
        \item $\text{on}(A,B)_t$, $\text{on}(B,A)_t$:积木位置
        \item $\text{ontable}(A)_t$, $\text{ontable}(B)_t$:在桌上
        \item $\text{unstack}(B,A)_t$, $\text{stack}(A,B)_t$, ...:动作
    \end{itemize}

    初始状态子句:
    \begin{align}
        &\text{ontable}(A)_0 \\
        &\text{on}(B,A)_0 \\
        &\neg\text{on}(A,B)_0 \\
        &\neg\text{ontable}(B)_0
    \end{align}

    目标子句:
    \begin{align}
        &\text{ontable}(B)_2 \\
        &\text{on}(A,B)_2
    \end{align}

    SAT求解器返回满足赋值,从中提取动作序列。
\end{example}

\section{规划问题的SMT编码}

\subsection{SMT理论}

\keyterm{SMT}(Satisfiability Modulo Theories)扩展了SAT,支持更丰富的理论:
\begin{itemize}
    \item 线性算术
    \item 数组理论
    \item 位向量
    \item 非线性算术
\end{itemize}

\subsection{时态规划的SMT方法}

SMT特别适合时态规划,因为可以直接处理连续时间变量和数值约束。

\section*{本章小结}

本章介绍了约束满足与规划的关系。CSP是一种强大的建模语言,规划问题可以编码为SAT或SMT问题。现代SAT/SMT求解器的强大能力使得这种方法在实践中非常有效。

\section*{习题}

\begin{enumerate}
    \item 将八皇后问题建模为CSP,并用回溯搜索求解。

    \item 为示例\ref{ex:sat-planning}中的积木问题写出完整的SAT编码。

    \item 比较SAT规划和启发式搜索规划的优缺点。

    \item 设计一个需要数值约束的规划问题,说明为什么SMT比SAT更适合。

    \item 实现一个简单的SAT规划器,测试其在小规模问题上的性能。
\end{enumerate}
